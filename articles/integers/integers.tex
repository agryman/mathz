\documentclass[11pt, oneside]{article}

\usepackage{preamble}
\addbibresource{../../shared/references.bib}

\usepackage{sets}
\usepackage{groups}
\usepackage{integers}

\title{Integers}
\author{Arthur Ryman, {\tt arthur.ryman@gmail.com}}
\date{\today}

\begin{document}

\maketitle

\begin{abstract}
This article contains Z Notation type declarations for the integers, $\num$, and some related objects.
It has been type checked by \fuzz.
\end{abstract}

\tableofcontents

\section{Introduction}

The integers, $\num$, are built-in to Z Notation.
This article provides type declarations for some related objects so that they can be used and type checked in formal Z specifications.

\section{Divisibility and Prime Numbers}

\subsection{Divisibility}

This section specifies divisibility of integers.

\subsubsection{$Divides$}

Given integers $x$ and $y$ we say the $x$ \textit{divides} $y$ if there is some integer $q$ such 
that $q x = y$.
Let the schema $Divides$ denote this situation.

\begin{schema}{Divides}
	x, y, q : \num
\where
	q * x = y
\end{schema}
\begin{itemize}
	\item $y$ is a multiple of $x$.
\end{itemize}

\subsubsection{$divides$}

Let $divides$ denote the divisibility relation between integers
where $(x,y) \in divides$ means that $x$ divides $y$.

\begin{axdef}
	divides : \num \rel \num
\where
	divides = \{~ Divides @ x \mapsto y ~\}
\end{axdef}

\subsubsection{\zcmd{divides}}

We introduce the usual notation $x \divides y$ to denote that $(x, y) \in divides$.
\begin{zed}
(\_ \divides \_) == divides
\end{zed}

\begin{example}
The integer $7$ divides $42$ because $6 * 7 = 42$.

\begin{zed}
	7 \divides 42
\end{zed}
\end{example}

\begin{remark}
Every integer $x$ divides $0$ because $0 * x = 0$.

\begin{zed}
	\forall x : \num @ x \divides 0
\end{zed}
\end{remark}

\subsubsection{$Divisor$}

Let $x$ be a nonzero integer that divides the integer $y$.
We say that $x$ is a \textit{divisor} of $y$.
Let the schema $Divisor$ denote this situation.

\begin{schema}{Divisor}
	x, y : \num
\where
	x \neq 0
\also
	x \divides y
\end{schema}
\begin{itemize}
	\item $x$ is nonzero.
	\item $x$ divides $y$.
\end{itemize}

\subsubsection{$divisors$}

Let the set $divisors(y)$ denote the set of all divisors of the integer $y$.

\begin{axdef}
	divisors : \num \fun \power \num
\where
	\forall y : \num @ \\
	\t1	divisors(y) = \{~ x : \num | Divisor ~\}
\end{axdef}

\begin{example}
The integer $6$ has the following divisors.

\begin{zed}
	divisors(6) = \{-6, -3, -2, -1, 1, 2, 3, 6 \}
\end{zed}
\end{example}

\subsubsection{$positive\_divisors$}

Let the set $positive\_divisors(y)$ denote the set of all positive divisors of the integer $y$.

\begin{axdef}
	positive\_divisors : \num \fun \power \nat_1
\where
	\forall y : \num @ \\
	\t1	positive\_divisors(y) = divisors(y) \cap \nat_1
\end{axdef}

\begin{example}
The integer $6$ has the following positive divisors.

\begin{zed}
	positive\_divisors(6) = \{1, 2, 3, 6 \}
\end{zed}
\end{example}

\subsection{Prime Numbers}

\subsubsection{$Prime$}

An integer $p$ is \textit{prime} if it is greater than one 
and only has one and itself as positive divisors.
Let the schema $Prime$ denote this situation.

\begin{schema}{Prime}
	p: \nat
\where
	p > 1
\also
	positive\_divisors(p) = \{ 1, p \}
\end{schema}
\begin{itemize}
	\item $p$ is greater than $1$.
	\item $1$ and $p$ are the only positive divisors of $p$.
\end{itemize}

\begin{example}
The integer $2$ is prime.

\begin{zed}
	\LET p == 2 @ Prime
\end{zed}
\end{example}

\subsubsection{$primes$}

Let $primes$ denote the set of all primes.

\begin{axdef}
	primes : \power \nat_1
\where
	primes = \{~ Prime @ p ~\}
\end{axdef}

\begin{example}
The natural numbers $2, 3, 5,$ and $7$ are primes.

\begin{zed}
	\{2, 3, 5, 7\} \subseteq primes
\end{zed}
\end{example}

\section{Integer Sequences}

\subsection{Addition of Integer Sequences}

\subsubsection{$AddIntegerSequences$}

Let $l$ be a natural number and
let $x$ and $y$ be two integer sequences of length $l$.
Their sum $z = x + y$ is the integer sequence of length $l$ defined by pointwise addition
of the terms in $x$ and $y$.
Let the schema $AddIntegerSequences$ denote this situation.

\begin{schema}{AddIntegerSequences}
	l : \nat \\
	x, y, z : \seq \num
\where
	l = \# x = \# y
\also
	z = (\lambda i : 1 \upto l @ x~i + y~i)
\end{schema}
\begin{itemize}
	\item The sequence $z$ is defined by pointwise addition of the sequences $x$ and $y$.
\end{itemize}

\subsubsection{$add\_int\_seq$}

Let the function $add\_int\_seq(x, y) = z$ be the sum of two equal-length integer sequences.

\begin{axdef}
	add\_int\_seq : \seq \num \cross \seq \num \pfun \seq \num
\where
	add\_int\_seq = \{~ AddIntegerSequences @ (x, y) \mapsto z ~\}
\end{axdef}

\subsubsection{\zcmd{addSeqZ}}

We introduce the notation $x \addSeqZ y = add\_int\_seq(x, y)$.

\begin{zed}
	(\_ \addSeqZ \_) == add\_int\_seq
\end{zed}

\printbibliography

\end{document}