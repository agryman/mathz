\documentclass[11pt, oneside]{article}

\usepackage{preamble}
\addbibresource{../../shared/references.bib}

\usepackage{sets}
\usepackage{groups}
\usepackage{integers}

\title{Integers}
\author{Arthur Ryman, {\tt arthur.ryman@gmail.com}}
\date{\today}

\begin{document}

\maketitle

\begin{abstract}
This article contains Z Notation type declarations for the integers, $\num$, and some related objects.
It has been type checked by \fuzz.
\end{abstract}

\section{Introduction}

The integers, $\num$, are built-in to Z Notation.
This article provides type declarations for some related objects so that they can be used and type checked in formal Z specifications.

\section{Integers}

\subsection{$AddIntegerSequences$}

Let $l$ be a natural number and
let $x$ and $y$ be two integer sequences of length $l$.
Their sum $z = x + y$ is the integer sequence of length $l$ defined by point-wise addition of 
of the terms in $x$ and $y$.
Let the schema $AddIntegerSequences$ denote this situation.

\begin{schema}{AddIntegerSequences}
	l : \nat \\
	x, y, z : \seq \num
\where
	l = \# x = \# y
\also
	z = (\lambda i : 1 \upto l @ x~i + y~i)
\end{schema}
\begin{itemize}
	\item The sequence $z$ is defined by pointwise addition of the sequences $x$ and $y$.
\end{itemize}

\subsection{$add\_int\_seq$}

Let the function $add\_int\_seq(x, y) = z$ be the sum of two equal-length integer sequences.

\begin{zed}
	add\_int\_seq == \{~ AddIntegerSequences @ (x, y) \mapsto z ~\}
\end{zed}

\subsection{\zcmd{addSeqZ}}

We introduce the notation $x \addSeqZ y = add\_int\_seq(x, y)$.

\begin{zed}
	(\_ \addSeqZ \_) == add\_int\_seq
\end{zed}

\printbibliography

\end{document}