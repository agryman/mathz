\documentclass[11pt, oneside]{article}

\usepackage{../../shared/preamble}
\addbibresource{../../shared/references.bib}

\usepackage{sets}

\title{Sets}
\author{Arthur Ryman, {\tt arthur.ryman@gmail.com}}
\date{\today}

\begin{document}

\maketitle

\begin{abstract}
This article contains Z Notation type declarations for concepts related to sets.
It has been type checked by \fuzz.
\end{abstract}

\section{Introduction}

Typed set theory forms the mathematical foundation of Z Notation
and many concepts relating to set theory are defined by its built-in mathematical tool-kit. 
This articles augments the tool-kit with some additional concepts.

\section{Arbitrary Sets}

\subsection{\zcmd{setT}, \zcmd{setU}, \dots, \zcmd{setZ}}

Let $\setT$, $\setU$, and $\setZ$ denote arbitrary sets.
These will be used throughout in the statement of theorems, remarks, and examples that are parameterized
by arbitrary sets.

\begin{zed}
	[\setT, \setU, \setV, \setW, \setX, \setY, \setZ]
\end{zed}

\section{Formal Arguments to Generic Constructions}

The following typographically distinctive symbols will be used as formal arguments to generic constructions:
$\genT, \genU, \genV, \genW, \genX, \genY, \genZ$. 
They denote arbitrary sets.

\section{Families}

\subsection{\zcmd{family}}

Let $\genT$ be a set.
A {\it family} of subsets of $\genT$ is a set of subsets of $\genT$.
Let $\family \genT$ denote the set of all families of subsets of $X$.

\begin{zed}
	\family \genT == \power(\power \genT)
\end{zed}

\section{Functions}


\subsection{\zcmd{const}}

Let $\genT$ and $\genU$ be sets and let $c \in \genU$ be some given point.
The mapping that sends every point of $\genT$ to $c$ is called the {\it constant mapping} defined by $c$.
Let $\const(c)$ denote the constant mapping.

\begin{gendef}[\genT, \genU]
	\const: \genU \fun (\genT \fun \genU)
\where
	\forall c: \genU @ \\
	\t1	\const(c) = (\lambda x: \genT @ c)
\end{gendef}


\subsection{\zcmd{restrictU}}

Let $\genT$ and $\genU$ be sets, let $f: \genT \fun \genU$, and let $T \subseteq \genT$.
Let $f \restrictU T$ denote the restriction of $f$ to $T$.

\begin{gendef}[\genT, \genU]
	\_ \restrictU \_: (\genT \fun \genU) \cross \power \genT \fun (\genT \pfun \genU)
\where
	\forall f: \genT \fun \genU; T: \power \genT @ \\
	\t1	f \restrictU T = T \dres f
\end{gendef}

\printbibliography

\end{document}