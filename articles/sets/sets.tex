\documentclass[11pt, oneside]{article}

\usepackage{../../shared/preamble}
\addbibresource{../../shared/references.bib}

\usepackage{sets}

\title{Sets}
\author{Arthur Ryman, {\tt arthur.ryman@gmail.com}}
\date{\today}

\begin{document}

\maketitle

\begin{abstract}
This article contains Z Notation type declarations for concepts related to sets.
It has been type checked by \fuzz.
\end{abstract}

\section{Introduction}

Typed set theory forms the mathematical foundation of Z Notation
and many concepts relating to set theory are defined by its built-in mathematical tool-kit. 
The articles augments the tool-kit with some additional concepts.

\section{Arbitrary Sets}

\subsection{\zcmd{setX}, \zcmd{setY}, and \zcmd{setZ}}

Let $\setX$, $\setY$, and $\setZ$ denote arbitrary sets.
These will be used throughout in the statement of theorems, remarks, and examples that are parameterized
by arbitrary sets.

\begin{zed}
	[\setX, \setY, \setZ]
\end{zed}

\section{Families}

\subsection{\zcmd{family}}

Let $X$ be a set.
A {\it family} of subsets of $X$ is a set of subsets of $X$.
Let $\family X$ denote the set of all families of subsets of $X$.

\begin{zed}
	\family X == \power(\power X)
\end{zed}

\section{Functions}


\subsection{\zcmd{const}}

Let $X$ and $Y$ be sets and let $c \in Y$ be some given point.
The mapping that sends every point of $X$ to $c$ is called the {\it constant mapping} defined by $c$.
Let $\const(c)$ denote the constant mapping.

\begin{gendef}[X,Y]
	\const: Y \fun (X \fun Y)
\where
	\forall c: Y @ \\
	\t1	\const(c) = (\lambda x: X @ c)
\end{gendef}


\subsection{\zcmd{restrictU}}

Let $f: X \fun Y$ and let $U \subseteq X$.
Let $f \restrictU U$ denote the restriction of $f$ to $U$.

\begin{gendef}[X,Y]
	\_ \restrictU \_: (X \fun Y) \cross \power X \fun (X \pfun Y)
\where
	\forall f: X \fun Y; U: \power X @ \\
	\t1	f \restrictU U = U \dres f
\end{gendef}

\printbibliography

\end{document}