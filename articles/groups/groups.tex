\documentclass{amsart}

\usepackage{mathz-sets}
\usepackage{mathz-groups}
\usepackage{mathz-preamble}

\addbibresource{../../shared/references.bib}

\begin{document}

\title{Groups}
\author{Arthur Ryman}
\email[Arthur Ryman]{arthur.ryman@gmail.com}
\date{\today}

\begin{abstract}
This article contains Z Notation type declarations for groups and some related objects.
It has been type checked by \fuzz.
\end{abstract}

\maketitle

\tableofcontents

\section{Introduction}

Groups are ubiquitous throughout mathematics and physics.
This article defines the basic algebraic objects related to groups and their homomorphisms.

\section{Binary Operations}

Let $\genT$ be a set. We refer to the members of $\genT$ as its {\em elements}.
A {\em binary operation} on $\genT$ is a function that maps pairs of elements to elements.

\subsection{\zcmd{binop}}

Let $\binop \genT$ denote the set of all binary operations on $\genT$.

\begin{zed}
\binop \genT == \genT \cross \genT \fun \genT
\end{zed}

\subsection{Infix Operator Symbols \zcmd{timesG}, \zcmd{mulG}, and \zcmd{addG}}

The result of applying a binary operation to the pair of elements $(x, y)$ 
is often denoted by an expression formed using an infix operator symbol,
e.g. $x \timesG y$, $x \mulG y$ or $x \addG y$.

\subsection{$MapPerservesOperation$}

Let $\genT$ and $\genU$ be sets and let $A$ and $B$ be binary operations on them.
Let $f$ be a function that maps $\genT$ to $\genU$.
The function $f$ is said to {\em preserve the operations} if it maps the product of elements to 
the product of the mapped elements.

Let $MapPreservesOperation$ denote this situation.

\begin{schema}{MapPreservesOperation}[\genT, \genU]
f: \genT \fun \genU \\
A: \binop \genT \\
B: \binop \genU
\where
\LET (\_ \mulG \_) == A; (\_ \timesG \_) == B @ \\
\t1	\forall x, y: \genT @ \\
\t2		f(x \mulG y) = (f~x) \timesG (f~y)
\end{schema}

\subsection{\zcmd{homBinOp}}

A map that preserves operations is said to be an {\em operation homomorphism}.

Let $A$ and $B$ be binary operations. 
Let $\homBinOp(A,B)$ denote the set of operation homomorphisms from $A$ to $B$.

\begin{gendef}[\genT, \genU]
\homBinOp: \binop \genT \cross \binop \genU  \fun \power (\genT \fun \genU)
\where
\homBinOp = (\lambda A: \binop \genT; B: \binop \genU @ \\
\t1	\{~ f: \genT \fun \genU | MapPreservesOperation[\genT, \genU] ~\})
\end{gendef}

\begin{remark}
The identity map is an operation homomorphism.
\end{remark}

\begin{remark}
The composition of two operation homomorphisms is an operation homomorphism.
\end{remark}

\section{Semigroups}

\subsection{$OperationIsAssociative$}

A binary operation is said to be {\em associative} if the result of applying it to three elements
is independent of the order in which it is applied pairwise.

Let $OperationIsAssociative$ denote this situation.

\begin{schema}{OperationIsAssociative}[\genT]
A: \binop \genT
\where
\LET (\_ \mulG \_) == A @ \\
\t1	\forall x, y, z: \genT @ \\
\t2		(x \mulG y) \mulG z = x \mulG (y \mulG z)
\end{schema}

\subsection{\zcmd{semigroup}}

Let $\semigroup \genT$ denote the set of all semigroups on the set of elements $\genT$.

\begin{zed}
\semigroup \genT == \{~ A: \binop \genT | OperationIsAssociative[\genT] ~\}
\end{zed}

\subsection{\zcmd{homSemigroup}}

A {\em semigroup homomorphism} from $A$ to $B$ is a homomorphism of the underlying binary operation.

Let $\homSemigroup(A, B)$ denote the set of all semigroup homomorphisms from $A$ to $B$.

\begin{gendef}[\genT, \genU]
\homSemigroup: \semigroup \genT \cross \semigroup \genU \fun \power (\genT \pfun \genU)
\where
\homSemigroup = \\
\t1	(\lambda A: \semigroup \genT; B: \semigroup \genU @ \homBinOp(A, B))
\end{gendef}

\begin{remark}
The identity mapping is a semigroup homomorphism.
\end{remark}

\begin{remark}
The composition of two semigroup homomorphisms is another semigroup homomorphism.
\end{remark}

\section{Monoids}

\subsection{$IdentityElement$}

Let $\genT$ be a set, let $A$ be a binary operation over $\genT$, and let $e$ be an element of $\genT$.
The element $e$ is said to be an {\em identity element} of $A$ if left and right 
products with it leave all elements unchanged.

Let $IdentityElement$ denote this situation.

\begin{schema}{IdentityElement}[\genT]
A: \binop \genT \\
e: \genT
\where
\LET (\_ \mulG \_) == A @ \\
\t1	\forall x: \genT @ \\
\t2		e \mulG x = x = x \mulG e
\end{schema}

\subsection{$identity\_element$}

Let $identity\_element$ denote the relation that associates a binary operation one of its identity elements.

\begin{gendef}[\genT]
identity\_element: \binop \genT \rel \genT
\where
identity\_element = \\
\t1	\{~ IdentityElement[\genT] @ A \mapsto e ~\}
\end{gendef}

\begin{remark}
If a binary operation has an identity element then it is unique.
\end{remark}

\begin{proof}
Let $\mulG$ be a binary operation. Suppose $e$ and $e'$ are identity elements.
\begin{argue}
e \\
\t1	= e \mulG e'	& $e'$ is an identity element \\
\t1	= e'			& $e$ is an identity element
\end{argue}
\end{proof}

\begin{remark}
Since identity elements are unique if they exist, the relation from binary operations to identity elements is a partial function.

\begin{zed}
identity\_element \in \binop \setT \pfun \setT
\end{zed}

\end{remark}


\subsection{Identity Element Symbols \zcmd{zeroG}, and \zcmd{oneG}}

Identity elements are typically denoted by the symbols  $\zeroG$ or $\oneG$.

\subsection{\zcmd{monoid}}

Let $\genT$ be a set of elements.
A {\em monoid} over $\genT$ is a semigroup over $\genT$ that has an identity element.

Let $\monoid \genT$ denote the set of all monoids over $\genT$.

\begin{zed}
\monoid \genT == \{~ A: \semigroup \genT | \exists e: \genT @ IdentityElement[\genT] ~\}
\end{zed}


\subsection{$MapPreservesIdentity$}

Let $A$ and $B$ be monoids and let $f$ map the elements of $A$ to the elements of $B$.
The map $f$ is said to {\em preserve the identity element} if it maps the identity element of $A$
to the identity element of $B$.

Let $MapPreservesIdentity$ denote this situation.

\begin{schema}{MapPreservesIdentity}[\genT, \genU]
f: \genT \fun \genU \\
A: \monoid \genT \\
B: \monoid \genU
\where
\LET e == identity\_element~A; \\
\t1	e' == identity\_element~B @ \\
\t2		f~e = e'
\end{schema}

\subsection{\zcmd{homMonoid}}

A {\em monoid homomorphism} from $A$ to $B$ is a homomorphism $f$ of the underlying semigroups
that preserves identity.

Let $\homMonoid(A, B)$ denote the set of all monoid homomorphisms from $A$ to $B$.

\begin{gendef}[\genT, \genU]
\homMonoid: \monoid \genT \cross \monoid \genU \fun \power (\genT \fun \genU)
\where
\homMonoid = \\
\t1	(\lambda A: \monoid \genT; B: \monoid \genU @ \\
\t2		\{~ f: \homSemigroup(A, B) | \\
\t3			MapPreservesIdentity[\genT, \genU] ~\})
\end{gendef}

\begin{remark}
The identity mapping is a monoid homomorphism.
\end{remark}

\begin{remark}
The composition of two monoid homomorphisms is another monoid homomorphism.
\end{remark}

\section{Groups}

\subsection{$InverseOperation$ and Postfix Operator symbol \zcmd{invG}}

Let $\genT$ be a set of elements and let $A$ be a monoid on $\genT$.
A function $inv \in \genT \fun \genT$ is said to be an {\em inverse operation} if it maps each element
to an element whose product with it is the identity element.
Typically, the expression $x \invG$ is used to denote the inverse of $x$.

Let $InverseOperation$ denote this situation.

\begin{schema}{InverseOperation}[\genT]
A: \monoid \genT \\
inv: \genT \fun \genT
\where
\LET (\_ \mulG \_) == A; \\
\t1	\oneG == identity\_element~A; \\
\t1	(\_ \invG) == inv @ \\
\t2		\forall x: \genT @ \\
\t3			x \mulG x \invG = \oneG = x \invG  \mulG x
\end{schema}

\subsection{$inverse\_operation$}

Let $inverse\_operation$ denote the relation between monoids and their inverse operations.

\begin{gendef}[\genT]
inverse\_operation: \monoid \genT \rel \genT \fun \genT
\where
inverse\_operation = \\
\t1	\{~ InverseOperation[\genT] @ A \mapsto inv ~\}
\end{gendef}

\begin{remark}
If a monoid has an inverse operation then it is unique.
\end{remark}

\begin{proof}
Let $x$ be any element.
Suppose $x \invG$ and $x \daggerG$ are inverses of $x$.
\begin{argue}
x\daggerG \\
\t1	= x\daggerG \mulG \oneG				& $\oneG$ is an identity element \\
\t1	= x\daggerG \mulG (x \mulG x \invG)		& $x \invG$ is an inverse \\
\t1	= (x\daggerG \mulG x) \mulG x \invG		& associativity \\
\t1	= \oneG \mulG x \invG				& $x \daggerG$ is an inverse \\
\t1	= x \invG							& $\oneG$ is an identity element
\end{argue}
\end{proof}

\begin{remark}
Since if inverse operation exist they are unique, the relation between monoids and inverse operations
is a partial function.

\begin{zed}
inverse\_operation \in \monoid \setT \pfun \setT \fun \setT
\end{zed}

\end{remark}

\subsection{$\group$}

A {\em group} is a monoid that has an inverse operation.

Let $\genT$ be a set of elements.
Let $\group \genT$ denote the set of all groups over $\genT$.

\begin{zed}
\group \genT == \{~ A: \monoid \genT | \exists inv: \genT \fun \genT @ InverseOperation[\genT] ~\}
\end{zed}

\subsection{$MapPreservesInverse$}

Let $\genT$ and $\genU$ be sets of elements,
let $A$ and $B$ be groups over $\genT$ and $\genU$, 
and let $f$ map $\genT$ to $\genU$.
The map $f$ is said to {\em preserve the inverses} if it maps the inverses of elements of $A$
to the inverses of the corresponding elements of $B$.

Let $MapPreservesInverse$ denote this situation.

\begin{schema}{MapPreservesInverse}[\genT, \genU]
f: \genT \fun \genU \\
A: \group \genT \\
B: \group \genU
\where
\LET (\_ \invG) == inverse\_operation~A; \\
\t1	(\_ \daggerG) == inverse\_operation~B @ \\
\t2		\forall x: \genT @ \\
\t3			f(x \invG) = (f~x) \daggerG
\end{schema}

\subsection{\zcmd{homGroup}}

Let $A$ and $B$ be groups.
A {\em group homomorphism} from $A$ to $B$ is a monoid homomorphism
from $A$ to $B$ that preserves inverses.

Let $\homGroup(A, B)$ denote the set of all group homomorphisms from $A$ to $B$.

\begin{gendef}[\genT, \genU]
\homGroup: \group \genT \cross \group \genU \fun \power (\genT \fun \genU)
\where
\homGroup = \\
\t1	(\lambda A: \group \genT; B: \group \genU @ \\
\t2		\{~ f: \homMonoid(A, B) | \\
\t3			MapPreservesInverse[\genT, \genU] ~\})
\end{gendef}

\begin{remark}
The identity mapping is a group homomorphism.
\end{remark}

\begin{remark}
The composition of two group homomorphisms is another group homomorphism.
\end{remark}

\subsection{$bij$}

Let $\genT$ be a set and let $bij[\genT]$ denote the set of a bijections $\genT \bij \genT$ from $\genT$ to itself.

\begin{gendef}[\genT]
	bij: \power (\genT \fun \genT)
\where
	bij = \genT \bij \genT
\end{gendef}

\begin{remark}
The composition of bijections is a bijection.

\begin{zed}
	\forall f, g: bij[\setT] @ \\
	\t1	f \circ g \in bij[\setT]
\end{zed}

\end{remark}

\begin{remark}
Composition is associative.

\begin{zed}
	\forall f, g, h: bij[\setT] @ \\
	\t1	f \circ (g \circ h) = (f \circ g) \circ h
\end{zed}

\end{remark}

\begin{remark}
The identity function $\id \setT$ acts as a left and right identity element under composition.

\begin{zed}
	\forall f: bij[\setT] @ \\
	\t1	 \id \setT \circ f = f = f \circ \id \setT
\end{zed}

\end{remark}

\begin{remark}
The inverse $f \inv$ of a bijection $f$ is its left and right inverse under composition.

\begin{zed}
	\forall f: bij[\setT] @ \\
	\t1	f \circ f \inv = \id \setT = f \inv \circ f
\end{zed}

\end{remark}

\subsection{$Bij$}

The preceding remarks show that set $bij[\genT]$ under the operation of composition has the structure of a group.
Let $Bij[\genT]$ denote this group.

\begin{gendef}[\genT]
	Bij: bij[\genT] \cross bij[\genT] \fun bij[\genT]
\where
	Bij = (\lambda f, g: bij[\genT] @ f \circ g)
\end{gendef}

\begin{example}
Let $\setT$ be any non-empty set.
The composition operation $Bij[\setT]$ is a group over the set of bijections $bij[\setT]$ from $\setT$ to $\setT$.

\begin{zed}
\setT \neq \emptyset \implies \\
\t1	Bij[\setT] \in \group bij[\setT]
\end{zed}

\end{example}

\section{Abelian Groups}

\subsection{OperationIsCommutative}

Let $\genT$ be a set of elements.
A binary operation $A$ over $\genT$ is said to be {\em commutative} when the product of two elements doesn't depend on 
their order.

Let $OperationIsCommutative$ denote this situation.

\begin{schema}{OperationIsCommutative}[\genT]
A: \binop \genT
\where
\LET (\_ \mulG \_) == A @ \\
\t1	\forall x, y: \genT @ \\
\t2		x \mulG y = y \mulG x
\end{schema}

\subsection{\zcmd{abgroup}}

An {\em Abelian group} is a group in which the binary operation is commutative.
Let $\genT$ be a set of elements.

Let $\abgroup \genT$ denote the set of all Abelian groups over $\genT$.

\begin{zed}
\abgroup \genT == \{~ A: \group \genT | OperationIsCommutative[\genT] ~\}
\end{zed}

\subsection{\zcmd{addG}, \zcmd{zeroG}, and \zcmd{negG}}

Often in an Abelian group the binary operation is denoted as addition $x \addG y$,
the identity element as a zero $\zeroG$, and the inverse operation as negation $\negG x$.

\begin{example}
Addition over the integers is an Abelian group.

\begin{zed}
	(\_ + \_) \in \abgroup \num
\end{zed}

\end{example}

\printbibliography

\end{document}