\documentclass[11pt, oneside]{article}

\usepackage{../../shared/preamble}
\addbibresource{../../shared/references.bib}

\usepackage{../sets/sets}
\usepackage{groups}

\title{Groups}
\author{Arthur Ryman, {\tt arthur.ryman@gmail.com}}
\date{\today}

\begin{document}

\maketitle

\begin{abstract}
This article contains Z Notation type declarations for groups and some related objects.
It has been type checked by \fuzz.
\end{abstract}

\section{Introduction}

Groups are ubiquitous throughout mathematics and physics.

\section{Groups}

A {\it group} is a set $G$ of elements $x, y, \ldots$ endowed with an associative multiplication operation $x \gmul y$
that has an identity element $\gone$ such that all elements have inverses $x \ginv, y \ginv, \ldots$.

\subsection{$Group$}

Let $Group[G]$ denote the set of all group structures defined on some given set $G$ of elements.

\begin{schema}{Group}[G]
	\_ \gmul \_: G \cross G \fun G \\
	\gone: G \\
	\_ \ginv: G \fun G
\where
	\forall x, y, z: G @ \\
	\t1	(x \gmul y) \gmul z = x \gmul (y \gmul z)
\also
	\forall x: G @ \\
	\t1	x \gmul \gone = x = \gone \gmul x
\also
	\forall x: G @ \\
	\t1	x \gmul x \ginv = \gone = x \ginv \gmul x
\end{schema}
\begin{itemize}

\item The group multiplication operation is associative.

\item The element $\gone$ is both a left and right identity element under group multiplication.

\item Every group element has an inverse element that is both a left and right inverse under group multiplication.

\end{itemize}

\begin{remark}
The group multiplication operation uniquely determines the set of elements.

\begin{zed}
	\forall Group[\setX] @ \\
	\t1	\setX = \ran (\_ \gmul \_)
\end{zed}

\end{remark}

\begin{remark}
The group multiplication operation uniquely determines the identity operation.

\begin{zed}
	\forall Group[\setX] @ \\
	\t1	\forall x, y: \setX @ \\
	\t2		x \gmul y = x \implies y = \gone
\end{zed}

\end{remark}

\begin{remark}
The group multiplication uniquely determines the inverses.

\begin{zed}
	\forall Group[\setX] @ \\
	\t1	\forall x, y: \setX @ \\
	\t2		x \gmul y = \gone \implies x = y \ginv \land y = x \ginv
\end{zed}

\end{remark}

\subsection{$group$}

Let $group[G]$ denote the set of all group multiplication operations on $G$.

\begin{gendef}[G]
	group: \power(G \cross G \fun G)
\where
	group = \{~ Group[G] @ (\_ \gmul \_) ~\}
\end{gendef}

\subsection{$bij$}

Let $X$ be a set and let $bij[X]$ denote the set of a bijections $X \bij X$ from $X$ to itself.

\begin{gendef}[X]
	bij: \power(X \fun X)
\where
	bij = X \bij X
\end{gendef}

\begin{remark}
The composition of bijections is a bijection.

\begin{zed}
	\forall f, g: bij[\setX] @ \\
	\t1	f \circ g \in bij[\setX]
\end{zed}

\end{remark}

\begin{remark}
Composition is associative.

\begin{zed}
	\forall f, g, h: bij[\setX] @ \\
	\t1	f \circ (g \circ h) = (f \circ g) \circ h
\end{zed}

\end{remark}

\begin{remark}
The identity function $\id X$ acts as a left and right identity element under composition.

\begin{zed}
	\forall f: bij[\setX] @ \\
	\t1	 \id \setX \circ f = f = f \circ \id \setX
\end{zed}

\end{remark}

\begin{remark}
The inverse $f \inv$ of a bijection $f$ is its left and right inverse under composition.

\begin{zed}
	\forall f: bij[\setX] @ \\
	\t1	f \circ f \inv = \id \setX = f \inv \circ f
\end{zed}

\end{remark}

\subsection{$Bij$}

The preceding remarks show that set $bij[X]$ under the operation of composition has the structure of a group.
Let $Bij[X]$ denote this group.

\begin{gendef}[X]
	Bij: bij[X] \cross bij[X] \fun bij[X]
\where
	Bij = (\lambda f, g: bij[X] @ f \circ g)
\end{gendef}

\begin{theorem}
The multiplication operation $Bij[X]$ is a group.

\begin{zed}
	Bij[\setX] \in group[bij[\setX]]
\end{zed}

\end{theorem}

\section{Abelian Groups}

An {\it abelian group} is a group in which the multiplication is commutative, i.e. $x \gmul y = y \gmul x$.
In this case, the group multiplication is denoted as addition $x \gadd y$, the identity element is denoted as a zero $\gzero$,
and the inverse of an element is denoted as its negative $\gneg x$.

\subsection{$AbelianGroup$}

Let $AbelianGroup[G]$ denote the abelian group structures on $G$.

\begin{schema}{AbelianGroup}[G]
	Group[G] \\
	\_ \gadd \_: G \cross G \fun G \\
	\gzero: G \\
	\gneg: G \fun G
\where
	(\_ \gadd \_) = (\_ \gmul \_)
\also
	\gzero = \gone
\also
	\gneg = (\_ \ginv)
\also
	\forall x, y: G @ \\
	\t1	x \gadd y = y \gadd x
\end{schema}
\begin{itemize}

\item The group multiplication is denoted as addition.

\item The group identify element is denoted as the zero element.

\item The inverse of a group element is denoted as the negative of the element.

\item The group addition is commutative.

\end{itemize}

\subsection{$abelianGroup$}

Let $abelianGroup[G]$ denote the subset of abelian groups in $group[G]$.

\begin{gendef}[G]
	abelianGroup: \power(G \cross G \fun G)
\where
	abelianGroup = \{~ AbelianGroup[G] @ (\_ \gadd \_) ~\}
\end{gendef}

\begin{remark}
The set $abelianGroup[G]$ is a subset of $group[G]$.

\begin{zed}
	abelianGroup[\setX] \subseteq group[\setX]
\end{zed}

\end{remark}

\begin{example}
The integer addition is an abelian group.

\begin{zed}
	(\_ + \_) \in abelianGroup[\num]
\end{zed}

\end{example}

\printbibliography

\end{document}