\documentclass{amsart}

\usepackage{mathz-integers}

\usepackage{mathz-preamble}

\addbibresource{mathz-references.bib}

\begin{document}

\title{Integers}
\author{Arthur Ryman}
\email[Arthur Ryman]{arthur.ryman@gmail.com}
\date{\today}

\begin{abstract}
This article contains \ZN\ definitions for 
concepts related to the integers, $\num$.
It has been type checked by \fuzz.
\end{abstract}

\maketitle

\tableofcontents

\section{Introduction}

The integers $\num$ are built-in to \ZN.
This article provides definitions for some related objects so that they can be used and type checked in formal Z specifications.

\section{Exponentiation}

Let $x \in \num$ and $n \in \nat$. Then $x$ raised to the exponent $n$ is the product of $x$ multiplied by itself $n$ times,
with the convention that for $n = 0$ the result is $1$.
Let $exp(x, n)$ denote the result.

\begin{axdef}
	exp: \num \cross \nat \fun \num
\where
	\forall x: \num @ exp(x, 0) = 1
\also
	\forall x: \num; n: \nat_1 @ exp(x, n) = x * exp(x, n - 1)
\end{axdef}

\begin{example}
\begin{zed}
	exp(5,2) = 25
\end{zed}
\end{example}

\begin{remark}
\begin{zed}
	\forall x: \num @ \\
	\t1	exp(x, 0) = 1
\end{zed}
\end{remark}

\begin{remark}
\begin{zed}
	\forall x: \num @ \\
	\t1	exp(x, 1) = x
\end{zed}
\end{remark}

\begin{remark}
\begin{zed}
	\forall x: \num; n, m: \nat @ \\
	\t1	exp(x, n + m) = exp(x, n) * exp(x, m)
\end{zed}
\end{remark}

\begin{remark}
\begin{zed}
	\forall x, y: \num; n: \nat @ \\
	\t1	exp(x * y, n) = exp(x, n) * exp(y, n)
\end{zed}
\end{remark}

Exponentiation is normally denoted $x^n$ but \ZN\ cannot reproduce that.
Instead we use the infix operator notation that is common to programming languages such as FORTRAN and Python.
Define the notation $x \expN n = exp(x, n)$.

\begin{zed}
	(\_ \expN \_) == exp
\end{zed}

\begin{example}
\begin{zed}
	5 \expN 2 = 25
\end{zed}
\end{example}

\section{Divisibility}

This section defines \textit{divisibility} of integers.

Given integers $x$ and $y$ we say that $x$ \textit{divides} $y$ if there is some integer $q$ such 
that $q x = y$.

\begin{schema}{Divides}
	x, y, q : \num
\where
	q * x = y
\end{schema}

Let $divides$ denote the divisibility relation between integers
where $(x,y) \in divides$ means that $x$ divides $y$.

\begin{zed}
	divides == \{~ Divides @ x \mapsto y ~\}
\end{zed}

\begin{remark}
\begin{zed}
	divides \in \num \rel \num
\end{zed}
\end{remark}

We introduce the usual infix notation $x \divides y$ to denote that $(x, y) \in divides$.
\begin{zed}
(\_ \divides \_) == divides
\end{zed}

\begin{example}
The integer $7$ divides $42$ because $6 * 7 = 42$.

\begin{zed}
	7 \divides 42
\end{zed}
\end{example}

\begin{remark}
Every integer $x$ divides $0$ because $0 * x = 0$.

\begin{zed}
	\forall x : \num @ x \divides 0
\end{zed}
\end{remark}

\section{Divisors}

Let $x$ be a nonzero integer that divides the integer $y$.
We say that $x$ is a \textit{divisor} of $y$.

\begin{schema}{Divisor}
	Divides
\where
	x \neq 0
\end{schema}

Let the relation $(x, y) \in is\_divisor\_of$ denote that $x$ is a divisor of $y$.

\begin{zed}
	is\_divisor\_of == \{~ Divisor @ x \mapsto y ~\}
\end{zed}

Let the set $divisors(y)$ denote the set of all divisors of the integer $y$.

\begin{zed}
	divisors == (\lambda y: \num @ \{~ x : \num | (x, y) \in is\_divisor\_of  ~\})
\end{zed}

\begin{remark}
\begin{zed}
	divisors \in \num \fun \power \num
\end{zed}
\end{remark}

\begin{example}
The integer $6$ has the following divisors.

\begin{zed}
	divisors(6) = \{-6, -3, -2, -1, 1, 2, 3, 6 \}
\end{zed}
\end{example}

Let the set $positive\_divisors(y)$ denote the set of all positive divisors of the integer $y$.

\begin{zed}
	positive\_divisors == (\lambda y: \num @ divisors(y)  \cap \nat_1)
\end{zed}

\begin{remark}
\begin{zed}
	positive\_divisors \in \num \fun \power \nat_1
\end{zed}
\end{remark}

\begin{example}
The integer $6$ has the following positive divisors.
\begin{zed}
	positive\_divisors(6) = \{1, 2, 3, 6 \}
\end{zed}
\end{example}

\section{Prime Numbers}

An integer $p$ is \textit{prime} if it is greater than one 
and only has one and itself as positive divisors.

\begin{schema}{Prime}
	p: \num
\where
	p > 1
\also
	positive\_divisors(p) = \{ 1, p \}
\end{schema}
\begin{itemize}
	\item $p$ is greater than $1$.
	\item $1$ and $p$ are the only positive divisors of $p$.
\end{itemize}

\begin{example}
The integer $2$ is prime.

\begin{zed}
	\LET p == 2 @ \\
	\t1	Prime
\end{zed}
\end{example}

Let $primes$ denote the set of all primes.

\begin{zed}
	primes == \{~ Prime @ p ~\}
\end{zed}

\begin{remark}
\begin{zed}
	primes \subset \nat_1
\end{zed}
\end{remark}

\begin{example}
The natural numbers $2, 3, 5,$ and $7$ are primes.

\begin{zed}
	\{2, 3, 5, 7\} \subseteq primes
\end{zed}
\end{example}

\section{Addition of Integer Sequences}

Let $l$ be a natural number and
let $x$ and $y$ be two integer sequences of length $l$.
Their sum $z = x + y$ is the integer sequence of length $l$ defined by pointwise addition
of the terms in $x$ and $y$.

\begin{schema}{AddIntegerSequences}
	l : \nat \\
	x, y, z : \seq \num
\where
	l = \# x = \# y
\also
	z = (\lambda i : 1 \upto l @ x~i + y~i)
\end{schema}
\begin{itemize}
	\item The sequence $z$ is defined by pointwise addition of the sequences $x$ and $y$.
\end{itemize}

Let the function $add\_int\_seq(x, y) = z$ be the sum of two equal-length integer sequences.

\begin{zed}
	add\_int\_seq == \{~ AddIntegerSequences @ (x, y) \mapsto z ~\}
\end{zed}

\begin{remark}
Addition is a partial function on the set of all pairs of integer sequences.
\begin{zed}
	add\_int\_seq \in \seq \num \cross \seq \num \pfun \seq \num
\end{zed}
\end{remark}

We introduce the usual infix notation $x \addSeqZ y = add\_int\_seq(x, y)$.

\begin{zed}
	(\_ \addSeqZ \_) == add\_int\_seq
\end{zed}

\printbibliography

\end{document}