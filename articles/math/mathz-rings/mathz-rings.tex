\documentclass{amsart}

% include mathz package dependencies here
\usepackage{mathz-core}
\usepackage{mathz-real-numbers}
\usepackage{mathz-groups}
\usepackage{mathz-rings}

\usepackage{mathz-preamble}
\usepackage{babel}
\usepackage{tikz-cd}

\addbibresource{mathz-references.bib}

\begin{document}

\title{Notes on Rings}
\author{Arthur Ryman}
\email[Arthur Ryman]{arthur.ryman@gmail.com}
\date{\today}

\begin{abstract}
    This article contains formal definitions for mathematical concepts related to rings.
    It uses \ZN\ and has been type checked by \fuzz.
\end{abstract}

\maketitle

\tableofcontents

\section*{Introduction}

This article contains notes from the course \textit{Computational Commutative Algebra and Algebraic Geometry}
taught by Professor Michael Stillman in Winter 2025 as part of the Fields Academy Shared Graduate Courses
program.
It contains formal definitions for mathematical concepts related to rings.
It uses \ZN\cite{spivey-zrm} and has been type checked by \fuzz\cite{spivey-fm}.

\subsection{Source Material}

The course is concerned with Computational Commutative Algebra and Algebraic Geometry.
The course uses \mzMtwo\ for computation.
I'll use \cite{atiyah-itca} as the source for Commutative Algebra
and \cite{hartshorne-ag} as the source for Algebraic Geometry.

\subsection{Type Checking}

I'll start by pulling in the set of real numbers $\R$, and its zero element $\zeroR$.
So far, these are just \LaTeX\ commands.

Next, I'll say something formal about them.

\begin{remark}
Zero is a real number.
\begin{zed}
	\zeroR \in \R
\end{zed}
\end{remark}

\subsection{TODO List}

Define enough terms so that I can express the problem sets.
Also try to write formal specifications for the data types and functions in \mzMtwo.

Define the following terms:
\begin{itemize}
	\item ring
	\item homomorphism
	\item ideal
	\item field
	\item quotient of ring modulo an ideal
	\item ideal quotient, colon ideal
	\item Hilbert series, function
	\item monomial order
	\item Gröbner basis
	\item elimination as in \mzMtwo
\end{itemize}

\section{Rings and Ideals}

Refer to \cite[Chapter~1]{atiyah-itca} for definitions.

\subsection{Rings and Ring Homomorphisms}

A \textit{ring} $A$ is a set with addition and multiplication  operations such that:
\begin{enumerate}
\item The set $A$ is an abelian group with respect to addition. 
The zero element is denoted by $0$ and the additive inverse of $x \in A$ is denoted by $-x$.
\item Multiplication is associative ($(xy)z = x(yz)$) and distributive over addition 
($x(y+z) = x y + x z, (y+z)x=yx+zx$).
\item The ring is said to be \textit{commutative} if the multiplication is commutative.
\item The ring is said to have an \textit{identity element} if it has an element that is a left and right multiplicative
identity 
\end{enumerate}

\subsubsection{Rings}

The first two axioms define a general ring.
Regarded as a structure, define a ring $\strucA$ to be a triple $(A, (\_ \addG, \_), (\_ \mulG \_))$ 
consisting of a set, an addition operation, and a multiplication operation.

\begin{schema}{Rng\_Core}[\genT]
	A: \power \genT \\
	\_ + \_, \_ * \_: pbin\_op[\genT] \\
	\strucA: \power \genT \cross pbin\_op[\genT] \cross pbin\_op[\genT]
\where
	(A, (\_ + \_)) \in abgroup[A]
\also
	(A, (\_ * \_)) \in semigroup[A]
\also
	\forall x, y, z: A @ x * (y + z) = (x * y) + (x * z)
\also
	\forall x, y, z: A @ (y + z) * x  = (y * x) + (z * x)
\also
	\strucA = (A, (\_ + \_), (\_ * \_))
\end{schema}

\begin{itemize}
	\item addition is an abelian group
	\item multiplication is a semigroup
	\item left multiplication distributes over addition
	\item right multiplication distributes over addition
	\item the structure is a triple consisting of the carrier and two operations
\end{itemize}

Here I have omitted the letter \textit{i} in the name $Rng$ to remind us that
a general ring is not required to have a multiplicative identity element.

The additive identity element is denoted $\zeroRng$, 
the additive inverse of $x$ is denoted $\negG x$, and
the sum of $x$ and $\negG y$ is denoted $x \subG y$.

\begin{schema}{Rng}[\genT]
	Rng\_Core[\genT] \\
	\zeroRng: \genT \\
	\negG: \genT \pfun \genT \\
	\_ - \_: pbin\_op[\genT]
\where
	\zeroRng = identity\_element (A, (\_ + \_))
\also
	(\lambda x: A @ \negG x) = inverse\_operation (A, (\_ + \_))
\also
	(\_ - \_) = (\lambda x, y: A @ x + (\negG y))
\end{schema}

\begin{itemize}
	\item $\zeroRng$ is the additive identity element
	\item $\negG x$ is the additive inverse of $x$
	\item subtraction is defined in terms of addition and negation
\end{itemize}

Define $rng[\genT]$ to be the set of all rings in $\genT$.

\begin{zed}
	rng[\genT] == \{~ Rng[\genT] @ \strucA ~\}
\end{zed}

\begin{example}
The integers with addition and multiplication is a ring.
\begin{zed}
	(\num, (\_ + \_), (\_ * \_)) \in rng[\num]
\end{zed}
\end{example}

\subsubsection{Ring Homomorphisms}

Let $\strucA$ and $\strucA'$ be rings.
A \textit{ring homomorphism} from $\strucA$ to $\strucA'$ is a function $f$ from $A$ to $A'$ 
that preserves the addition and multiplication operations.
As a structure, we represent a ring homomorphism $F$ as the pair $(\strucA, \strucA') \mapsto f$.

\begin{schema}{Rng\_Hom}[\genT, \genU]
	Rng[\genT] \\
	Rng'[\genU] \\
	f: \genT \pfun \genU \\
	F: (rng[\genT] \cross rng[\genU]) \cross (\genT \pfun \genU)
\where
	f \in A \fun A'
\also
	\forall x, y: A @ f(x + y) = f(x) +' f(y)
\also
	\forall x, y:	 A @ f(x * y) = f(x) *' f(y)
\also
	F = (\strucA, \strucA') \mapsto f
\end{schema}

\begin{itemize}
	\item $f$ maps $A$ to $A'$
	\item $f$ preserves addition
	\item $f$ preserves multiplication
	\item the homomorphism as a structure consists of the pair of rings and the map between them
\end{itemize}

Define $rng\_Hom[\genT, \genU]$ to be the set of all ring homomorphisms from rings in $\genT$
to rings in $\genU$.

\begin{zed}
	rng\_Hom[\genT, \genU] == \{~ Rng\_Hom[\genT, \genU] @ F ~\}
\end{zed}

Define $rng\_hom(\strucA, \strucA')$ to be the set of all ring homomorphism from $\strucA$ to $\strucA'$.

\begin{zed}
	rng\_hom[\genT, \genU] == \\
	\t1	(\lambda \strucA: rng[\genT]; \strucA': rng[\genU] @ \\
	\t2		\{~ (\strucA, \strucA') ~\} \dres rng\_Hom[\genT, \genU])
\end{zed}

\subsubsection{Identity Maps}

Define $rng\_id[\genT]$ to be the function that maps rings in $\genT$ to their identity maps.

\begin{zed}
	rng\_id[\genT] == \{~ Rng[\genT] @ \strucA \mapsto ((\strucA, \strucA) \mapsto \id A) ~\}
\end{zed}

\begin{remark}
	The identity map on any ring is a homomorphism.
	
\begin{zed}
	rng\_id[\setT] \in rng[\setT] \fun  rng\_Hom[\setT, \setT]
\end{zed}

\end{remark}

\subsubsection{Composition}

Given ring homomorphisms $f$ from $A$ to $A'$ and $f'$ from $A'$ to $A''$,
we can define their \textit{composition} $g = f' \circ f$ from $A$ to $A''$.

\begin{schema}{Rng\_Composition}[\genT, \genU, \genV]
	Rng\_Hom[\genT, \genU] \\
	Rng\_Hom'[\genU, \genV] \\
	g: \genT \pfun \genV \\
	G: (rng[\genT] \cross rng[\genV]) \cross (\genT \pfun \genV)
\where
	g = f' \circ f
\also
	G = (\strucA, \strucA'') \mapsto g
\end{schema}

\begin{remark}
The composition of ring homomorphisms is a ring homomorphism.

\begin{zed}
	\forall Rng\_Composition[\setT, \setU, \setV] @ G \in rng\_hom(\strucA, \strucA'')
\end{zed}

\end{remark}

Let $G = F' \circRng F$ denote the composition of ring homomorphisms.

\begin{zed}
	(\_ \circRng \_)[\genT, \genU, \genV] == \{~ Rng\_Composition[\genT, \genU, \genV] @ (F', F) \mapsto G ~\}
\end{zed}

\begin{remark}
The identity map is a left and right identity element under composition of ring homomorphisms.

\begin{zed}
	\forall Rng\_Hom[\setT, \setU] @ \\
	\t1	F \circRng rng\_id(\strucA) = F = rng\_id(\strucA') \circRng F
\end{zed}

\end{remark}

The preceding remark states that the diagram in Figure~\ref{fig:comp-id} commutes.

\begin{figure}[h]
\centering
\begin{tikzcd}
\strucA \arrow[d, "\id"] \arrow[rd, "F"] \arrow[r, "F"]	& \strucA' \arrow[d, "\id"] \\
\strucA \arrow[r, "F"]	\arrow[r, "F"]				& \strucA'
\end{tikzcd}
\caption{Composition with the identity homomorphism}
\label{fig:comp-id}
\end{figure}

\subsubsection{Commutative Rings}

A ring is said to be \textit{commutative} if its multiplication is commutative.

\begin{schema}{CommRng}[\genT]
	Rng[\genT]
\where
	\forall x, y: A @ x * y = y * x
\end{schema}

\begin{itemize}
	\item multiplication is commutative
\end{itemize}

Define $commrng[\genT]$ to be the set of all commutative rings in $\genT$.

\begin{zed}
	commrng[\genT] == \{~ CommRng[\genT] @ \strucA ~\}
\end{zed}

\begin{remark}
A commutative ring in $\genT$ is a ring in $\genT$.

\begin{zed}
	commrng[\setT] \subseteq rng[\setT]
\end{zed}

\end{remark}

A homomorphism of commutative rings is simply a homomorphism of the underlying rings.

\begin{schema}{CommRng\_Hom}[\genT, \genU]
	CommRng[\genT] \\
	CommRng'[\genU] \\
	Rng\_Hom[\genT, \genU]
\end{schema}

Define $commrng\_Hom[\genT, \genU]$ to be the set all homomorphisms of commutative rings
in $\genT$ to commutative rings in $\genU$.

\begin{zed}
	commrng\_Hom[\genT, \genU] == \{~ CommRng\_Hom[\genT, \genU] @ F ~\}
\end{zed}

\begin{remark}
A homomorphism of commutative rings is a homomorphism of rings.

\begin{zed}
	commrng\_Hom[\setT, \setU] \subseteq rng\_Hom[\setT, \setU]
\end{zed}

\end{remark}

Define $commrng\_hom(\strucA, \strucA')$ to be the set all homomorphisms from the commutative ring
$\strucA$ to the commutative ring $\strucA'$.

\begin{zed}
	commrng\_hom[\genT, \genU] == \\
	\t1	(\lambda \strucA: commrng[\genT]; \strucA': commrng[\genU] @ \\
	\t2		\{~ (\strucA, \strucA') ~\} \dres commrng\_Hom[\genT, \genU])
\end{zed}

Define $commrng\_id[\genT]$ to be the function that maps commutative rings in $\genT$ to their identity maps.

\begin{zed}
	commrng\_id[\genT] == \{~ CommRng[\genT] @ \strucA \mapsto ((\strucA, \strucA) \mapsto \id A) ~\}
\end{zed}

\begin{remark}
The identity map of a commutative ring is a commutative ring homomorphism from the ring to itself.

\begin{zed}
	\forall \strucA: commrng[\setT] @ commrng\_id(\strucA) \in commrng\_hom(\strucA, \strucA)
\end{zed}

\end{remark}

Given commutative ring homomorphisms $f$ from $A$ to $A'$ and $f'$ from $A'$ to $A''$,
we can define their \textit{composition} $g = f' \circ f$ from $A$ to $A''$.

\begin{schema}{CommRng\_Composition}[\genT, \genU, \genV]
	CommRng\_Hom[\genT, \genU] \\
	CommRng\_Hom'[\genU, \genV] \\
	g: \genT \pfun \genV \\
	G: (commrng[\genT] \cross commrng[\genV]) \cross (\genT \pfun \genV)
\where
	g = f' \circ f
\also
	G = (\strucA, \strucA'') \mapsto g
\end{schema}

\begin{remark}
The composition of commutative ring homomorphisms is a commutative ring homomorphism.

\begin{zed}
	\forall CommRng\_Composition[\setT, \setU, \setV] @ G \in commrng\_hom(\strucA, \strucA'')
\end{zed}

\end{remark}

Let $G = F' \circCommrng F$ denote the composition of commutative ring homomorphisms.

\begin{zed}
	(\_ \circCommrng \_)[\genT, \genU, \genV] == 
	\{~ CommRng\_Composition[\genT, \genU, \genV] @ (F', F) \mapsto G ~\}
\end{zed}


TODO: define homomorphisms, identity maps, composition, and composition with identity maps

\subsubsection{Unital Rings}

A ring is said to have an \textit{identity element} if it has a left and right multiplicative identity element.
In other words, the multiplication operation is a monoid.
A ring with an identity element is also said to be a \textit{unital} ring.
The multiplicative identity element of a unital ring is denoted $\oneRing$.

\begin{schema}{Ring}[\genT]
	Rng[\genT] \\
	\oneRing: \genT
\where
	(A, (\_ * \_)) \in monoid[A]
\also
	\oneRing = identity\_element(A, (\_ * \_))
\end{schema}

\begin{itemize}
	\item the multiplication operation is a monoid
	\item the multiplicative identity element is denoted $\oneRing$
\end{itemize}

Define $ring[\genT]$ to be the set of all unital rings in $\genT$.

\begin{zed}
	ring[\genT] == \{~ Ring[\genT] @ \strucA ~\}
\end{zed}

TODO: define homomorphisms, identity maps, composition, and composition with identity maps

\subsubsection{Commutative Unital Rings}

Commutative algebra is primarily concerned with commutative, unital rings.

\begin{schema}{CommRing}[\genT]
	Ring[\genT] \\
	CommRng[\genT]
\end{schema}

Define $commring[\genT]$ to be the set of commutative unital rings in $\genT$.

\begin{zed}
	commring[\genT] == \{~ CommRing[\genT] @ \strucA ~\}
\end{zed}

TODO: define homomorphisms, identity maps, composition, and composition with identity maps

For the remainder of this article the term \textit{ring} will mean a commutative unital ring.
However, the formal notation will always be explicit.

\subsubsection{Zero Rings}

If the additive and multiplicative identity elements are the same then the ring is said to be a \textit{zero ring}.

\begin{schema}{ZeroRing}[\genT]
	Ring[\genT]
\where
	\oneRing = \zeroRng
\end{schema}

\begin{itemize}
	\item the additive and multiplicative identity elements are the same
\end{itemize}

\begin{remark}
A zero ring contains exactly one element, namely the zero element.
\begin{zed}
	\forall ZeroRing[\setT] @ A = \{ \zeroRng \}
\end{zed}

\begin{proof}
\begin{argue}
x: A 					& assumption-intro\\
x \\
\t1	= x * \oneRing		& $\oneRing$ is the identity element \\
\t1	= x * \zeroRng		& $\oneRing = \zeroRng$ by $ZeroRing$ \\
\t1	= \zeroRng		& $\zeroRng$ is the zero element \\
x: A \implies x = \zeroRng	& assumption-elim \\
A = \{ \zeroRng \}		& set extensionality
\end{argue}
\end{proof}

\end{remark}

TODO: remark on the universal properties of the zero ring in each of the four categories of rings

\subsubsection{Ring Homomorphisms}

A homomorphism of commutative unital rings is a mapping $f$ from ring $A$ into ring $A'$ that
preserves addition, multiplication, and identity elements.

\begin{schema}{CommRing\_Hom}[\genT, \genU]
	CommRing[\genT] \\
	CommRing'[\genU] \\
	Rng\_Hom[\genT, \genU] \\
	f: \genT \pfun \genU
\where
	f \in A \fun A'
\also
	\forall x, y: A @ f(x + y) = f(x) +' f(y)
\also
	\forall x, y: A @ f(x * y) = f(x) *' f(y)
\also
	f(\oneRing) = \oneRing'
\end{schema}

TODO: merge this in with the general discussion of homomorphisms

\subsubsection{Subrings}

A subring $A$ of $A'$ is a subset of elements that contains the identity element and is closed under
addition and multiplication.

TODO: use $S$ and $A$ to match textbook

\begin{schema}{CommRing\_Subring}[\genT]
	CommRing'[\genT] \\
	A: \power \genT
\where
	A \subseteq A'
\also
	\oneRing' \in A
\also
	\forall x, y: A @ x +' y \in A
\also
	\forall x, y: A @ x *' y \in A	
\end{schema}

A subring itself becomes a ring by restriction of the enclosing ring operations.

\begin{schema}{CommRing\_Restriction}[\genT]
	CommRing\_Subring[\genT] \\
	CommRing[\genT]
\where
	(\_ + \_) = (\lambda x, y: A @ x +' y)
\also
	(\_ * \_) = (\lambda x, y: A @ x *' y)
\end{schema}

Set inclusion defines a map $f$ from the subring to the ring.

\begin{schema}{CommRing\_Inclusion}[\genT]
	CommRing\_Restriction[\genT] \\
	f: \genT \pfun \genT \\
	F: (commring[\genT] \cross commring[\genT]) \cross (\genT \pfun \genT)
\where
	f = \id A
\also
	F = (\strucA, \strucA') \mapsto f
\end{schema}

\begin{remark}
Subring inclusion is a ring homomorphism.

\begin{zed}
\forall CommRing\_Inclusion[\setT] @ CommRing\_Hom[\setT, \setT]
\end{zed}

\end{remark}

\subsubsection{Composition}

Given homomorphisms $f : A \fun A'$ and $f': A' \fun A''$ their composition 
$f' \circ f$ is a mapping $g: A \fun A''$.

\begin{schema}{CommRing\_Composition}[\genT, \genU, \genV]
	CommRing\_Hom[\genT, \genU] \\
	CommRing\_Hom'[\genU, \genV] \\
	g: \genT \pfun \genV
\where
	g = f' \circ f
\end{schema}

\begin{remark}
The composition of homomorphisms is a homomorphism.
\end{remark}

TODO: merge with general discussion

NOTE: the preceding sections should be completed and made consistent with each other,
however, I will continue on with formalizing the content of Atiyah-MacDonald so I can determine
if anything is actually hard to formalize, and also so that I can be more effective with Macaulay 2.

\subsection{Ideals. Quotient rings}

An \textit{ideal} $\fraka$ of a ring $A$ is a subset of $A$ that is an additive subgroup
and is such that $A\fraka \subseteq \fraka$.

\begin{schema}{Ideal}[\genT]
	CommRing[\genT] \\
	\fraka: \power \genT
\where
	\fraka \subseteq A
\also
	\forall x, y: \fraka @ x + y \in \fraka \land x - y \in \fraka
\also
	\forall x: A; y: \fraka @ x * y \in \fraka
\end{schema}

\begin{itemize}
	\item the ideal is a subset of the ring
	\item the ideal is closed under addition and subtraction, making it a subgroup
	\item the ideal is closed under multiplication by elements of the ring
\end{itemize}

The quotient group $A/\fraka$ inherits a well-defined multiplication from $A$
making it a ring called the \textit{quotient ring} (or \textit{residue class ring} $A/\fraka$.

\begin{schema}{QuotientRing}[\genT]
	CommRing\_Hom[\genT, \power \genT] \\
	Ideal[\genT]
\where
	f = (\lambda x: A @ \{~ y: \fraka @ x + y ~\})
\also
	A' = \ran f
\end{schema}

TODO: first define the quotient group and the projection and cosets, showing that the projection
is a homomorphism. we need to show that the cosets from an additive group.
Then show that the cosets form a monoid.
The moral of the story is that I can't skip any steps. Otherwise the definitions get big and repetitive.

\subsection{Zero-divisors. Nilpotent elements. Units}

\subsection{Prime ideals and maximal ideals}

\subsection{Nilradical and Jacobson radical}

\subsection{Operations on ideals}

\subsection{Extension and contraction}

\subsection{Exercises}

\printbibliography

\end{document}
