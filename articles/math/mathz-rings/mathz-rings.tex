\documentclass{amsart}

% include mathz package dependencies here
\usepackage{mathz-core}
\usepackage{mathz-real-numbers}
\usepackage{mathz-groups}
\usepackage{mathz-rings}

\usepackage{mathz-preamble}
\usepackage{babel}

\addbibresource{mathz-references.bib}

\begin{document}

\title{Notes on Rings}
\author{Arthur Ryman}
\email[Arthur Ryman]{arthur.ryman@gmail.com}
\date{\today}

\begin{abstract}
    This article contains formal definitions for mathematical concepts related to rings.
    It uses \ZN\ and has been type checked by \fuzz.
\end{abstract}

\maketitle

\tableofcontents

\section{Introduction}

This article contains notes from the course \textit{Computational Commutative Algebra and Algebraic Geometry}
taught by Professor Michael Stillman in Winter 2025 as part of the Fields Academy Shared Graduate Courses
program.
It contains formal definitions for mathematical concepts related to rings.
It uses \ZN\cite{spivey-zrm} and has been type checked by \fuzz\cite{spivey-fm}.

\subsection{Source Material}

The course is concerned with Computational Commutative Algebra and Algebraic Geometry.
The course uses \mzMtwo\ for computation.
I'll use \cite{atiyah-itca} as the source for Commutative Algebra
and \cite{hartshorne-ag} as the source for Algebraic Geometry.

\subsection{Type Checking}

I'll start by pulling in the set of real numbers $\R$, and its zero element $\zeroR$.
So far, these are just \LaTeX\ commands.

Next, I'll say something formal about them.

\begin{remark}
Zero is a real number.
\begin{zed}
	\zeroR \in \R
\end{zed}
\end{remark}

\subsection{TODO List}

Define enough terms so that I can express the problem sets.
Also try to write formal specifications for the data types and functions in \mzMtwo.

Define the following terms:
\begin{itemize}
	\item ring
	\item homomorphism
	\item ideal
	\item field
	\item quotient of ring modulo an ideal
	\item ideal quotient, colon ideal
	\item Hilbert series, function
	\item monomial order
	\item Gröbner basis
	\item elimination as in \mzMtwo
\end{itemize}

\section{Rings and Ideals}

Refer to \cite[Chapter~1]{atiyah-itca} for definitions.

\subsection{Rings and Ring Homomorphisms}

A \textit{ring} $A$ is a set with addition and multiplication  operations such that:
\begin{enumerate}
\item The set $A$ is an abelian group with respect to addition. 
The zero element is denoted by $0$ and the additive inverse of $x \in A$ is denoted by $-x$.
\item Multiplication is associative ($(xy)z = x(yz)$) and distributive over addition 
($x(y+z) = x y + x z, (y+z)x=yx+zx$).
\item The ring is said to be \textit{commutative} if the multiplication is commutative.
\item The ring is said to have an \textit{identity element} if it has an element that is a left and right multiplicative
identity 
\end{enumerate}

\subsubsection{Rings}

The first two axioms define a general ring.
As a structure, we define a ring $\strucA$ to be a triple $(A, (\_ \addG, \_), (\_ \mulG \_))$ 
consisting of a set, an addition operation, and a multiplication operation.

\begin{schema}{Ring\_Core}[\genT]
	A: \power \genT \\
	\_ + \_, \_ * \_: PBinOp[\genT] \\
	\strucA: \power \genT \cross PBinOp[\genT] \cross PBinOp[\genT]
\where
	(A, (\_ + \_)) \in abgroup[A]
\also
	(A, (\_ * \_)) \in semigroup[A]
\also
	\forall x, y, z: A @ x * (y + z) = (x * y) + (x * z)
\also
	\forall x, y, z: A @ (y + z) * x  = (y * x) + (z * x)
\also
	\strucA = (A, (\_ + \_), (\_ * \_))
\end{schema}

\begin{itemize}
	\item addition is an abelian group
	\item multiplication is a semigroup
	\item left multiplication distributes over addition
	\item right multiplication distributes over addition
	\item the structure is a triple consisting of the carrier and two operations
\end{itemize}

The additive identity element is denoted $\zeroG$, 
the additive inverse of $x$ is denoted $\negG x$, and
the sum of $x$ and $\negG y$ is denoted $x \subG y$.

\begin{schema}{Ring}[\genT]
	Ring\_Core[\genT] \\
	\zeroG: \genT \\
	\negG: \genT \pfun \genT \\
	\_ - \_: PBinOp[\genT]
\where
	\zeroG = identity\_element (A, (\_ + \_))
\also
	(\lambda x: A @ \negG x) = inverse\_operation (A, (\_ + \_))
\also
	(\_ - \_) = (\lambda x, y: A @ x + (\negG y))
\end{schema}

\begin{itemize}
	\item $\zeroG$ is the additive identity element
	\item $\negG x$ is the additive inverse of $x$
	\item subtraction is defined in terms of addition and negation
\end{itemize}

\subsubsection{Commutative Rings}

A ring is said to be \textit{commutative} if its multiplication is commutative.

\begin{schema}{CommutativeRing}[\genT]
	Ring[\genT]
\where
	\forall x, y: A @ x * y = y * x
\end{schema}

\begin{itemize}
	\item multiplication is commutative
\end{itemize}

\subsubsection{Unital Rings}

A ring is said to have an \textit{identity element} if it has a left and right multiplicative identity element.
In other words, the multiplication operation is a monoid.
A ring with an identity element is also said to be a \textit{unital} ring.
The multiplicative identity element of a unital ring is denoted $\oneG$.

\begin{schema}{UnitalRing}[\genT]
	Ring[\genT]\\
	\oneG: \genT
\where
	(A, (\_ * \_)) \in monoid[A]
\also
	\oneG = identity\_element(A, (\_ * \_))
\end{schema}

\begin{itemize}
	\item the multiplication operation is a monoid
	\item the multiplicative identity element is denoted $\oneG$
\end{itemize}

\subsubsection{Commutative Unital Rings}

Commutative algebra is primarily concerned with commutative, unital rings.

\begin{schema}{CURing}[\genT]
	CommutativeRing[\genT] \\
	UnitalRing[\genT]
\end{schema}

For the remainder of this article the term \textit{ring} will denote a commutative ring with an identity element.
However, the formal notation will always be explicit.

\subsubsection{Zero Rings}

If the additive and multiplicative identity elements are the same then the ring is said to be a \textit{zero ring}.

\begin{schema}{ZeroRing}[\genT]
	CURing[\genT]
\where
	\oneG = \zeroG
\end{schema}

\begin{itemize}
	\item the additive and multiplicative identity elements are the same
\end{itemize}

\begin{remark}
A zero ring contains exactly one element, namely the zero element.
\begin{zed}
	\forall ZeroRing[\setT] @ A = \{ \zeroG \}
\end{zed}

\begin{proof}
\begin{argue}
x: A 					& assumption-intro\\
x \\
\t1	= x * \oneG		& $\oneG$ is the identity element \\
\t1	= x * \zeroG		& $\oneG = \zeroG$ by $ZeroRing$ \\
\t1	= \zeroG			& $\zeroG$ is the zero element \\
x: A \implies x = \zeroG	& assumption-elim \\
A = \{ \zeroG \}			& set extensionality
\end{argue}
\end{proof}

\end{remark}

\subsubsection{Ring Homomorphisms}

A ring homomorphism is a mapping $f$ from ring $A$ into ring $A'$ that
preserves addition, multiplication, and identity elements.

\begin{schema}{CURing\_Hom}[\genT, \genU]
	CURing[\genT] \\
	CURing'[\genU] \\
	f: \genT \pfun \genU
\where
	f \in A \fun A'
\also
	\forall x, y: A @ f(x + y) = f(x) +' f(y)
\also
	\forall x, y: A @ f(x * y) = f(x) *' f(y)
\also
	f(\oneG) = \oneG'
\end{schema}

\subsubsection{Subrings}

A subring $A$ of $A'$ is a subset of elements that contains the identity element and is closed under
addition and multiplication.

\begin{schema}{CURing\_Subring}[\genT]
	CURing'[\genT] \\
	A: \power \genT
\where
	A \subseteq A'
\also
	\oneG' \in A
\also
	\forall x, y: A @ x +' y \in A
\also
	\forall x, y: A @ x *' y \in A	
\end{schema}

A subring itself becomes a ring by restriction of the enclosing ring operations.

\begin{schema}{CURing\_Restriction}[\genT]
	CURing\_Subring[\genT] \\
	CURing[\genT]
\where
	(\_ + \_) = (\lambda x, y: A @ x +' y)
\also
	(\_ * \_) = (\lambda x, y: A @ x *' y)
\end{schema}

Set inclusion defines a map $f$ from the subring to the ring.

\begin{schema}{CURing\_Inclusion}[\genT]
	CURing\_Restriction[\genT] \\
	f: \genT \pfun \genT
\where
	f = (\lambda x: A @ x)
\end{schema}

\begin{remark}
Subring inclusion is a ring homomorphism.

\begin{zed}
\forall CURing\_Inclusion[\setT] @ CURing\_Hom[\setT, \setT]
\end{zed}

\end{remark}

\subsubsection{Composition}

Given homomorphisms $f : A \fun A'$ and $f': A' \fun A''$ their composition 
$f' \circ f$ is a mapping $g: A \fun A''$.

\begin{schema}{CURing\_Composition}[\genT, \genU, \genV]
	CURing\_Hom[\genT, \genU] \\
	CURing\_Hom'[\genU, \genV] \\
	g: \genT \pfun \genV
\where
	g = f' \circ f
\end{schema}

\begin{remark}
The composition of homomorphisms is a homomorphism.
\end{remark}

\printbibliography

\end{document}
