\documentclass{amsart}

\usepackage{mathz_core}
\usepackage{mathz_sets}
\usepackage{mathz_topological_spaces}
\usepackage{mathz_preamble}

\addbibresource{mathz-references.bib}

\begin{document}

\title{Topological Spaces}
\author{Arthur Ryman}
\email[Arthur Ryman]{arthur.ryman@gmail.com}
\date{\today}

\begin{abstract}
This article contains Z Notation definitions for topological spaces and related concepts.
It has been type checked by \fuzz.
\end{abstract}

\maketitle

\tableofcontents

\section{Topological Spaces}

\subsection{$Topology$}

A {\it topology} $\tau$ on $X$ is a family of subsets of $X$, referred to as the {\it open} subsets of $X$, that satisfy the following axioms.

\begin{schema}{Topology}[X]
	\tau: \Fam X
\where
	\emptyset \in \tau
\also
	X \in \tau
\also
	\forall F: \finset \tau @ \bigcap F \in \tau
\also
	\forall F: \power \tau @ \bigcup F \in \tau
\end{schema}

\begin{itemize}
\item The empty set is open.
\item The whole set is open.
\item The intersection of a finite family of open sets is open.
\item The union of any family of open sets is open. 
\end{itemize}

\subsection{$top$ and $tops$}

Let $top[X]$ denote the set of all topologies on $X$.

\begin{gendef}[X]
	top: \power(\Fam X)
\where
	top = \{~ Topology[X] @ \tau ~\}
\end{gendef}

Let $tops[X]$ denote the set of all topologies on subsets $U \subseteq X$.

\begin{gendef}[X]
	tops: \power(\Fam X)
\where
	tops = \bigcup \{~ U: \power X @ top[U] ~\}
\end{gendef}

\subsection{$discrete$ and $indiscrete$}

The {\it discrete} topology on $X$ consists of all subsets of $X$.
The {\it indiscrete} topology on $X$ consists of just $X$ and $\emptyset$.
Let $discrete[X]$ and $indiscrete[X]$ denote the discrete and indiscrete topologies on $X$.

\begin{gendef}[X]
	discrete, indiscrete: \Fam X
\where
	discrete = \power X
\also
	indiscrete =  \{ \emptyset, X \}
\end{gendef}

\begin{example}
Let $\setX$ be an arbitrary set.
Then $discrete[\setX]$ and $indiscrete[\setX]$ are topologies on $\setX$.

\begin{zed}
	discrete[\setX] \in top[\setX] 
\also
	indiscrete[\setX] \in top[\setX]
\end{zed}

\end{example}

\subsection{$topGen$}

\begin{remark}

The intersection of a set of topologies on $X$ is also a topology on $X$.

\end{remark}

Given a family $B$ of subsets of $X$, the topology {\it generated by} $B$ is the intersection of all
topologies that contain $B$.
The set $B$ is referred to as a {\it basis} for the topology it generates.
Let $topGen[X]~B$ denote the topology on $X$ generated by the basis $B$.

\begin{gendef}[X]
	topGen: \Fam X \fun top[X]
\where
	\forall B: \Fam X @ \\
	\t1	topGen~B = \bigcap \{~ \tau: top[X] | B \subseteq \tau ~\}
\end{gendef}

\begin{example}
Let $\setX$ be an arbitrary set.

\begin{zed}
	topGen[\setX] \emptyset = indiscrete[\setX]
\also
	topGen[\setX] \{ \emptyset \} = indiscrete[\setX]
\also
	topGen[\setX] \{ \setX \} = indiscrete[\setX]
\end{zed}

\end{example}

\subsection{$topSpace$}

Let $X$ be a set.
A {\it topological space} is a pair $(X, \tau)$ where $\tau$ is a topology on $X$.
Let $topSpace[X]$ denote the set of all topological spaces $(X,\tau)$.

\begin{zed}
	topSpace[X] == \{~ \tau: top[X] @ (X, \tau) ~\}
\end{zed}

\begin{example}
Let $\setX$ be an arbitrary set.

\begin{zed}
	(\setX, indiscrete[\setX]) \in topSpace[\setX]
\also
	(\setX, discrete[\setX]) \in topSpace[\setX]
\end{zed}


\end{example}

\subsection{$topSpaces$}

Let $topSpaces[t]$ dentote the set of all topological spaces $(X,\tau)$ where $X$ is a subset of $t$.

\begin{gendef}[t]
	topSpaces: \power t \rel \Fam t
\where
	topSpaces = \{~ X: \power t; \tau: \Fam t | \tau \in top[X] ~\}
\end{gendef}

\begin{remark}

\begin{zed}
	topSpace[\setX] \subseteq topSpaces[\setX]
\end{zed}

\end{remark}

\section{Continuous Mappings}

Let $(X,\tau)$ and $(Y,\sigma)$ be topological spaces.

\subsection{$Continuous$}

A mapping $f \in X \fun Y$ is said to be {\it continuous} if the inverse image of every open set is open.

\begin{schema}{Continuous}[X,Y]
	f: X \fun Y \\
	\tau: top[X] \\
	\sigma: top[Y]
\where
	\forall U: \sigma @ \\
	\t1	f\inv\limg U \rimg \in \tau
\end{schema}

\subsection{\zcmd{CzeroTT}}

Let $A$ and $B$ be topological spaces, and
let $\CzeroTT(A,B)$ denote the set of continuous mappings from $A$ to $B$.

\begin{gendef}[X,Y]
	\CzeroTT: topSpace[X] \cross topSpace[Y] \fun \power (X \fun Y)
\where
	\forall \tau: top[X]; \sigma: top[Y] @ \\
	\t1	\LET A == (X, \tau); B == (Y, \sigma) @ \\
	\t2		\CzeroTT(A,B) = \{~ f: X \fun Y | Continuous[X,Y] ~\}
\end{gendef}

\subsection{The Identity Mapping}

\begin{remark}
The identity mapping is continuous.

\begin{zed}
	\forall \tau: top[\setX] @ \\
	\t1	\LET A == (\setX, \tau) @ \\
	\t2		\id \setX \in \CzeroTT(A, A)
\end{zed}

\end{remark}

\begin{remark}
The constant mapping is continuous.

\begin{zed}
	\forall \tau: top[\setX]; \sigma: top[\setY]; c: \setY @ \\
	\t1	\LET A == (\setX, \tau); B == (\setY, \sigma) @ \\
	\t2		\const[\setX,\setY] c \in \CzeroTT(A,B)
\end{zed}

\end{remark}

\subsection{Composition of Continuous Mapping}

\begin{remark}
Let $\setX$, $\setY$, and $\setZ$ be arbitrary sets.
The composition of continuous mappings is a continuous mapping.

\begin{zed}
	\forall A: topSpace[\setX]; B: topSpace[\setY]; C: topSpace[\setZ] @ \\
	\t1	\forall f: \CzeroTT(A, B); g: \CzeroTT(B, C) @ \\
	\t2		g \circ f \in \CzeroTT(A, C)
\end{zed}

\end{remark}

\section{Induced Topology}

Let $A = (X, \tau)$ be a topological space and let $U \subseteq X$ be a subset.
The topology on $X$ {\it induces} a topology on $U$.
This topology is variously referred to as the {\it induced}, {\it relative}, or {\it subspace} topology on $U$.

\subsection{\zcmd{inducedFam}}

Let $\phi$ be a family of subsets of $X$ and let $U$ be a subset of $X$.
The family of subsets of $U$ {\it induced} by $\phi$ is the set of intersections of the members of $\phi$ with $U$.
Let $\phi \inducedFam U$ denote the family on $U$ induced by $\phi$.

\begin{gendef}[X]
	\_ \inducedFam \_:  \Fam X \cross \power X \fun \Fam X
\where
	\forall \phi: \Fam X; U: \power X @ \\
	\t1	\phi \inducedFam U = \{~ Y: \phi @ Y \cap U ~\}
\end{gendef}

\begin{remark}
If $\tau$ is a topology on $X$ then $\tau \inducedFam U$ is a topology on $U$.

\begin{zed}
	\forall \tau: top[\setX]; U: \power \setX @ \\
	\t1	\tau \inducedFam U \in top[U]
\end{zed}

\end{remark}

\subsection{\zcmd{inducedTopSp}}

Let $(X, \tau) \inducedTopSp U$ denote the corresponding induced topological space.

\begin{gendef}[X]
	\_ \inducedTopSp \_: topSpace[X] \cross \power X \fun topSpaces[X]
\where
	\forall \tau: top[X]; U: \power X @ \\
	\t1	(X, \tau) \inducedTopSp U = (U, \tau \inducedFam U)
\end{gendef}

\section{Product Topology}

Let $(X, \tau)$ and $(Y, \sigma) $ be topological spaces.
There is a natural topology on $X \cross Y$ generated by the products of the sets in $\tau$ and $\sigma$.

\subsection{\zcmd{prodFam}}

Let $X$ and $Y$ be sets and let $\phi$ and $\psi$ be families on them.
The product of these families is the family that consists of the products of the sets in them and is a family on $X \cross Y$.
Let $\phi \prodFam \psi$ denote the product of the families.

\begin{gendef}[X,Y]
	\_ \prodFam \_ : \Fam X \cross \Fam Y \fun \Fam(X \cross Y)
\where
	\forall \phi: \Fam X; \psi: \Fam Y @ \\
	\t1	\phi \prodFam \psi = \{~ U: \phi; V: \psi @ U \cross V ~\}
\end{gendef}

\begin{remark}

If $\tau$ and $sigma$ are topologies then $\tau \prodFam \sigma$ is not, in general, a topology.
However, we can use it to generate a topology.

\end{remark}

\subsection{\zcmd{prodTop}}

Let $\tau \prodTop \sigma$ denote the topology generated by $\tau \prodFam \sigma$.

\begin{gendef}[X,Y]
	\_ \prodTop \_: top[X] \cross top[Y] \fun top[X \cross Y]
\where
	\forall \tau: top[X]; \sigma: top[Y] @ \\
	\t1	\tau \prodTop \sigma = topGen(\tau \prodFam \sigma)
\end{gendef}

\subsection{\zcmd{prodTopSp}}

Let $(X, \tau) \prodTop (Y, \sigma)$ denote the product topological space.

\begin{gendef}[X,Y]
	\_ \prodTopSp \_: topSpace[X] \cross topSpace[Y] \fun topSpace[X \cross Y]
\where
	\forall \tau: top[X]; \sigma: top[Y] @ \\
	\t1	(X, \tau) \prodTopSp (Y, \sigma) = (X \cross Y, \tau \prodTop \sigma)
\end{gendef}

\printbibliography

\end{document}