\documentclass[11pt, oneside]{article}

\usepackage{../../shared/preamble}
\addbibresource{../../shared/references.bib}

\usepackage{../real-numbers/real-numbers}
%\usepackage{manifolds}

\title{Manifolds}
\author{Arthur Ryman, {\tt arthur.ryman@gmail.com}}
\date{\today}

% Document
\begin{document}

\maketitle

\begin{abstract}
This article contains Z Notation type declarations for manifolds and some related objects.
It has been type checked by \fuzz.
\end{abstract}


    \section{Introduction}

    Manifolds can be defined in several ways.
    The way I prefer to think about them is that, first of all, they are  based on topological spaces.
    A manifold is therefore a topological space with some additional structure.
    This additional structure allows one to regard a manifold as, locally, being like an open subset of $\R^n$
    for some natural number $n$ referred to as the dimension of the manifold.
    In the following, let $M$ be a topological space of dimension $n$.

    \section{Charts}
    A chart $\phi$ on $M$ is a continuous injection of some open subset $U \subseteq M$ into $\R^n$.
    A chart gives every point $p \in U$ in its domain of definition a tuple of $n$ real number coordinates.
    \begin{equation}
        \phi: U \inj \R^n
    \end{equation}

    \subsection{Transition Functions}
    Let $U, V, W$ be open subsets of $M$ with $W = U \cap V$.
    Let $\phi: U \inj \R^n$ and $\psi: V \inj \R^n$ be charts.
    Every point $p \in W$ is therefore given two, typically distinct, tuples of coordinates.
    The mapping from one coordinate tuple to the other is called the transition function defined by the pair of charts.
    Let $t_{\phi,\psi}$ denote that transition function that maps the $\phi$ coordinates to the $\psi$ coordinates.
    \begin{equation}
        \forall x \in \phi(W) @ t_{\phi,\psi}(x) = \psi(\phi^{-1}(x))
    \end{equation}

    \subsection{Compatible Charts}
    Let $\mathcal{F}$ be some family of partial injections from $\R^n$ to $\R^n$, e.g. continuous, differentiable, smooth, defined on the open subsets.
    \begin{equation}
        \mathcal{F} \subseteq \R^n \pinj \R^n
    \end{equation}

    A pair of charts are said to be compatible with respect to $\mathcal{F}$ when their transition functions belong to $\mathcal{F}$.

    \section{Atlases}
    A set of pairwise compatible charts that cover $M$ is called an atlas for $M$.
    An atlas gives $M$ a manifold structure.
    If the charts are only required to be continuous then $M$ is called a topological manifold.
    If the charts are required to be differentiable then the atlas is called a differential or differentiable structure and $M$ is called
    a differentiable manifold.
    Infinitely differentiable charts are called smooth charts.
    We are only concerned with smooth charts and manifolds.

    In general, we normally consider an atlas to be a maximal set of charts.
    A given set of mutually compatible charts belongs to a unique maximal atlas.
    The given set is said to generate the maximal atlas.

    \section{Smooth Mappings}
    Mappings from one smooth manifold to another are called smooth when they are smooth when expressed in their coordinate charts.
    A smooth mapping that has a smooth inverse is called a diffeomorphism.

    \section{Tangent Vectors}
    A tangent vector $X$ at the point $p \in M$ is a mapping from the set of smooth functions at $p$ to $\R$ that satisfies the following
    for all $c \in \R$ and $f,g \in C^{\infty}(M,p)$
    \begin{align}
        X(cf) &= cX(f) \\
        X(f + g) &= X(f) + X(g) \\
        X(fg) &= g(p)X(f) + f(p)X(g)
    \end{align}

    A smooth curve $\gamma: \R \fun M$ defines a tangent vector $X$ at $p=\gamma(0)$ by
    \begin{equation}
        X(f) = \left.\frac{df(\gamma(t))}{dt}\right|_{t=0}
    \end{equation}

    \section{Tangent Bundles}

    The set of all tangent vectors at $p$ is denoted $M_p$ or $T_p(M)$.
    It is an $n$-dimensional vector space and is called the tangent space at $p$.
    The set of all tangent spaces is called the tangent bundle and is denoted $T(M)$
    \begin{equation}
        T(M) = \{~ (p,X) | p \in M, X \in M_p ~\}
    \end{equation}

    The tangent bundle $T(M)$ is a smooth vector bundle over $M$ under the natural projection
    $\pi: T(M) \fun M, \pi(p,X) = p$.

    \printbibliography

\end{document}