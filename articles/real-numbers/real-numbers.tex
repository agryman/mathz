\documentclass[11pt, oneside]{article}

\usepackage{../../shared/preamble}
\addbibresource{../../shared/references.bib}

\usepackage{../sets/sets}
\usepackage{../topological-spaces/topological-spaces}
\usepackage{real-numbers}

\title{Real Numbers}
\author{Arthur Ryman, {\tt arthur.ryman@gmail.com}}
\date{\today}

\begin{document}

\maketitle

\begin{abstract}
This article contains Z Notation type declarations for the real numbers, $\R$, and some related objects.
It has been type checked by \fuzz.
\end{abstract}

\section{Introduction}

The real numbers, $\R$, are foundational to many mathematical objects such as vector spaces and manifolds,
but are not built-in to Z Notation.
This article provides type declarations for $\R$ and related objects so that they can be used and type checked in formal Z specifications.

No attempt has been made to provide complete, axiomatic definitions of all these objects since that would only be of use for proof checking.
Although proof checking is highly desirable, it is beyond the scope of this article.
The type declarations given here are intended to provide a basis for future axiomatization.

\section{Real Numbers}

Z notation does not predefine the set of real numbers, so we define it here.

\subsection{\zcmd{R}}

Let $\R$ denote the set of real numbers.
We define it to be simply a given set.
We'll add further axioms as needed below.

\begin{zed}
[\R]
\end{zed}

\subsection{\zcmd{addR}, \zcmd{zeroR}, \zcmd{negR}, and \zcmd{subR}}

Let $x$ and $y$ be real numbers.
Let $x \addR y$ denote addition,
let $\zeroR$ denote zero,
let $\negR x$ denote negative,
and let $x \subR y$ denote subtraction.

Although these real number objects are displayed the same as the corresponding integer objects,
they represent distinct mathematical objects.
This distinction is apparent to the \fuzz\ type-checker and should not cause confusion to the human reader
because the underlying types of objects will, as a rule, be clear from the context.
Visually distinct symbols will be used in cases where confusion is possible.

The real numbers form an Abelian group under addition.

\begin{axdef}
	\_ \addR \_: \R \cross \R \fun \R \\
	\zeroR: \R \\
	\negR: \R \fun \R
	\where
	\exists_1 A: AbelianGroup[\R] @ \\
	\t1	A.elements = \R \land \\
	\t1	A.(\_ \addG \_) = (\_ \addR \_) \land \\
	\t1	A.\zeroG = \zeroR \land \\
	\t1	A.\negG = \negR
\end{axdef}

Subtraction is defined in terms of addition and negative.

\begin{axdef}
	\_ \subR \_: \R \cross \R \fun \R
	\where
	\forall x, y: \R @ x \subR y = x \addR (\negR y)
\end{axdef}

\subsection{\zcmd{Rnz}}

Let $\Rnz$ denote the set of non-zero real numbers.

\begin{zed}
	\Rnz == \R \setminus \{ \zeroR \}
\end{zed}

\subsection{\zcmd{mulR}}

Let $x$ and $y$ be real numbers.
Let $x \mulR y$ denote multiplication.

\begin{axdef}
	\_ \mulR \_: \R \cross \R \fun \R
\end{axdef}

\subsection{\zcmd{mulRnz}, \zcmd{oneR}, \zcmd{invRnz}, and \zcmd{divR}}

Let $(\_ \mulRnz \_)$ denote the restriction of $(\_ \mulR \_)$ to $\Rnz$.

\begin{axdef}
	\_ \mulRnz \_: \Rnz \cross \Rnz \fun \Rnz
	\where
	(\_ \mulRnz \_) = (\lambda x, y: \Rnz @ x \mulR y)
\end{axdef}

Let $x$ be real number and let $y$ be a non-zero real number.
let $\oneR$ denote one,
let $y \invRnz$ denote inverse,
and let $x \divR y$ denote division.

\begin{axdef}
	\oneR: \Rnz \\
	\_ \invRnz: \Rnz \fun \Rnz
\end{axdef}

The non-negative real numbers form an Abelian group under multiplication.

\begin{zed}
	\exists_1 A: AbelianGroup[\Rnz] @ \\
	\t1	A.elements = \Rnz \land \\
	\t1	A.(\_ \mulG \_) = (\_ \mulRnz \_) \land \\
	\t1	A.\oneG = \oneR \land \\
	\t1	A.(\_ \invG) = (\_ \invRnz)
\end{zed}

Division is defined in terms of multiplicative inverse.

\begin{axdef}
	\_ \divR \_: \R \cross \Rnz \fun \R
	\where
	\forall x: \R; y: \Rnz @ x \divR y = x \mulR (y \invRnz)
\end{axdef}

Addition is distributive over multiplication.

\begin{zed}
	\forall x, y, z: \R @ (x \addR y) \mulR z = x \mulR z \addR y \mulR z
\end{zed}

\section{The Real Numbers}

\subsection{\zcmd{R}}

Let $\R$ denote the set of all real numbers.
\begin{zed}
	[\R]
\end{zed}

\subsection{\zcmd{zeroR} and \zcmd{oneR}}

Let $\zeroR$ and $\oneR$ denote the zero and unit elements of the real numbers.

\begin{axdef}
	\zeroR: \R \\
	\oneR: \R
\end{axdef}

\subsection{\zcmd{Rnz}}

Let $\Rnz$ denote the set of non-zero real numbers, 
also referred to as the {\it punctured} real number line.

\begin{zed}
	\Rnz == \R \setminus \{ \zeroR \}
\end{zed}

\subsection{\zcmd{addR}, \zcmd{subR}, \zcmd{mulR}, and \zcmd{divR}}

Let $x \addR y$, $x \subR y$, $x \mulR y$, and $x \divR y$ denote the usual arithmetic operations of
addition, subtraction, multiplication, and division.

\begin{axdef}
	\_ \addR \_: \R \cross \R \fun \R \\
	\_ \subR \_: \R \cross \R \fun \R \\
	\_ \mulR \_: \R \cross \R \fun \R \\
	\_ \divR \_: \R \cross \Rnz \fun \R
\end{axdef}

\subsection{\zcmd{negR}}

Let $\negR~x$ denote the negative of $x$.

\begin{axdef}
	\negR: \R \fun \R
\where
	\forall x : \R @ \\
	\t1	\negR~x = \zeroR \subR x
\end{axdef}

\subsection{\zcmd{ltR}, \zcmd{leR}, \zcmd{gtR}, and \zcmd{geR}}

Let $x \ltR y$, $x \leR y$, $x \gtR y$, and $x \geR y$ denote the usual comparison relations.

\begin{axdef}
	\_ \ltR \_: \R \rel \R \\
	\_ \leR \_: \R \rel \R \\
	\_ \gtR \_: \R \rel \R \\
	\_ \geR \_: \R \rel \R
\end{axdef}

\subsection{\zcmd{absR}}

Let $\absR(x)$ denote $\abs{x}$, the absolute value of $x$.

\begin{axdef}
	\absR: \R \fun \R
\where
	\forall x: \R @ \\
	\t1	\absR(x) = \IF x \geR \zeroR \THEN x \ELSE \negR~x
\end{axdef}

\subsection{\zcmd{Rpos}}

Let $\Rpos$ denote the set of positive real numbers.

\begin{zed}
	\Rpos == \{~ x: \R | x \gtR \zeroR ~\}
\end{zed}

\subsection{\zcmd{sqrtR}}

For non-negative $x$, let $\sqrtR(x)$ denote $\sqrt{x}$, the non-negative square root of $x$.

\begin{axdef}
	\sqrtR: \R \pfun \R
\where
	\sqrtR = \{~ x: \R | x \geR \zeroR @ x \mulR x \mapsto x ~\}
\end{axdef}

\section{Open Sets}

\subsection{\zcmd{intervalR}}

For any real numbers $a$ and $b$, let $\intervalR(a,b)$ denote $(a,b)$, the open interval bounded by $a$ and $b$.

\begin{axdef}
	\intervalR: \R \cross \R \fun \power \R
\where
	\forall a, b: \R @ \\
	\t1	\intervalR(a,b) = \{~ x: \R | a \ltR x \ltR b ~\}
\end{axdef}

\begin{remark}
If $a \geR b$ then $\intervalR(a,b) = \emptyset$.
\end{remark}

\subsection{\zcmd{ballR}}

For any real numbers $x$ and $r$, let $\ballR(x,r)$ denote the set of all real numbers within distance $r$ of $x$.

\begin{axdef}
	\ballR: \R \cross \R \fun \power \R
\where
	\forall x, r: \R @ \\
	\t1	\ballR(x,r) = \{~ x': \R | \absR(x' \subR x) \ltR r ~\}
\end{axdef}

\begin{remark}
Balls are intervals.

\begin{zed}
	\forall x, r: \R @ \\
	\t1	\ballR(x,r) = \intervalR(x \subR r, x \addR r)
\end{zed}

\end{remark}

\begin{remark}
If $r \gtR 0$ then $x$ is in $\ballR(x, r)$. 

\begin{zed}
	\forall x: \R; r: \Rpos @ \\
	\t1	x \in \ballR(x,r)
\end{zed}

\end{remark}

\begin{remark}
If $r \leR 0$ then $\ballR(x,r)$ is empty.

\begin{zed}
	\forall x, r: \R | r \leR \zeroR @ \\
	\t1	\ballR(x,r) = \emptyset
\end{zed}

\end{remark}

\subsection{\zcmd{ballsR}}

Let $\ballsR$ denote the set of all open balls in $\R$.

\begin{axdef}
	\ballsR: \family~\R
\where
	\ballsR = \{~ x, r: \R @ \ballR(x,r) ~\}
\end{axdef}

\subsection{\zcmd{openR}}

A subset $U$ of $\R$ is said to be {\it open} if every point $x \in U$ is surrounded by some open ball $B \subset U$ that lies
strictly within $U$.
Let $\openR$ denote the set of all open subsets of $\R$.

\begin{axdef}
	\openR: \family~\R
\where
	\openR = \{~ U:  \power \R | \\
	\t1	\forall x: U @ \\
	\t2		\exists B: \ballsR @ x \in B \subset U ~\}
\end{axdef}

\begin{remark}
All balls are open.

\begin{zed}
	\ballsR \subset \openR
\end{zed}

\end{remark}

\begin{remark}
The empty set is open.

\begin{zed}
	\emptyset \in \openR
\end{zed}

\end{remark}

\begin{remark}
The set of all real numbers is open.

\begin{zed}
	\R \in \openR
\end{zed}

\end{remark}

\subsection{\zcmd{tauR}}

The topology generated by the open balls of $\R$ is referred to as the {\it usual} or {\it standard} topology on $\R$ .
Let $\tauR$ denote the usual topology on $\R$.

\begin{axdef}
	\tauR: top[\R]
\where
	\tauR = topGen[\R] \ballsR
\end{axdef}

\begin{remark}

\begin{zed}
	\tauR = \openR
\end{zed}

\end{remark}

\subsection{\zcmd{Rtau}}

Let $\Rtau$ denote the topological space defined by $\R$ with the usual topology.

\begin{axdef}
	\Rtau: topSpaces[\R]
\where
	\Rtau = (\R, \tauR)
\end{axdef}

\begin{example}

\begin{zed}
	\Rtau \in topSpace[\R]
\end{zed}

\end{example}

\subsection{\zcmd{neighR}}

Let $x$ be a real number.
Any open set that contains $x$ is called a {\it neighbourhood} of it.
Let $\neighR(x)$ denote the set of all neighbourhoods of $x$.

\begin{axdef}
	\neighR: \R \fun \family~\R
\where
	\forall x: \R @ \\
	\t1	\neighR(x) = \{~ U: \openR | x \in U~\}
\end{axdef}

Clearly, every real number has an infinity of neighbourhoods.

\begin{remark}
Any open ball that contains $x$ is a neighbourhood of $x$.

\begin{zed}
	\forall x: \R; B: \ballsR | \\
	\t1	x \in B @ \\
	\t2		B \in \neighR(x)
\end{zed}

\end{remark}

\section{Functions}

The following sections define continuity, limits, and differentiability, which are point-wise properties of functions.
These properties are {\it local} in the sense that in order to determine if they hold at a given point it is sufficient to
consider the restriction of the function to an arbitrarily small neighbourhood of the point.
It is therefore useful to first introduce the set of {\it locally defined} functions, 
namely those functions that are defined in some neighbourhood of each point of  their domains.

\subsection{\zcmd{FunR}}

For $x$ a real number,
let $\FunR(x)$ denote the set of all real-valued, partial functions on $\R$ that are locally defined at $x$.

\begin{axdef}
	\FunR: \R \fun \power(\R \pfun \R)
\where
	\forall x: \R @ \\
	\t1	\FunR(x) = \{~ f: \R \pfun \R | \exists U: \neighR(x) @ U \subseteq \dom f ~\}
\end{axdef}

\begin{remark}
The function $\sqrtR$ is not locally defined at $0$ because it's defined only for non-negative numbers,
but every neighbourhood of $0$ contains some negative numbers.

\begin{zed}
	\sqrtR \notin \FunR(\zeroR)
\end{zed}

\end{remark}

\subsection{\zcmd{FunPR}}

For $U$ a subset of $\R$,
let $\FunPR(U)$ denote the set of all real-valued functions on $U$ that are locally defined at each point of $U$.

\begin{axdef}
	\FunPR: \power \R \fun \power (\R \pfun \R)
\where
	\forall U: \power \R @ \\
	\t1	\FunPR(U) = \{~ f: U \fun \R | \forall x: U @ f \in \FunR(x) ~\}
\end{axdef}

\begin{remark}
If $f \in \FunPR(U)$ then $U \in \openR$.

\begin{zed}
	\forall U: \power \R @ \\
	\t1	\FunPR(U) \neq \emptyset \implies U \in \openR
\end{zed}

\end{remark}

\section{Continuity}

Let $f$ be a real-valued partial function on $\R$ 
that is locally defined at $x$ and let $U$ be a neighbourhood of $x$
contained within the domain of $f$.
The function $f$ is said to be {\it continuous} at $x$ if 
for any $\epsilon > 0$ there is some $\delta > 0$ for which 
$f(x')$ is always within $\epsilon$ of $f(x)$
when $x' \in U$ is within $\delta$ of $x$.
\begin{argue}
\forall \epsilon > 0 @ \exists \delta > 0 @ \forall x' \in U @ \\
\t1	\abs{x' - x} < \delta \implies \abs{f(x') - f(x)} < \epsilon
\end{argue}

\begin{schema}{RealContinuous}
	f: \R \pfun \R \\
	x: \R
\where
	f \in \FunR(x)
\also
	\forall \epsilon: \Rpos @ \exists \delta: \Rpos@ \forall x': \dom f @ \\
	\t1	\absR(x' \subR x) \ltR \delta \implies \absR(f(x') \subR f(x)) \ltR \epsilon
\end{schema}

\subsection{\zcmd{CzeroR}}

Let $\CzeroR(x)$ denote the set of all real-valued partial functions on $\R$ that are continuous at $x$.
\begin{axdef}
	\CzeroR: \R \fun \power(\R \pfun \R)
\where
	\forall x: \R @ \\
	\t1	\CzeroR(x) = \{~ f: \FunR(x) | RealContinuous ~\}
\end{axdef}

\subsection{\zcmd{CzeroPR}}

Let $U$ be any subset of $\R$. 
Define $\CzeroPR(U)$ to be the set of all real-valued functions on $U$ that are continuous at each point in $U$.

\begin{axdef}
	\CzeroPR: \power \R \fun \power (\R \pfun \R)
\where
	\forall U: \power \R @ \\
	\t1	\CzeroPR(U) = \{~ f: \FunPR(U) | \forall x: U @ f \in   \CzeroR(x) ~\}
\end{axdef}

\begin{remark}
If $f \in \CzeroPR(U)$ then $U$ is open.

\begin{zed}
	\forall U: \power \R @ \\
	\t1	\CzeroPR(U) \neq \emptyset \implies U \in \openR
\end{zed}

\end{remark}

\begin{remark}
The $\epsilon-\delta$ definition of continuity given above is compatible with the definition
of continuity for mappings between topological spaces when we consider $\Rtau$, the usual topology on $\R$,
and $\Rtau \inducedTopSp U$, the topology induced on the subset $U$.

\begin{zed}
	\forall U: \openR @ \\
	\t1	\CzeroPR(U) = \CzeroTT(\Rtau \inducedTopSp U, \Rtau)
\end{zed}

\end{remark}

\section{Limits}

Let $x$ and $l$ be real numbers and
let $f$ be a real-valued partial function on $\R$ that is defined everywhere in some
neighbourhood $U$ of $x$, except possibly at $x$.
The function $f$ is said to approach the limit $l$ at $x$ if $f \oplus \{ x \mapsto l \}$ is continuous at $x$.
$$
	\lim_{x' \to x}{f(x')} = l
$$

\begin{schema}{Limit}
	f: \R \pfun \R \\
	x, l: \R
\where
	f \oplus \{x \mapsto l\} \in \CzeroR(x)
\end{schema}

\subsection{\zcmd{limRR}}

Let $\limRR(x,l)$ denote the set of all real-valued partial functions on $\R$ that approach the limit $l$ at $x$.

\begin{axdef}
	\limRR: \R \cross \R \fun \power(\R \pfun \R)
\where
	\forall x, l: \R @ \\
	\t1	\limRR(x,l) = \{~ f: \R \pfun \R | Limit ~\}
\end{axdef}

\begin{theorem}
If a function $f$ approaches some limit at $x$ then that limit is unique.
\begin{zed}
	\forall x, l, l': \R @ \\
	\t1	\forall f : \limRR(x,l) \cap \limRR(x,l') @ \\
	\t2		l = l'
\end{zed}
\end{theorem}

\begin{proof}
Suppose we are given real numbers
\begin{argue}
	x, l, l' \in \R 
\end{argue}
and a function
\begin{argue}
	f \in \limRR(x,l) \cap \limRR(x,l')
\end{argue}
Let $\epsilon$ be any positive real number
\begin{argue}
	\epsilon > 0
\end{argue}
Since $f$ approaches limits $l$ and $l'$ at $x$ there exists a real number $\delta > 0$ such that
\begin{argue}
	\forall x' \in \R |  \\
	\t1	\zeroR \ltR \abs{x' \subR x}< \delta @ \\
	\t2		 \abs{f(x') - l} < \epsilon \land \abs{f(x') - l'} < \epsilon
\end{argue}
For any such real number $x'$ we have
\begin{argue}
	\abs{l' - l} \\
	\t1	= \abs{(f(x') - l) - (f(x') - l')} 			& add and subtract $f(x')$ \\
	\t1	\leq \abs{f(x') - l} + \abs{f(x') - l'} 	& triangle inequality \\
	\t1	= 2\epsilon					& definition of limits
\end{argue}
Since the above holds for any $\epsilon > 0$ we must have
\begin{argue}
	l = l'
\end{argue}

\end{proof}

\subsection{\zcmd{limFR}}

If $f$ approaches the limit $l$ at $x$ then let $\limFR(f,x)$ denote $l$.
By the preceding theorem, $\limFR(f,x)$ is well-defined when it exists.

\begin{axdef}
	\limFR: (\R \pfun \R) \cross \R \pfun \R
\where
	\limFR = \{~ Limit @ (f, x) \mapsto l ~\}
\end{axdef}

\section{Differentiability}

Let $f$ be a real-valued partial function on $\R$, let $x$ be a real number,
and let $f$ be defined on some neighbourhood $U$ of $x$.

The function $f$ is said to be {\it differentiable} at $x$ if the following limit holds for some number denoted by $f'(x)$.
$$
\lim_{h \to 0} \frac{f(x+h) - f(x)}{h} = f'(x)
$$

\begin{remark}
If $f$ is differentiable at $x$ then $f$ is continuous at $x$.
\end{remark}

The geometric intuition behind the concept of differentiability is that $f$ is differentiable at $x$
when, very near $x$, the graph of $f$ is approximately a straight line through the point $(x, f(x))$ with slope $f'(x)$.
$$
f(x + h) \approx f(x) + f'(x) h \quad \text{when} \quad \abs{h} \approx 0
$$
The slope $f'(x)$ is called the {\it derivative} of $f$ at $x$
and $f'$ is called the {\it derived function}.

We can read this definition as saying that the approximate slope function $m(h)$ defined for 
small enough, non-zero values of $h$ by
$$
	m(h) = \frac{f(x + h) - f(x)}{h}
$$
approaches the limit $l = f'(x)$ as $h \to 0$.
$$
	\lim_{h\to 0}{m(h)} = l = f'(x)
$$

\begin{schema}{Differentiable}
	f: \R \pfun \R \\
	x, l: \R
\where
	f \in \CzeroR(x)
\also
	\LET m == (\lambda h: \Rnz | x \addR h \in \dom f @ (f(x \addR h) \subR f(x)) \divR h) @ \\
	\t1	\limFR(m, \zeroR) = l
\end{schema}

\begin{remark}
If $f$ is differentiable at $x$ then the limit $l$ is unique.
\end{remark}

\subsection{\zcmd{diffRR}}

Let $\diffRR(x,l)$ denote the set of all functions $f$ that are differentiable at $x$ with $f'(x) = l$.

\begin{axdef}
	\diffRR: \R \cross \R \fun \power(\R \pfun \R)
\where
	\forall x, l: \R @ \\
	\t1	\diffRR(x, l) = \{~ f: \R \pfun \R | Differentiable ~\}
\end{axdef}

\subsection{\zcmd{diffR}}

Let $\diffR(x)$ denote the set of all functions that are differentiable at $x$.

\begin{axdef}
	\diffR: \R \fun \power(\R \pfun \R)
\where
	\forall x: \R @ \\
	\t1	\diffR(x) = \bigcup \{~ l: \R @ \diffRR(x,l) ~\}
\end{axdef}

\subsection{\zcmd{diffPR}}

Let $U$ be any subset of $\R$. 
Let $\diffPR(U)$ denote the set of all functions on $U$
that are differentiable at each point of $U$.

\begin{axdef}
	\diffPR: \power \R \fun \power(\R \pfun \R)
\where
	\forall U: \power \R @ \\
	\t1	\diffPR(U) = \{~ f: \CzeroPR(U) | \forall x: U @ f \in \diffR(x) ~\}
\end{axdef}

\section{Derivatives}

\subsection{\zcmd{derivFR}}

Let $\derivFR(f,x)$ denote $f'(x)$, the derivative of $f$ at $x$.

\begin{axdef}
	\derivFR: (\R \pfun \R) \cross \R \pfun \R
\where
	\derivFR = \{~ Differentiable @ (f,x) \mapsto l ~\}
\end{axdef}

\subsection{\zcmd{derivF}}

Let $\derivF(f)$ denote $f'$, the derived function.

\begin{axdef}
	\derivF: (\R \pfun \R) \fun (\R \pfun \R)
\where
	\forall f: \R \pfun \R @ \\
		\t1	\derivF f = (\lambda x: \R | f \in \diffR(x) @ \derivFR(f,x)) 
\end{axdef}

\begin{remark}
If $f$ is differentiable on $U$ then $f'$ is not necessarily continuous on $U$.
Counterexamples exist.
\end{remark}

\begin{remark}
If $f$ is uniformly differentiable on $U$ then $f'$ is continuous on $U$.
A further discussion of uniform differentiability is beyond the scope of this article.
\end{remark}

\section{Higher Order Derivatives}

Let $n$ be a natural number and let $x$ be a real number.
In differential geometry we normally deal with $C^n(x)$, the set of functions
that possess continuous derivatives of order $0, \ldots, n$ at $x$.

\subsection{\zcmd{CnR}}

Let $\CnR(n,x)$ denote the set of all functions that have continuous derivatives of order $0, \ldots, n$ at $x$.

\begin{axdef}
	\CnR: \nat \cross \R \fun \power(\R \pfun \R)
\where
	\forall x: \R @ \\
	\t1	\CnR(0,x) = \CzeroR(x)
\also
	\forall n: \nat; x: \R @ \\
	\t1	\CnR(n + 1, x) = \{~ f: \diffR(x) | \derivF f \in \CnR(n,x) ~\}
\end{axdef}

\subsection{\zcmd{CnPR}}

Let $n$ be a natural number and let $U$ be a subset of $\R$.
Let $\CnPR(n,U)$ denote the set of all functions on $U$ that have continuous derivatives of order $0, \ldots, n$
at every point of $U$.

\begin{axdef}
	\CnPR: \nat \cross \power \R \fun \power(\R \pfun \R)
\where
	\forall n: \nat; U: \power \R @ \\
	\t1	\CnPR(n,U) = \{~ f: \FunPR(U) | \forall x: U @ f \in \CnR(n,x) ~\}
\end{axdef}

\section{Smoothness}

\subsection{\zcmd{smoothR}}

A function is said to be {\it smooth} if it possesses continuous derivatives of all orders.
Let $x$ be a real number.
Let $\smoothR(x)$ denote the set of all functions that are smooth at $x$.

\begin{axdef}
	\smoothR: \R \fun \power(\R \pfun \R)
\where
	\forall x: \R @ \\
	\t1	\smoothR(x) = \{~ f: \FunR(x) | \forall n: \nat @ f \in \CnR(n, x) ~\}
\end{axdef}

\subsection{\zcmd{smoothPR}}

Let $\smoothPR(U)$ denote the set of all functions on $U$ that are smooth at every point of $U$.

\begin{axdef}
	\smoothPR: \power \R \fun \power (\R \pfun \R)
\where
	\forall U: \power \R @ \\
	\t1	\smoothPR(U) = \{~ f: \FunPR(U) | \forall x: U @ f \in \smoothR(x) ~\}
\end{axdef}

\printbibliography

\end{document}