\documentclass{amsart}

\usepackage{tikz-cd}
\usepackage{mathz-sets}
\usepackage{mathz-categories}
\usepackage{mathz-preamble}

\addbibresource{../../shared/references.bib}

\begin{document}

\title{Categories}
\author{Arthur Ryman}
\email[Arthur Ryman]{arthur.ryman@gmail.com}
\date{\today}

\begin{abstract}
    This article contains \ZN\cite{spivey-zrm} definitions for
    concepts related to categories.
    It has been type checked by \fuzz\cite{spivey-fm}.
\end{abstract}

\maketitle

\tableofcontents

\section{Introduction}

The definitions in this article are primarily based on those in \cite{maclane-cftwm}.
Wikipedia and other sources will be used as needed.

Category theory provides a useful conceptual framework for mathematics.
It abstracts and generalizes many concepts and constructions that occur in other branches.
For example, maps between sets, homomorphisms between groups, and linear transformations between vector spaces
all form categories.
The practical utility of category derives from the many examples that occur throughout mathematics.
This situation presents a small dilemma for the scope of this article.
Should this article treat categories as foundational or advanced?

I have taken the position that this article should be foundational and 
should therefore not depend on any other articles.
Accordingly, the definitions presented here will not be illustrated with formal examples from
other articles.
For example, although this article does assert that homomorphisms between groups form a category, 
it does not use the formal definition of group or homomorphism.
Such a formal use will be deferred to other articles.
This approach allows other, more advanced, articles to reference the foundational definitions 
contained here without introducing circularity.

\section{Categories, Functors, and Natural Transformations}

\subsection{Axioms for Categories}

Mac Lane\cite{maclane-cftwm} defines the concepts of \textit{metagraph}, \textit{metacategory}, \textit{large set},  \textit{small set}, and others 
in order to avoid the well-known paradoxes of set theory.
However, these concepts are unnecessary when using \ZN\ which uses \textit{simple type theory} to avoid the paradoxes.
Indeed, simple type theory was conceived by Russel specifically to put set theory on a firmer foundation.

However, there is no free lunch.
The price one pays when using \ZN\ is to explicitly parameterize generic definitions with given sets.
Specifically, a category generically depends on two given sets, one for its objects and another for its arrows.

A \textit{graph} consists of \textit{objects} $a, b, c, \dots$, \textit{arrows} $f, g, h, \dots$, 
and two operations, \textit{domain} and \textit{codomain}, that assign objects to arrows.
Arrows are also referred to as \textit{morphisms}.
The domain and codomain of an arrow are also referred to as its \textit{source} and \textit{target}.
\begin{schema}{Category\_Graph0}[\genO, \genA]
	objects: \power \genO \\
	arrows: \power \genA \\
	domain, codomain: \genA \pfun \genO
\where
	domain \in arrows \fun objects
\also
	codomain \in arrows \fun objects
\end{schema}

An arrow $f$ with domain $a$ and codomain $b$ is diagrammed as an arrow pointing from $a$ to $b$.
$$
 \begin{tikzcd}
	a \arrow{r}{f} & b 
\end{tikzcd}
$$

The domain of the arrow $f$ is donated $\domCat f$ and its codomain is denoted $\codCat f$.
The set of all arrows with domain $a$ and codomain $b$ is denoted $a \arrCat b$.

\begin{schema}{Category\_Graph}[\genO, \genA]
	Category\_Graph0[\genO, \genA] \\
	\domCat, \codCat: \genA \pfun \genO \\
	\_ \arrCat \_: \genO \cross \genO \pfun \power \genA
\where
	\domCat = domain
\also
	\codCat = codomain
\also
	(\_ \arrCat \_) = (\lambda a, b: objects @ \{~ f: arrows | \domCat f = a \land \codCat f = b ~\})
\end{schema}

A \textit{category} is a graph with two additional operations, \textit{identity} and \textit{composition}.

The identity operation maps each object to its \textit{identity arrow}.
\begin{schema}{Category\_Identity0}[\genO, \genA]
	Category\_Graph[\genO, \genA] \\
	identity: \genO \pfun \genA
\where
	identity \in objects \fun arrows
\end{schema}

The identity arrow for the object $a$ is denoted $\idCat a$.
\begin{schema}{Category\_Identity}[\genO, \genA]
	Category\_Identity0[\genO, \genA] \\
	\idCat: \genO \pfun \genA
\where
	\idCat = identity
\end{schema}

A pair of arrows $g, f$ is \textit{composable} if the domain of $g$ is the codomain of $f$.
Composition maps each composable pair of arrows $g: b \arrCat c$ and $f: a \arrCat b$ to an arrow in $a \arrCat c$.
\begin{schema}{Category\_Composition0}[\genO, \genA]
	Category\_Graph[\genO, \genA] \\
	composition: \genA \cross \genA \pfun \genA
\where
	composition \in arrows \cross arrows \pfun arrows
\also
	\dom composition = \{ g, f: arrows | \domCat g = \codCat f ~\}
\also
	\forall f, g:arrows | \\
	\t1	(g, f) \in \dom composition @ \\
	\t2		composition(g, f) \in \domCat f \arrCat \codCat g
\end{schema}

The composition $h: a \arrCat c$ of arrows $g: b \arrCat c$ and $f: a \arrCat b$ is diagrammed as a triangle whose sides are $f, g, h$.
$$
  \begin{tikzcd}
    a \arrow{r}{f} \arrow[swap]{dr}{h} & b \arrow{d}{g} \\
     & c
  \end{tikzcd}
$$

If arrows $g, f$ are composable their composition is denoted $g \compCat f$.
\begin{schema}{Category\_Composition}[\genO, \genA]
	Category\_Composition0[\genO, \genA] \\
	\_ \compCat \_: \genA \cross \genA \pfun \genA
\where
	(\_ \compCat \_) = composition
\end{schema}

The composition and identity operations satisfy \textit{associativity} and the \textit{unit law}.
\begin{schema}{Category\_Associativity}[\genO, \genA]
	Category\_Composition[\genO, \genA]
\where
	\forall a, b, c, d: objects; f, g, k: arrows | \\
	\t1	domain~f = a \land codomain~f = b \land \\
	\t1	domain~g = b \land codomain~g = c \land \\
	\t1	domain~k = c \land codomain~k = d @\\
	\t2		composition(k, composition(g, f)) = composition(composition(k, g), f)	
\end{schema}
TODO: simplify this definition using notation

\begin{schema}{Category\_UnitLaw}[\genO, \genA]
	Category\_Identity[\genO, \genA] \\
	Category\_Composition[\genO, \genA]
\where
	\forall a, b, c: objects; f, g: arrows | \\
	\t1	domain~f = a \land codomain~f = b \land \\
	\t1	domain~g = b \land codomain~g = c @ \\
	\t2		composition(identity~b, f) = f \land \\
	\t2		composition(g, identity~b) = g
\end{schema}

\begin{schema}{Category}[\genO, \genA]
	Category\_Associativity[\genO, \genA] \\
	Category\_UnitLaw[\genO, \genA]
\end{schema}

TODO: Define notations for the above.

TODO: Include commutative diagrams for the above.

\printbibliography

\end{document}
