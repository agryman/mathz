\documentclass[11pt, oneside]{article}

\usepackage{../../shared/preamble}
\addbibresource{../../shared/references.bib}

\usepackage{../sets/sets}
\usepackage{../groups/groups}
\usepackage{../topological-spaces/topological-spaces}
\usepackage{../real-numbers/real-numbers}
\usepackage{vector-spaces}

\title{Vector Spaces}
\author{Arthur Ryman, {\tt arthur.ryman@gmail.com}}
\date{\today}

\begin{document}

\maketitle

\begin{abstract}
This article contains Z Notation type declarations for vector spaces and some related objects.
It has been type checked by \fuzz.
\end{abstract}

\section{Real Vector Spaces}

Real vector spaces are multidimensional generalizations of real numbers.
They are the objects studied in linear algebra and are foundational to differential geometry.

In the following let $\genT$ denote a set of elements which we'll refer to as {\em vectors}
and let $A$ denote an Abelian group over the vectors in which the binary operation is denoted as addition.
Let $v$ and $w$ denote vectors and
and let $x$ and $y$ denote real numbers.

\subsection{Notation for Vector Addition, Zero, and Negative: \zcmd{addV}, \zcmd{zeroV}, and \zcmd{negV}}

Let $v \addV w$ denote vector addition,
let $\zeroV$ denote the zero vector,
and let $\negV v$ denote the negative vector.

\subsection{Real Scalar Multiplication: \zcmd{mulS}, \zcmd{timesS}, and $RealScalarMultiplication$}

A {\em real scalar multiplication} operation on the vectors is an operation $smul$ 
that maps the pair $(x, v)$ to another vector, typically denoted $x \mulS v$ or $x \timesS y$,
such that
multiplication by 0 maps all vectors to the group identity element,
multiplication by 1 maps each vector to itself,
multiplication preserves group addition, 
and multiplication distributes over both real and group addition.

Let $RealScalarMultiplication$ denote this situation.

\begin{schema}{RealScalarMultiplication}[\genT]
A: \abgroup \genT \\
smul: \R \cross \genT \fun \genT
\where
\LET (\_ \addV \_) == A; \\
\t1	\zeroV == identity\_element~A; \\
\t1	(\_ \mulS \_) == smul @ \\
\t2		\forall x, y: \R; v, w: \genT @ \\
\t3			\zeroR \mulS v = \zeroV \land \\
\t3			\oneR \mulS v = v \land \\
\t3			(x \mulR y) \mulS v = x \mulS (y \mulS v) \land \\
\t3			(x \addR y) \mulS v = x \mulS v \addV y \mulS v \land \\
\t3			x \mulS (v \addV w) = x \mulS v \addV x \mulS w
\end{schema}

\begin{itemize}
	\item Multiplying by $\zeroR$ gives the zero vector.
	\item Multiplying by $\oneR$ gives the same vector.
	\item Scalar multiplication is associative.
	\item Scalar addition distributes over scalar multiplication.
	\item Vector addition distributes over scalar multiplication.
\end{itemize}

\subsection{The Set of All Real Vector Spaces: \zcmd{vecR}}

A {\em real vector space} is a pair $(A, smul)$ where $A$ is an Abelian group and $smul$
is a real scalar multiplication on the elements of $A$.
The elements of $A$ are referred to as \textit{vectors}.

Let $\vecR \genT$ denote the set of all real vector spaces over $\genT$,

\begin{zed}
\vecR \genT == \{~ RealScalarMultiplication[\genT] @ (A, smul) ~\}
\end{zed}

\subsection{Real Linear Transformations: $RealLinearTransformation$}

Let $V_1$ and $V_2$ be real vector spaces and let $f$ be a homomorphism of the underlying
Abelian groups.
The map $f$ is said to be a {\em linear transformation} if $f$ maps scalar multiples of vectors to the
scalar multiple of the mapped vectors.

Let $RealLinearTransformation$ denote this situation.

\begin{schema}{RealLinearTransformation}[\genT, \genU]
f: \genT \fun \genU \\
V_1: \vecR \genT \\
V_2: \vecR \genU
\where
\LET A_1 == first~V_1; (\_ \mulS \_) == second~V_1; \\
\t1	A_2 == first~V_2; (\_ \timesS \_) == second~V_2 @ \\
\t2		f \in \homGroup(A_1, A_2) \land \\
\t2		(\forall x: \R; v: \genT @ \\
\t3			f(x \mulS v) = x \timesS(f~v))
\end{schema}

\begin{itemize}
\item The vector space $V_1$ has Abelian group $A_1$ and scalar multiplication $(\_ \mulS \_)$.
\item The vector space $V_2$ has Abelian group $A_2$ and scalar multiplication $(\_ \timesS \_)$.
\item The map $f$ is a homomorphism of the underlying Abelian groups.
\item The map $f$ maps scalar multiples of vectors in $\genT$ to scalar multiples of the mapped vectors in $\genU$.
\end{itemize}

\subsection{The Set of All Real Linear Transformations: \zcmd{homVecR}}

Let $V_1$ and $V_2$ be real vector spaces.
Let $\homVecR(V_1, V_2)$ denote the set of all linear transformations from $V_1$ to $V_2$.
A linear transformation is also referred to as a \textit{homomorphism} of vector spaces.

\begin{zed}
\homVecR[\genT, \genU] == \\
\t1	(\lambda V_1: \vecR \genT; V_2: \vecR \genU @ \\
\t2		\{~ f: \genT \fun \genU | \\
\t3			RealLinearTransformation[\genT, \genU] ~\})
\end{zed}

\section{Real $n$-tuples}

The preceding section described real vector spaces abstractly.
In this section we define a family of finite-dimensional real vector spaces
whose elements are finite sequences of real numbers, also referred to as \textit{real tuples}.

\subsection{The Set of All Finite Sequences of Real Numbers: \zcmd{Rinf}}

Let $n$ be a natural number.
A finite sequence of $n$ real numbers is called a {\it real $n$-tuple}.
Let $\Rinf$ denote the set of all real $n$-tuples for any $n$.

\begin{zed}
	\Rinf == \seq \R
\end{zed}

\subsection{The Component Projection Function: \zcmd{piRinf}}

The real numbers that comprise an $n$-tuple are called its \textit{components}.
Let $v$ be a real $n$-tuple and let $i$ be an integer where $1 \le i \le n$.
The real number $v(i)$ is the $i$-th component of $v$.
Let $\piRinf(i)$ be the projection function that maps an $n$-tuple $v$ to its $i$-th component $v(i)$.

\begin{axdef}
	\piRinf: \nat_1 \fun \Rinf \pfun \R
\where
	\forall i: \nat_1 @ \\
	\t1	\piRinf(i) = (\lambda v: \Rinf | i \in \dom v @ v(i))
\end{axdef}

\subsection{The Set of All Well-Dimensioned Subsets of $\Rinf$: \zcmd{DeltaRinf}}

A non-empty subset of $\Rinf$ is said to be \textit{well-dimensioned} if each of its elements has the same number
of components.
Let $\DeltaRinf$ denote the family of all well-dimensioned subsets of $\Rinf$ .

\begin{axdef}
	\DeltaRinf: \family~\Rinf
\where
	\DeltaRinf = \{~ S: \power_1 \Rinf | \forall v, w: S @ \# v = \# w ~\}
\end{axdef}

\subsection{The Dimension of a Well-Dimensioned Set of Tuples: \zcmd{dimRinf}}

Let $S \in \DeltaRinf$ be a well-dimensioned set of tuples.
The number of components of each tuple in $S$ is called its \textit{dimension}.
Let $\dimRinf(S)$ denote the dimension of $S$.

\begin{axdef}
\dimRinf: \DeltaRinf \fun \nat
\where
%	\forall n: \nat @ \forall U: \power(\Rtup(n)) @ \\
%	\t1	\dimRinf(U) = n
\forall S: \DeltaRinf @ \\
\t1	\dimRinf S = (\mu v: S @ \# v)
\end{axdef}

\subsection{The Set of All Compatible Pairs of Tuples: \zcmd{RinfDelta}}

The pair of real tuples $(v, w)$ is said to be \textit{compatible} if each member has the same number of components.
Let $\RinfDelta$ denote the set of all compatible pairs of real tuples.
If the pair $(v, w)$ is compatible then $v$ and $w$ are said to be compatible with each other.

\begin{axdef}
	\RinfDelta: \Rinf \rel \Rinf
\where
	\RinfDelta = \{~ v, w: \Rinf | \# v = \# w ~\}
\end{axdef}

\subsection{Addition of Compatible Tuples: \zcmd{addRinf}}

Let $v$ and $w$ be $n$-tuples.
Vector addition of $v$ and $w$ is the $n$-tuple $v \addRinf w$ defined by component-wise addition.

\begin{axdef}
	\_ \addRinf \_: \RinfDelta \fun \Rinf
\where
	\langle \rangle \addRinf \langle \rangle = \langle \rangle
\also
\forall n: \nat_1; v, w: \Rinf | n = \#v = \# w @ \\
\t1	v \addRinf w = (\lambda i: 1 \upto n @ v~i \addR w~i)
\end{axdef}

\subsection{Subtraction of Compatible Tuples: \zcmd{subRinf}}

Vector subtraction is defined similarly.

\begin{axdef}
	\_ \subRinf \_: \RinfDelta \fun \Rinf
\where
	\langle \rangle \subRinf \langle \rangle = \langle \rangle
\also
\forall n: \nat_1; v, w: \Rinf | n = \#v = \# w @ \\
\t1	v \subRinf w = (\lambda i: 1 \upto n @ v~i \subR w~i)
\end{axdef}

\subsection{The Negative of a Tuple: \zcmd{negRinf}}

Let $\negRinf v$ denote the negative of $v$.

\begin{axdef}
\negRinf: \Rinf \fun \Rinf
\where
\negRinf \langle \rangle = \langle \rangle
\also
\forall n: \nat_1; v: \Rinf | n = \# v @ \\
\t1	\negRinf v = (\lambda i: 1 \upto n @ \negR(v~i))
\end{axdef}

\subsection{Scalar Multiplication of a Tuple: \zcmd{smulRinf}}

Let $v$ be an $n$-tuple and let $c$ be a real number.
Scalar multiplication of $v$ by $c$ is the $n$-tuple $c \smulRinf v$ defined by component-wise multiplication.

\begin{axdef}
\_ \smulRinf \_ : \R \cross \Rinf \fun \Rinf 
\where
\forall c: \R @ \\
\t1	c \smulRinf \langle \rangle = \langle \rangle
\also
\forall c: \R; n: \nat_1; v: \Rinf | n = \# v @ \\
\t1	c \smulRinf v = (\lambda i: 1 \upto n @ c \mulR (v~i))
\end{axdef}

\begin{remark}
Scalar multiplication is associative in the sense that $(a \mulR b) \smulRinf v = a \smulRinf (b \smulRinf v)$

\begin{zed}
	\forall a, b: \R; v: \Rinf @ \\
	\t1	(a \mulR b) \smulRinf v = a \smulRinf (b \smulRinf v)
\end{zed}

\end{remark}

\subsection{The Set of All Real $n$-tuples: \zcmd{Rtup}}

Let $\Rtup(n)$ denote $\R^n$, the set of all $n$-tuples for some given $n$.
\begin{axdef}
	\Rtup: \nat \fun \power \Rinf
\where
	\forall n: \nat @ \\
	\t1	\Rtup(n) = \{~ v: \Rinf | \# v = n ~\}
\end{axdef}

\begin{remark}

\begin{zed}
	\Rinf = \bigcup \{~ n: \nat @ \Rtup(n) ~\}
\end{zed}

\end{remark}

\begin{remark}
The subset $\Rtup(n)$ is well-dimensioned.

\begin{zed}
	\forall n: \nat @ \\
	\t1	\Rtup(n) \in \DeltaRinf
\end{zed}

\end{remark}

\begin{remark}
The dimension of $\Rtup(n)$ is $n$.

\begin{zed}
	\forall n: \nat @ \\
	\t1	\dimRinf(\Rtup(n)) = n
\end{zed}

\end{remark}

\subsection{Addition of $n$-tuples: $addRtup$}

Let $addRtup(n)$ denote the restriction of addition to $\Rtup(n)$.

\begin{zed}
addRtup == \\
\t1	(\lambda n: \nat @ \\
\t2		(\lambda v, w: \Rtup(n) @ v \addRinf w))
\end{zed}

\begin{example}
The binary operation $addRtup(n)$ defines an Abelian group over $\Rtup(n)$.

\begin{zed}
\forall n: \nat @ \\
\t1	addRtup(n) \in \abgroup(\Rtup(n))
\end{zed}

\end{example}

\subsection{Subtraction of $n$-tuples: $subRtup$}

Let $subRtup(n)$ denote the restriction of subtraction to $\Rtup(n)$.

\begin{zed}
subRtup == \\
\t1	(\lambda n: \nat @ \\
\t2		(\lambda v, w: \Rtup(n) @ v \subRinf w))
\end{zed}

\subsection{The Negative of an $n$-tuple: $negRtup$}

Let $negRtup(n)$ denote the restriction of the negative operation to $\Rtup(n)$.

\begin{zed}
negRtup == \\
\t1	(\lambda n: \nat @ \\
\t2		(\lambda v: \Rtup(n) @ \negRinf v))
\end{zed}

\begin{remark}
The operation $negRtup(n)$ is the inverse operation of the Abelian group $addRtup(n)$.

\begin{zed}
\forall n: \nat @ \\
\t1	negRtup(n) = inverse\_operation(addRtup(n))
\end{zed}

\end{remark}


\subsection{The Zero Real $n$-tuple: \zcmd{zeroRtup}}

Let $\zeroRtup(n)$ denote the $n$-tuple consisting of all zeroes.

\begin{axdef}
	\zeroRtup: \nat \fun \Rinf
\where
	\zeroRtup(0) = \langle \rangle
\also
	\forall n: \nat_1 @ \\
	\t1	\zeroRtup(n) = (\lambda i: 1 \upto n @ \zeroR)
\end{axdef}

\begin{remark}
Every component of $\zeroRtup(n)$ is $\zeroR$.

\begin{zed}
\forall n: \nat @ \\
\t1	\forall i: 1 \upto n @ \\
\t2	(\piRinf i)(\zeroRtup n) = \zeroR
\end{zed}

\end{remark}

\begin{remark}
The tuple $\zeroRtup(n)$ is in $\Rtup(n)$.

\begin{zed}
\forall n: \nat @ \\
\t1	\zeroRtup(n) \in \Rtup(n)
\end{zed}

\end{remark}

\begin{remark}
The tuple $\zeroRtup(n)$ is the identity element of the Abelian group $addRtup(n)$.

\begin{zed}
\forall n: \nat @ \\
\t1	\zeroRtup(n) = identity\_element(addRtup(n))
\end{zed}

\end{remark}

\subsection{Scalar Multiplication of an $n$-tuple: $smulRtup$}

Let $smulRtup(n)$ denote scalar multiplication restricted to $\Rtup(n)$.

\begin{zed}
smulRtup == \\
\t1	(\lambda n: \nat @ \\
\t2		(\lambda c: \R; v: \Rtup(n) @ c \smulRinf v))
\end{zed}

\subsection{The Real Vector Space of $n$-tuples: $vecRtup$}

Let $vecRtup(n)$ denote the real vector space of $n$-tuples.

\begin{zed}
vecRtup == \\
\t1	(\lambda n: \nat @ (addRtup(n), smulRtup(n)))
\end{zed}

\begin{remark}
The pair $vecRtup(n)$ defines a vector space over $\Rtup(n)$.

\begin{zed}
\forall n: \nat @ \\
\t1	vecRtup(n) \in \vecR(\Rtup(n))
\end{zed}

\end{remark}

\subsection{Linear Transformations of $n$-tuples: \zcmd{linRtup}}

Define $\linRtup(n,m)$ to be the set of all linear transformations from $\R^n$ to $\R^m$.
\begin{axdef}
	\linRtup: \nat \cross \nat \fun \power(\Rinf \pfun \Rinf)
\where
	\forall n,m: \nat @ \\
	\t1	\linRtup(n,m) = \homVecR(vecRtup(n), vecRtup(m))
\end{axdef}

\subsection{The Identity Transformation of $n$-tuples: \zcmd{idRtup}}

Let $\idRtup(n)$ denote the identity function on $\Rtup(n)$.

\begin{axdef}
	\idRtup: \nat \fun \Rinf \pfun \Rinf
\where
	\forall n: \nat @ \\
	\t1	\idRtup(n) = \id(\Rtup(n))
\end{axdef}

\begin{remark}
The function $\idRtup(n)$ is a linear transformation.

\begin{zed}
	\forall n: \nat @ \\
	\t1	\idRtup(n) \in \linRtup(n, n)
\end{zed}

\end{remark}

\section{The Metric Topology on Real $n$-tuples}

\subsection{The Dot Product of Tuples: \zcmd{dotRinf}}

The {\it inner} or {\it dot} product of $n$-tuples $v$ and $w$ is the real number $v \dotRinf w$ 
defined by the sum of the component-wise products.

\begin{axdef}
	\_ \dotRinf \_ : \RinfDelta \fun \R
\where
	\langle \rangle \dotRinf \langle \rangle = \zeroR
\also
	\forall x, y: \R; v, w: \Rinf | \# v = \# w @ \\
	\t1	(\langle x \rangle \cat v) \dotRinf (\langle y \rangle \cat w) = x \mulR y \addR v \dotRinf w
\end{axdef}

Each $\Rtup(n)$ is a real inner product space under the operation of dot product defined above.

\subsection{The Norm of a Tuple: \zcmd{normRinf}}

The norm $\norm{v}$ of the $n$-tuple $v$ is the positive square root of its dot product with itself.
$$
	\norm{v} = \sqrt{v \dotRinf v}
$$

Define $\normRinf(v)$ to be $\norm{v}$.
\begin{axdef}
	\normRinf: \Rinf \fun \R
\where
	\forall v: \Rinf @ \\
	\t1	\normRinf(v) = \sqrtR(v \dotRinf v)
\end{axdef}

The concepts of continuity, limits, and differentiability extend to functions between normed vector spaces such as $\R^n$.

\subsection{The Open Ball at a Tuple: \zcmd{ballRinf}}

Let $\ballRinf(v,r)$ denote the \textit{open ball} in $\Rtup(n)$ of radius $r  \in \Rpos$ centred at $v \in \Rtup(n)$.

\begin{axdef}
\ballRinf: \Rinf \cross \Rpos \fun \power \Rinf
\where
\forall v: \Rinf; r: \Rpos @ \\
\t1	\LET n == \# v @ \\
\t2		\ballRinf(v, r) = \{~ w: \Rtup(n) | \normRinf(v \subRinf w) \ltR r ~\}
\end{axdef}

\subsection{The Set of All Open Balls at an $n$-tuple: \zcmd{ballsRtup}}

Let $\ballsRtup(n)$ denote the family of all open balls in $\Rtup(n)$.

\begin{axdef}
	\ballsRtup: \nat \fun \family~\Rinf
\where
	\forall n: \nat @ \\
	\t1	\ballsRtup(n) =  \{~ v: \Rtup(n); r: \Rpos @ \ballRinf(v,r) ~\}
\end{axdef}

\begin{remark}
The set of all open balls in $\Rtup(n)$ is a family of sets in $\Rtup(n)$.

\begin{zed}
\forall n: \nat @ \\
\t1	\ballsRtup(n) \in \family(\Rtup(n))
\end{zed}

\end{remark}

\subsection{The Usual Topology on $n$-tuples: \zcmd{tauRtup}}

The \textit{usual topology} on $\Rtup(n)$ is the topology generated by the open balls in $\Rtup(n)$.
Let $\tauRtup(n)$ denote the usual topology on $\Rtup(n)$.

\begin{axdef}
	\tauRtup: \nat \fun \family~\Rinf
\where
	\forall n: \nat @ \\
	\t1	\tauRtup(n) = topGen[\Rtup(n)] (\ballsRtup(n))
\end{axdef}

\begin{remark}

If $n \in \nat$ then $\tauRtup(n)$ is a topology on $\Rtup(n)$.

\begin{zed}
	\forall n: \nat @ \tauRtup(n) \in top[\Rtup(n)]
\end{zed}
\end{remark}

\subsection{The Set of All Neighbourhoods of a Tuple: \zcmd{neighRinf}}

Let $v \in \Rtup(n)$. 
An open set $U$ in the usual topology $\tauRtup(n)$ that contains $v$
is called a \textit{neighbourhood} of $v$.
Let $\neighRinf(v)$ denote the set of all neighbourhoods of $x$.

\begin{axdef}
\neighRinf: \Rinf \fun \family~\Rinf
\where
\forall n: \nat; v: \Rinf | n = \#v @ \\
\t1	\neighRinf(v) = \{~ U: \tauRtup(n) | v \in U ~\}
\end{axdef}

\begin{remark}

The set of all neighbourhoods of $v \in \Rtup(n)$ is a family of sets in $\Rtup(n)$.

\begin{zed}
\forall n: \nat; v: \Rinf | n = \#v @ \\
\t1	\neighRinf(v) \in \family(\Rtup(n))
\end{zed}

\end{remark}

\subsection{The Topological Space of $n$-tuples: \zcmd{tsRtup}}

Let $\tsRtup(n)$ denote the topological space defined by the usual topology on $\Rtup(n)$.

\begin{axdef}
	\tsRtup: \nat \fun topSpaces[\Rinf]
\where
	\forall n: \nat @ \\
	\t1	\tsRtup(n) = (\Rtup(n), \tauRtup(n))
\end{axdef}

\section{Continuity}

\subsection{Real-Valued Functions That Are Continuous on the Set of All $n$-tuples: \zcmd{CzeroRtup}}

A function $f \in \R^n \fun \R$ is said to be \textit{continuous} if it is continuous with respect to the usual topologies
on $\R^n$ and $\R$.
Let $\CzeroRtup(n)$ denote the set of these continuous functions.

\begin{axdef}
	\CzeroRtup: \nat \fun \power(\Rinf \pfun \R)
\where
	\forall n: \nat @ \\
	\t1	\CzeroRtup(n) = \CzeroTT(\tsRtup(n), \Rtau)
\end{axdef}

\subsection{Real-Valued Functions That Are Continuous on a Subset of $n$-tuples: \zcmd{CzeroSubsetRtup}}

Let $U$ be a subset of $\R^n$.
A function $f \in U \fun \R$ is said to be \textit{continuous on} $U$
if it is continuous with respect to the topology induced on $U$.
Let $\CzeroSubsetRtup(U)$ denote the set of these continuous functions.

\begin{axdef}
	\CzeroSubsetRtup: \DeltaRinf \fun \power(\Rinf \pfun \R)
\where
	\forall U: \DeltaRinf @ \\
	\t1	\LET n == \dimRinf U @ \\
	\t2		\CzeroSubsetRtup(U) = \CzeroTT(\tsRtup(n) \inducedTopSp U, \Rtau)
\end{axdef}

\subsection{Real-Valued Functions That Are Continuous at an $n$-tuple: \zcmd{CzeroPointRtup}}

A partial function $f$ from $\R^n$ to $\R$ is said to be \textit{continuous at} $x \in \R^n$ if its domain contains a neighbourhood $U$ of $x$
such that its restriction to $U$ is continuous on $U$.
Let $\CzeroPointRtup(x)$ denote the set of such functions.

\begin{axdef}
	\CzeroPointRtup: \Rinf \fun \power(\Rinf \pfun \R)
\where
	\forall x: \Rinf @ \\
	\t1	\LET n == \# x @ \\
	\t2		\CzeroPointRtup(x) = \{~ f: \Rtup(n) \pfun \R | \exists U: \neighRinf(x) | U \subseteq \dom f @ U \dres f \in \CzeroSubsetRtup(U) ~\}
\end{axdef}

\subsection{$m$-tuple-Valued Functions That Are Continuous on the Set of All $n$-tuples: \zcmd{CzeroRtupRtup}}

A mapping $f$ from $\Rtup(n)$ to $\Rtup(m)$ is said to be continuous if it is continuous with respect to the usual topologies.
Let $\CzeroRtupRtup(n,m)$ denote the set of these continuous mappings.

\begin{axdef}
	\CzeroRtupRtup: \nat \cross \nat \fun \power(\Rinf \pfun \Rinf)
\where
	\forall n, m: \nat @ \\
	\t1	\CzeroRtupRtup(n,m) = \CzeroTT(\tsRtup(n), \tsRtup(m))
\end{axdef}

\begin{example}
The function $\idRtup(n)$ is continuous.

\begin{zed}
	\forall n: \nat @ \\
	\t1	\idRtup(n) \in \CzeroRtupRtup(n,n)
\end{zed}

\begin{theorem}
Linear functions are continuous.

\begin{zed}
	\forall n, m: \nat@ \\
	\t1	\linRtup(n, m) \subseteq \CzeroRtupRtup(n,m)
\end{zed}

\end{theorem}

\end{example}

\subsection{$m$-tuple-Valued Functions That Are Continuous on a Subset of $n$-tuples: \zcmd{CzeroSubsetRtupRtup}}

Let $U$ be any subset of $\Rtup(n)$.
Let $\CzeroSubsetRtupRtup(U,m)$ denote the set of continuous mappings from the topology induced by $\tsRtup(n)$ on $U$ to $\tsRtup(m)$.

\begin{axdef}
	\CzeroSubsetRtupRtup: \DeltaRinf \cross \nat \fun \power (\Rinf \pfun \Rinf)
\where
	\forall n, m: \nat @ \\
	\t1	\forall U: \DeltaRinf | \dimRinf(U) = n @ \\
	\t2		\CzeroSubsetRtupRtup(U, m) = \CzeroTT(\tsRtup(n) \inducedTopSp U, \tsRtup(m))
\end{axdef}

\begin{remark}

\begin{zed}
	\forall n, m: \nat @ \\
	\t1	\CzeroSubsetRtupRtup(\Rtup(n),m) = \CzeroRtupRtup(n,m)
\end{zed}

\end{remark}

\subsection{$m$-tuple-Valued Functions That Are Continuous at an $n$-tuple: 
$VectorConinuous$, \zcmd{CzeroPointRtupRtup}}

Let $x \in \Rtup(n)$ and let $f$ be a partial function from $\Rtup(n)$ to $\Rtup(m)$
whose domain includes some neighbourhood $U$ of $x$ such that $f$ restricted to $U$ is continuous.
In this case $f$ is said to be {\it continuous at $x$}.

\begin{schema}{VectorContinuous}
	n, m: \nat \\
	f: \Rinf \pfun \Rinf \\
	x: \Rinf
\where
	f \in \Rtup(n) \pfun \Rtup(m)
\also
	\exists U: \neighRinf(x) | \\
	\t1	U \subseteq \dom f @ \\
	\t2		U \dres f \in \CzeroSubsetRtupRtup(U,m)
\end{schema}

Let $\CzeroPointRtupRtup(x,m)$ denote the set of all partial functions $f$ from $\Rtup(n)$ to $\Rtup(m)$
that are continuous at $x$.

\begin{axdef}
	\CzeroPointRtupRtup: \Rinf \cross \nat \fun \power (\Rinf \pfun \Rinf)
\where
	\forall n, m: \nat @ \forall x: \Rtup(n)  @ \\
	\t1	\CzeroPointRtupRtup(x, m) = \\
	\t2		\{~ f: \Rtup(n) \pfun \Rtup(m) | VectorContinuous ~\}
\end{axdef}

\begin{example}
The function $\idRtup(n)$ is continuous at every point $x \in \Rtup(n)$.

\begin{zed}
	\forall n: \nat @ \forall x: \Rtup(n) @ \\
	\t1	\idRtup(n) \in \CzeroPointRtupRtup(x, n)
\end{zed}

\end{example}

\begin{theorem}
Linear functions are continuous everywhere.

\begin{zed}
	\forall n, m: \nat @ \\
	\t1	\forall x: \Rtup(n); L: \linRtup(n,m) @ \\
	\t2		L \in \CzeroPointRtupRtup(x, m)
\end{zed}

\end{theorem}


\section{Differentiability}

Let $x \in \R^n$ and let $f: \R^n \pfun \R^m$ be continuous at $x$.
Then $f$ is said to be {\it differentiable at $x$} if there exists a linear transformation $L: \R^n \fun \R^m$
such that $f(x + h) - f(x)$ is approximately linear in $h$ for very small $h$.
$$
f(x + h) - f(x) \approx  L(h) + O(h^2) \quad \text{when} \quad \norm{h} \approx 0
$$

This condition can be written as a limit.
$$
\lim_{h \to 0} \frac{\norm{f(x+h) - f(x) - L(h)}}{\norm{h}} = 0
$$

\subsection{The Difference Quotient: $DifferenceQuotient$ and $diffQuot$}

The limit exists when the following difference quotient function $q: \R^n \pfun \R$ is continuous at $0$.
$$
q(h) = 
\begin{cases}
	\frac{\norm{f(x+h) - f(x) - L(h)}}{\norm{h}}	&	\text{if } h \neq 0\\
	0								&	\text{otherwise}
\end{cases}
$$

Given a function $f$ that is continuous at $x$, and a linear transformation $L$,
we can define the difference quotient $q$.
Clearly $q$ is uniquely determined by $f$, $x$, and $L$.
Let $DifferenceQuotient$ denote this situation.

\begin{schema}{DifferenceQuotient}
	VectorContinuous \\
	L: \Rinf \pfun \Rinf \\
	q: \Rinf \pfun \R
\where
	L \in \linRtup(n, m)
\also
	\dom q = \{~ h: \Rtup(n) | x \addRinf h \in \dom f ~\}
\also
	\forall h: \dom q | h \neq \zeroRtup(n) @ \\
	\t1	q(h) = \normRinf(f(x \addRinf h) \subRinf f(x) \subRinf L(h)) \divR \normRinf(h)
\also
	q(\zeroRtup(n)) = \zeroR
\end{schema}

\begin{itemize}
\item $L$ is a linear transformation from $\Rtup(n)$ to $\Rtup(m)$.
\item The difference quotient $q$ is defined on a subset of $\Rtup(n)$ that contains $\zeroRtup(n)$.
\item $q(h)$ is defined as the quotient when $h$ is non-zero.
\item $q(0)$ is defined as zero.
\end{itemize}

Let $diffQuot(f,x,L)$ denote the difference quotient $q$.

\begin{zed}
diffQuot == \{~ DifferenceQuotient @ (f, x, L) \mapsto q ~\}
\end{zed}

\subsection{The Derivative of a Continuous $m$-tuple-Valued Function: $VectorDifferentiable$}

The continuous function $f$ is \textit{differentiable at} $x$ when there exists a linear transformation $L$ 
such that the difference quotient $q$ is continuous at $0$.
In this case $L$ is unique and is referred to as the \textit{derivative at} $x$.

\begin{schema}{VectorDifferentiable}
VectorContinuous \\
L: \Rinf \pfun \Rinf
\where
\LET q == diffQuot(f, x, L) @ \\
\t1	q \in \CzeroPointRtup(\zeroRtup(n))
\end{schema}

\begin{itemize}
\item The continuous function $f$ is differentiable at $x$ with derivative $L$ if the resulting
difference quotient $q$ is continuous at $\zeroRtup(n)$.
\end{itemize}

\begin{remark}
If $L$ exists then it is unique.
\end{remark}


Let $\smoothRnm(x,m)$ denote the set of all functions $f \in \Rtup(n) \pfun \Rtup(m)$ that are smooth at $x \in \Rtup(n)$.

\printbibliography

\end{document}