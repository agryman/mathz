\documentclass[11pt, oneside]{article}

\usepackage{../../shared/preamble}
\addbibresource{../../shared/references.bib}

\usepackage{../sets/sets}
\usepackage{../groups/groups}
\usepackage{../topological-spaces/topological-spaces}
\usepackage{../real-numbers/real-numbers}
\usepackage{vector-spaces}
\usepackage{amsmath}

\title{Vector Spaces}
\author{Arthur Ryman, {\tt arthur.ryman@gmail.com}}
\date{\today}

\begin{document}

\maketitle

\begin{abstract}
This article contains Z Notation type declarations for vector spaces and some related objects.
It has been type checked by \fuzz.
\end{abstract}

\section{Real Vector Spaces}

\subsection{\zcmd{addV}, \zcmd{zeroV}, \zcmd{negV}, and \zcmd{mulS}}

Let $v$ and $w$ denote vectors and let $x$ denote a real number.
Let $v \addV w$ denote vector addition,
let $\zeroV$ denote the zero vector,
let $\negV v$ denote the negative vector,
and let $x \mulS v$ denote scalar multiplication.

Let $RealVectorSpace$ denote the set of all real vector spaces.

\begin{schema}{RealVectorSpace}[\genT]
	vectors: \power \genT \\
	\_ \addV \_: \genT \cross \genT \pfun \genT \\
	\zeroV: \genT \\
	\negV: \genT \pfun \genT \\
	\_ \mulS \_: \R \cross \genT \pfun \genT
	\where
	\exists_1 A: AbelianGroup[\genT] @ \\
	\t1	A.elements = vectors \land \\
	\t1	A.(\_ \addG \_) = (\_ \addV \_) \land \\
	\t1	A.\zeroG = \zeroV \land \\
	\t1	A.\negG = \negV
	\also
	(\_ \mulS \_) \in \R \cross vectors \fun vectors
	\also
	\forall v: vectors @ \zeroR \mulS v = \zeroV
	\also
	\forall v: vectors @ \oneR \mulS v = v
	\also
	\forall x, y: \R; v: vectors @ (x \mulR y) \mulS v = x \mulS (y \mulS v)
	\also
	\forall x, y: \R; v: vectors @ (x \addR y) \mulS v = x \mulS v \addV y \mulS v
	\also
	\forall x: \R; v, w: vectors @ x \mulS (v \addV w) = x \mulS v \addV x \mulS w
\end{schema}

\begin{itemize}
	\item Vectors form an Abelian group under addition.
	\item Multiplying a vector by a scalar gives a vector.
	\item Multiplying by $\zeroR$ gives the zero vector.
	\item Multiplying by $\oneR$ gives the same vector.
	\item Scalar multiplication is associative.
	\item Scalar addition distributes over scalar multiplication.
	\item Vector addition distributes over scalar multiplication.
\end{itemize}

\subsubsection{$\RealVectorSpaceAbelianGroup$}

We can forget scalar multiplication on vectors and obtain an Abelian group.
Let $\RealVectorSpaceAbelianGroup$ denote the function that maps a real vector space to
its underlying Abelian group.

\begin{gendef}[\genT]
	\RealVectorSpaceAbelianGroup: RealVectorSpace[\genT] \fun AbelianGroup[\genT]
	\where
	\RealVectorSpaceAbelianGroup = \\
	\t1	(\lambda V: RealVectorSpace[\genT] @ \\
	\t2		(\mu A: AbelianGroup[\genT] | \\
	\t3			A.elements = V.vectors \land \\
	\t3			A.(\_ \addG \_) = V.(\_ \addV \_) \land \\
	\t3			A.\zeroG = V.\zeroV \land \\
	\t3			A.\negG = V.\negV))
\end{gendef}

\subsection{Linear Transformations}

Let $X$ and $Y$ be vector spaces.
A {\em linear transformation} from $X$ to $Y$ is a homomorphism of the underlying Abelian groups
that preserves scalar multiplication.

\section{Introduction}

Real vector spaces are multidimensional generalizations of real numbers.
They are the objects studied in linear algebra and are foundational to differential geometry.

\section{Real $n$-tuples}

\subsection{\zcmd{Rinf}}

Let $n$ be a natural number.
A finite sequence of $n$ real numbers is called a {\it real $n$-tuple}.
Let $\Rinf$ denote the set of all real $n$-tuples for any $n$.

\begin{zed}
	\Rinf == \seq \R
\end{zed}

\subsection{\zcmd{Rtuples}}

Let $\Rtuples(n)$ denote $\R^n$, the set of all $n$-tuples for some given $n$.
\begin{axdef}
	\Rtuples: \nat \fun \power \Rinf
\where
	\forall n: \nat @ \\
	\t1	\Rtuples(n) = \{~ v: \Rinf | \# v = n ~\}
\end{axdef}

\begin{remark}

\begin{zed}
	\Rinf = \bigcup \{~ n: \nat @ \Rtuples(n) ~\}
\end{zed}

\end{remark}

\subsection{\zcmd{DeltaR}}

Let $\DeltaR$ denote the family of subsets of $\Rinf$ such that all tuples in each subset have the same number of components.
Such as subset is said to be {\it well-dimensioned}.

\begin{axdef}
	\DeltaR: \family~\Rinf
\where
	\DeltaR = \bigcup \{~ n: \nat @ \power(\Rtuples(n)) ~\}
\end{axdef}

\begin{example}
The subset $\Rtuples(n)$ is well-dimensioned.

\begin{zed}
	\forall n: \nat @ \\
	\t1	\Rtuples(n) \in \DeltaR
\end{zed}

\end{example}

\subsection{\zcmd{dimR}}

Let $\dimR(U)$ denote the number of components of the tuples in $U \in \DeltaR$.

\begin{axdef}
	\dimR: \DeltaR \fun \nat
\where
	\forall n: \nat @ \forall U: \power(\Rtuples(n)) @ \\
	\t1	\dimR(U) = n
\end{axdef}

\begin{example}
The dimension of $\Rtuples(n)$ is $n$.

\begin{zed}
	\forall n: \nat @ \\
	\t1	\dimR(\Rtuples(n)) = n
\end{zed}

\end{example}

\subsection{\zcmd{zeroRn}}

Let $\zeroRn(n)$ denote the $n$-tuple consisting of all zeroes.

\begin{axdef}
	\zeroRn: \nat \fun \Rinf
\where
	\zeroRn(0) = \langle \rangle
\also
	\forall n: \nat_1 @ \\
	\t1	\zeroRn(n) = (\lambda i: 1 \upto n @ \zeroR)
\end{axdef}

\begin{remark}
The tuple $\zeroRn(n)$ is in $\Rtuples(n)$.

\begin{zed}
	\forall n: \nat @
		\zeroRn(n) \in \Rtuples(n)
\end{zed}

\end{remark}

\subsection{\zcmd{piR}}

The real numbers that comprise an $n$-tuple are called its components.
The real number $v(i)$ is the $i$-th component of the $n$-tuple $v$ where
$1 \le i \le n$.
Let $\piR(i)$ be the projection function that maps an $n$-tuple $v$ to its $i$-th component $v(i)$.

\begin{axdef}
	\piR: \nat_1 \fun \Rinf \pfun \R
\where
	\forall i: \nat_1 @ \\
	\t1	\piR(i) = (\lambda v: \Rinf | i \in \dom v @ v(i))
\end{axdef}

\begin{remark}
Every component of $\zeroRn(n)$ is $\zeroR$.

\begin{zed}
	\forall n: \nat @ \forall i: 1 \upto n @ \\
	\t1	\piR(i)(\zeroRn(n)) = \zeroR
\end{zed}

\end{remark}

\section{Scalar Multiplication}

\subsection{\zcmd{smulR}}

Let $v$ be an $n$-tuple and let $c$ be a real number.
Scalar multiplication of $v$ by $c$ is the $n$-tuple $c \smulR v$ defined by component-wise multiplication.

\begin{axdef}
	\_ \smulR \_ : \R \cross \Rinf \fun \Rinf 
\where
	\forall c: \R @ \\
	\t1	c \smulR \langle \rangle = \langle \rangle
\also
	\forall c: \R; n: \nat_1 @ \\
	\t1	\forall v: \Rtuples(n); i: 1 \upto n @ \\
	\t2		(c \smulR v)(i) = c \mulR v(i)
\end{axdef}

\begin{remark}
Scalar multiplication is associative in the sense that $(a \mulR b) \smulR v = a \smulR (b \smulR v)$

\begin{zed}
	\forall a, b: \R; v: \Rinf @ \\
	\t1	(a \mulR b) \smulR v = a \smulR (b \smulR v)
\end{zed}

\end{remark}

\section{Vector Addition and Subtraction}

\subsection{\zcmd{Rdelta}}

Let $\Rdelta$ denote the set of all pairs of tuples that have the same number of components.

\begin{axdef}
	\Rdelta: \Rinf \rel \Rinf
\where
	\Rdelta = \{~ v, w: \Rinf | \# v = \# w ~\}
\end{axdef}

\subsection{\zcmd{vaddR}}

Let $v$ and $w$ be $n$-tuples.
Vector addition of $v$ and $w$ is the $n$-tuple $v \vaddR w$ defined by component-wise addition.

\begin{axdef}
	\_ \vaddR \_: \Rdelta \fun \Rinf
\where
	\langle \rangle \vaddR \langle \rangle = \langle \rangle
\also
	\forall n: \nat_1 @ \\
	\t1	\forall v, w: \Rtuples(n); i: 1 \upto n @ \\
	\t2		(v \vaddR w)(i) = v(i) \addR w(i)
\end{axdef}

\subsection{\zcmd{vsubR}}

Vector subtraction is defined similarly.

\begin{axdef}
	\_ \vsubR \_: \Rdelta \fun \Rinf
\where
	\langle \rangle \vsubR \langle \rangle = \langle \rangle
\also
	\forall n: \nat_1 @ \\
	\t1	\forall v, w: \Rtuples(n); i: 1 \upto n @ \\
	\t2		(v \vsubR w)(i) = v(i) \subR w(i)
\end{axdef}

Each $\Rtuples(n)$ is a real vector space under the operations of scalar multiplication and vector addition
defined above. 

\section{Vector Spaces}

The sets $\R^n$ with the operations of scalar multiplication and vector addition form vector spaces.
In general, a vector space is a set of vectors endowed with scalar multiplication and vector addition operations that
follow rules analogous to those for $\R^n$.

\subsection{$VectorSpace$}

Let $V$ be a set and let $VectorSpace[V]$ denote the set of all vector spaces whose vectors are $V$.

\begin{schema}{VectorSpace}[V]
	\zeroV: V \\
	\_ \addV \_: V \cross V \fun V \\
	\_ \smulV \_: \R \cross V \fun V
\where
	\forall v: V @ \\
	\t1	\zeroV \addV v = v = v \addV \zeroV
\also
	\forall v, w: V @ \\
	\t1	v \addV w = w \addV v
\also
	\forall u, v, w: V @ \\
	\t1	u \addV (v \addV w) = (u \addV v) \addV w
\also
	\forall v: V @ \\
	\t1	\zeroR \smulV v = \zeroV
\also
	\forall v: V @ \\
	\t1	\oneR \smulV v = v
\also
	\forall a, b: \R; v: V @ \\
	\t1	(a \addR b) \smulV v = (a \smulV v) \addV (b \smulV v)
\also
	\forall a, b: \R; v: V @ \\
	\t1	(a \mulR b) \smulV v = a \smulV (b \smulV v)
\also
	\forall a: \R; v, w: V @ \\
	\t1	a \smulV (v \addV w) = (a \smulV v) \addV (a \smulV w)
\end{schema}
\begin{itemize}
	\item the zero vector $\zeroV$ is the identity element for vector addition
	\item vector addition is commutative
	\item vector addition is associative
	\item scalar multiplication by $\zeroR$ gives the zero vector
	\item scalar multiplication by $\oneR$ leaves any vector unchanged
	\item real addition distributes over scalar multiplication
	\item real multiplication associates over scalar multiplication
	\item scalar multiplication distributes over vector addition
\end{itemize}

\subsection{$vectorSpace$}

Let $vectorSpace[V]$ the set of all triples consisting of a zero vector, a vector addition operation, and a scalar multiplication operation
that define a vector space whose vectors are $V$,

\section{Linear Transformations}

\subsection{Linear}

Let $n$ and $m$ be natural numbers.
A mapping $L$ from $\R^n$ to $\R^m$ is said to be a {\it linear transformation} if it preserves scalar multiplication and vector addition.
\begin{schema}{Linear}
	n, m: \nat \\
	L: \Rinf \pfun \Rinf
\where
	L \in \Rtuples(n) \fun \Rtuples(m)
\also
	\forall c: \R; v: \Rtuples(n) @ \\
	\t1	L(c \smulR v) = c \smulR L(v)
\also
	\forall v, w: \Rtuples(n) @ \\
	\t1	L(v \vaddR w) = L(v) \vaddR L(w)
\end{schema}

\subsection{\zcmd{linR}}

Define $\linR(n,m)$ to be the set of all linear transformations from $\R^n$ to $\R^m$.
\begin{axdef}
	\linR: \nat \cross \nat \fun \power(\Rinf \pfun \Rinf)
\where
	\forall n,m: \nat @ \\
	\t1	\linR(n,m) = \{~ L: \Rinf \pfun \Rinf | Linear ~\}
\end{axdef}

\subsection{\zcmd{In}}

Let $\In(n)$ denote the identity function on $\Rtuples(n)$.

\begin{axdef}
	\In: \nat \fun \Rinf \pfun \Rinf
\where
	\forall n: \nat @ \\
	\t1	\In(n) = \id(\Rtuples(n))
\end{axdef}

\begin{remark}
The function $\In(n)$ is a linear transformation.

\begin{zed}
	\forall n: \nat @ \\
	\t1	\In(n) \in \linR(n, n)
\end{zed}

\end{remark}

\section{The Dot Product}

\subsection{\zcmd{dotR}}

The {\it inner} or {\it dot} product of $n$-tuples $v$ and $w$ is the real number $v \dotR w$ defined by the sum of the component-wise products.

\begin{axdef}
	\_ \dotR \_ : \Rdelta \fun \R
\where
	\langle \rangle \dotR \langle \rangle = \zeroR
\also
	\forall x, y: \R; v, w: \Rinf | \# v = \# w @ \\
	\t1	(\langle x \rangle \cat v) \dotR (\langle y \rangle \cat w) = x \mulR y \addR v \dotR w
\end{axdef}

Each $\Rtuples(n)$ is a real inner product space under the operation of dot product defined above.

\section{The Norm}

\subsection{\zcmd{normR}}

The norm $\norm{v}$ of the $n$-tuple $v$ is the positive square root of its dot product with itself.
$$
	\norm{v} = \sqrt{v \dotR v}
$$

Define $\normR(v)$ to be $\norm{v}$.
\begin{axdef}
	\normR: \Rinf \fun \R
\where
	\forall v: \Rinf @ \\
	\t1	\normR(v) = \sqrtR(v \dotR v)
\end{axdef}

The concepts of continuity, limits, and differentiability extend to functions between normed vector spaces such as $\R^n$.

\subsection{\zcmd{ballRn}}

Let $\ballRn(v,r)$ denote the open ball in $\Rtuples(n)$ of radius $r  \in \R$ centred at $v \in \Rtuples(n)$.

\begin{axdef}
	\ballRn: \Rinf \cross \R \fun \power \Rinf
\where
	\forall v: \Rinf; r: \R @ \LET n == \# v @ \\
	\t1	\ballRn(v, r) = \{~ w: \Rtuples(n) | \normR(v \vsubR w) \ltR r ~\}
\end{axdef}

\subsection{\zcmd{ballsRn}}

Let $\ballsRn(n)$ denote the family of all open balls in $\Rtuples(n)$.

\begin{axdef}
	\ballsRn: \nat \fun \family~\Rinf
\where
	\forall n: \nat @ \\
	\t1	\ballsRn(n) =  \{~ v: \Rtuples(n); r: \R @ \ballRn(v,r) ~\}
\end{axdef}

\subsection{\zcmd{tauRn}}

The usual topology on $\Rtuples(n)$ is the topology generated by the open balls in $\Rtuples(n)$.
Let $\tauRn(n)$ denote the usual topology on $\Rtuples(n)$.

\begin{axdef}
	\tauRn: \nat \fun \family~\Rinf
\where
	\forall n: \nat @ \\
	\t1	\tauRn(n) = topGen[\Rtuples(n)] (\ballsRn(n))
\end{axdef}

\begin{remark}

If $n \in \nat$ then $\tauRn(n)$ is a topology on $\Rtuples(n)$.

\begin{zed}
	\forall n: \nat @ \tauRn(n) \in top[\Rtuples(n)]
\end{zed}
\end{remark}

\subsection{\zcmd{neighRn}}

Let $x \in \Rtuples(n)$. 
Let $\neighRn(x)$ denote the set of all open sets $U$ in the usual topology $\tauRn(n)$ that contain $x$.
Such a set $U$ is called a neighbourhood of $x$.

\begin{axdef}
	\neighRn: \Rinf \fun \family~\Rinf
\where
	\forall x: \Rinf @ \LET n == \# x @\\
	\t1	\neighRn(x) = \{~ U: \tauRn(n) | x \in U ~\}
\end{axdef}

\begin{remark}

\begin{zed}
	\forall v: \Rinf @ \LET n == \# v @ \neighRn(v) \in \family(\Rtuples(n))
\end{zed}

\end{remark}

\subsection{\zcmd{RtauN}}

Let $\RtauN(n)$ denote the topological space defined by the usual topology on $\Rtuples(n)$.

\begin{axdef}
	\RtauN: \nat \fun topSpaces[\Rinf]
\where
	\forall n: \nat @ \\
	\t1	\RtauN(n) = (\Rtuples(n), \tauRn(n))
\end{axdef}

\section{Continuity}

\subsection{\zcmd{CzeroN}}

A function $f$ from $\R^n$ to $\R$ is said to be continuous if it is continuous with respect to the usual topologies.
Let $\CzeroN(n)$ denote the set of these continuous mappings.

\begin{axdef}
	\CzeroN: \nat \fun \power(\Rinf \pfun \R)
\where
	\forall n: \nat @ \\
	\t1	\CzeroN(n) = \CzeroTT(\RtauN(n), \Rtau)
\end{axdef}

\subsection{\zcmd{CzeroPRn}}

Let $U$ be a subset of $\R^n$.
A function $f \in U \fun \R$ is said to be continuous if it is continuous with respect to the topology induced on $U$.
Let $\CzeroPRn(U)$ denote the set of these continuous functions.

\begin{axdef}
	\CzeroPRn: \DeltaR \fun \power(\Rinf \pfun \R)
\where
	\forall U: \DeltaR @ \\
	\t1	\LET n == \dimR U @ \\
	\t2		\CzeroPRn(U) = \CzeroTT(\RtauN(n) \inducedTopSp U, \Rtau)
\end{axdef}

\subsection{\zcmd{CzeroRn}}

A partial function $f$ from $\R^n$ to $\R$ is said to be continuous at $x \in \R^n$ if its domain contains a neighbourhood $U$ of $x$
such that its restriction to $U$ is continuous on $U$.
Let $\CzeroRn(x)$ denote the set of such functions.

\begin{axdef}
	\CzeroRn: \Rinf \fun \power(\Rinf \pfun \R)
\where
	\forall x: \Rinf @ \\
	\t1	\LET n == \# x @ \\
	\t2		\CzeroRn(x) = \{~ f: \Rtuples(n) \pfun \R | \exists U: \neighRn(x) | U \subseteq \dom f @ U \dres f \in \CzeroPRn(U) ~\}
\end{axdef}

\subsection{\zcmd{CzeroNN}}

A mapping $f$ from $\Rtuples(n)$ to $\Rtuples(m)$ is said to be continuous if it is continuous with respect to the usual topologies.
Let $\CzeroNN(n,m)$ denote the set of these continuous mappings.

\begin{axdef}
	\CzeroNN: \nat \cross \nat \fun \power(\Rinf \pfun \Rinf)
\where
	\forall n, m: \nat @ \\
	\t1	\CzeroNN(n,m) = \CzeroTT(\RtauN(n), \RtauN(m))
\end{axdef}

\begin{example}
The function $\In(n)$ is continuous.

\begin{zed}
	\forall n: \nat @ \\
	\t1	\In(n) \in \CzeroNN(n,n)
\end{zed}

\begin{theorem}
Linear functions are continuous.

\begin{zed}
	\forall n, m: \nat@ \\
	\t1	\linR(n, m) \subseteq \CzeroNN(n,m)
\end{zed}

\end{theorem}

\end{example}

\subsection{\zcmd{CzeroPRnN}}

Let $U$ be any subset of $\Rtuples(n)$.
Let $\CzeroPRnN(U,m)$ denote the set of continuous mappings from the topology induced by $\RtauN(n)$ on $U$ to $\RtauN(m)$.

\begin{axdef}
	\CzeroPRnN: \DeltaR \cross \nat \fun \power (\Rinf \pfun \Rinf)
\where
	\forall n, m: \nat @ \\
	\t1	\forall U: \DeltaR | \dimR(U) = n @ \\
	\t2		\CzeroPRnN(U, m) = \CzeroTT(\RtauN(n) \inducedTopSp U, \RtauN(m))
\end{axdef}

\begin{remark}

\begin{zed}
	\forall n, m: \nat @ \\
	\t1	\CzeroPRnN(\Rtuples(n),m) = \CzeroNN(n,m)
\end{zed}

\end{remark}

\subsection{\zcmd{CzeroRnN}}

Let $x \in \Rtuples(n)$ and let $f$ be a partial function from $\Rtuples(n)$ to $\Rtuples(m)$
whose domain includes some neighbourhood $U$ of $x$ such that $f$ restricted to $U$ is continuous.
In this case $f$ is said to be {\it continuous at $x$}.

\begin{schema}{VectorContinuous}
	n, m: \nat \\
	f: \Rinf \pfun \Rinf \\
	x: \Rinf
\where
	f \in \Rtuples(n) \pfun \Rtuples(m)
\also
	\exists U: \neighRn(x) | \\
	\t1	U \subseteq \dom f @ \\
	\t2		U \dres f \in \CzeroPRnN(U,m)
\end{schema}

Let $\CzeroRnN(x,m)$ denote the set of all partial functions $f$ from $\Rtuples(n)$ to $\Rtuples(m)$
that are continuous at $x$.

\begin{axdef}
	\CzeroRnN: \Rinf \cross \nat \fun \power (\Rinf \pfun \Rinf)
\where
	\forall n, m: \nat @ \forall x: \Rtuples(n)  @ \\
	\t1	\CzeroRnN(x, m) = \\
	\t2		\{~ f: \Rtuples(n) \pfun \Rtuples(m) | VectorContinuous ~\}
\end{axdef}

\begin{example}
The function $\In(n)$ is continuous at every point $x \in \Rtuples(n)$.

\begin{zed}
	\forall n: \nat @ \forall x: \Rtuples(n) @ \\
	\t1	\In(n) \in \CzeroRnN(x, n)
\end{zed}

\end{example}

\begin{theorem}
Linear functions are continuous everywhere.

\begin{zed}
	\forall n, m: \nat @ \\
	\t1	\forall x: \Rtuples(n); L: \linR(n,m) @ \\
	\t2		L \in \CzeroRnN(x, m)
\end{zed}

\end{theorem}


\section{Differentiability}

Let $x \in \R^n$ and let $f: \R^n \pfun \R^m$ be continuous at $x$.
Then $f$ is said to be {\it differentiable at $x$} if there exists a linear transformation $L: \R^n \fun \R^m$
such that $f(x + h) - f(x)$ is approximately linear in $h$ for very small $h$.
$$
f(x + h) - f(x) \approx  L(h) + O(h^2) \quad \text{when} \quad \norm{h} \approx 0
$$

This condition can be written as a limit.
$$
\lim_{h \to 0} \frac{\norm{f(x+h) - f(x) - L(h)}}{\norm{h}} = 0
$$

\subsection{$diffQuot$}

The limit exists when the following difference quotient function $q: \R^n \pfun \R$ is continuous at $0$.
$$
q(h) = 
\begin{cases}
	\frac{\norm{f(x+h) - f(x) - L(h)}}{\norm{h}}	&	\text{if } h \neq 0\\
	0								&	\text{otherwise}
\end{cases}
$$

\begin{schema}{DifferenceQuotient}
	VectorContinuous \\
	L: \Rinf \pfun \Rinf \\
	q: \Rinf \pfun \R
\where
	L \in \linR(n, m)
\also
	\dom q = \{~ h: \Rtuples(n) | x \vaddR h \in \dom f ~\}
\also
	\forall h: \dom q | h \neq \zeroRn(n) @ \\
	\t1	q(h) = \normR(f(x \vaddR h) \vsubR f(x) \vsubR L(h)) \divR \normR(h)
\also
	q(\zeroRn(n)) = \zeroR
\end{schema}

The function $f$ is differentiable at $x$ when there exists a linear transformation $L$ such that the difference quotient $q$ is
continuous at $0$.

\begin{schema}{VectorDifferentiable}
	DifferenceQuotient
\where
	q \in \CzeroRn(\zeroRn(n))
\end{schema}

Clearly $q$ is uniquely determined by $f$, $x$, and $L$.
Let $diffQuot(f,x,L)$ denote the difference quotient.

\begin{axdef}
	diffQuot: (\Rinf \pfun \Rinf) \cross \Rinf \cross (\Rinf \pfun \Rinf) \pfun (\Rinf \pfun \R)
\where
	diffQuot = \{~ VectorDifferentiable @ (f, x, L) \mapsto q ~\}
\end{axdef}




Let $\smoothRnm(x,m)$ denote the set of all functions $f \in \Rtuples(n) \pfun \Rtuples(m)$ that are smooth at $x \in \Rtuples(n)$.

\printbibliography

\end{document}