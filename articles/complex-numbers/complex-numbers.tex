\documentclass{amsart}

\usepackage{mathz-formal-specifications}
\usepackage{mathz-sets}
\usepackage{mathz-topological-spaces}
\usepackage{mathz-groups}
\usepackage{mathz-real-numbers}
\usepackage{mathz-complex-numbers}
\usepackage{mathz-preamble}

\addbibresource{../../shared/mathz-references.bib}

\begin{document}

\title{Complex Numbers}
\author{Arthur Ryman}
\email[Arthur Ryman]{arthur.ryman@gmail.com}
\date{\today}

\begin{abstract}
This article contains \ZN\ definitions for the complex numbers, $\C$, and some related objects.
It has been type checked by \fuzz.
\end{abstract}

\maketitle

\tableofcontents

\section{Introduction}

The complex numbers, $\C$, are foundational to many mathematical objects such as vector spaces and manifolds,
but are not built into Z Notation.
This article provides type declarations for $\C$ and related objects so that they can be used and type checked in formal Z specifications.

No attempt has been made to provide complete, axiomatic definitions of all these objects since that would only be of use for proof checking.
Although proof checking is highly desirable, it is beyond the scope of this article.
The type declarations given here are intended to provide a basis for future axiomatization.

\subsection{The Set of Complex Numbers}

Z Notation does not predefine the set of complex numbers, so we define
them and operations on them here.

Although complex number operations are displayed using the same symbols as the 
analogous real number operations,
they are distinct mathematical objects.
This distinction is apparent to the \fuzz\ type-checker and should not cause confusion to the human reader
because the underlying types of the objects will, as a rule, be clear from the context.
Visually distinct symbols will be used in cases where confusion is possible.

\subsubsection{$COMPLEX$}

A complex number can be thought of a pair of real numbers.
However, it is not correct to view any pair of real numbers as a complex number.
Therefore to denote a complex number we map a pair of real numbers
into the free type $COMPLEX$.

\begin{zed}
	COMPLEX ::= complex \ldata \Rtwo \rdata
\end{zed}

Here the function $complex$ is a constructor that constructs a
complex number from a pair of real numbers.

\subsubsection{\zcmd{C}}

We introduce the usual notation $\C = COMPLEX$ for the set of complex numbers.

\begin{zed}
\C == COMPLEX
\end{zed}

\subsection{Real and Imaginary Parts}

\subsubsection{$Complex$}

Given real numbers $x$ and $y$, we can construct the complex number
$z = complex(x,y)$.
The real numbers $x$ and $y$ are referred to as the \textit{real} and \textit{imaginary} parts of $z$.
It's useful to introduce the schema $Complex$ that relates the complex number $z$ to its
real and imaginary parts $x$ and $y$.

\begin{schema}{Complex}
	z : \C \\
	x, y : \R
\where
	z = complex(x, y)
\end{schema}

\subsubsection{$real\_of\_complex$}

Let $real\_of\_complex(z)$ denote the real part $x$ of $z$.

\begin{axdef}
	real\_of\_complex: \C \fun \R
\where
	real\_of\_complex = \{~ Complex @ z \mapsto x ~\}
\end{axdef}

\subsubsection{\zcmd{realC}}

We introduce the usual notation $x = \realC(z)$ for the real part of $z$.

\begin{zed}
	\realC == real\_of\_complex
\end{zed}

\subsubsection{$imag\_of\_complex$}

Let $imag\_of\_complex(z)$ denote the imaginary part $y$ of $z$.

\begin{axdef}
	imag\_of\_complex: \C \fun \R
\where
	imag\_of\_complex = \{~ Complex @ z \mapsto y ~\}
\end{axdef}

\subsubsection{\zcmd{imagC}}

We introduce the usual notation $y = \imagC(z)$ for the imaginary part of $z$.

\begin{zed}
	\imagC == imag\_of\_complex
\end{zed}

\subsection{Real Numbers as Complex Numbers}

The complex numbers contain a natural copy of the real numbers,
namely the set of complex numbers with vanishing imaginary part.

\subsubsection{$real\_as\_complex$}

Let the function $real\_as\_complex(x) = z$ map the real number $x$
to its corresponding complex number $z$ which has $x$ as its real part and $0$ as its imaginary part.

\begin{axdef}
	real\_as\_complex: \R \fun \C
\where
	real\_as\_complex = (\lambda x: \R @ complex(x, \zeroR))
\end{axdef}

\subsubsection{\zcmd{asRC}}

We introduce the notation $\asRC x = real\_as\_complex(x)$.

\begin{zed}
	\asRC == real\_as\_complex
\end{zed}

\subsection{Conjugation}

\subsubsection{$ComplexConjugate$}

Let $z' = z^*$ be the \textit{complex conjugate} of the complex number $z$.
Let the schema $ComplexConjugate$ denote this situation.

\begin{schema}{ComplexConjugate}
	Complex \\
	Complex'
\where
	x' = x \\
	y' = \negR y
\end{schema}
\begin{itemize}
\item The complex conjugate $z'$ of a complex number $z$ has the same real part $x$ 
and the negative imaginary part $-y$.
\end{itemize}

\subsubsection{$complex\_conjugate$}

Let the function $complex\_conjugate(z) = z^*$ map the complex number $z$ to its complex conjugate $z^*$.

\begin{axdef}
	complex\_conjugate: \C \fun \C
\where
	complex\_conjugate = \{~ ComplexConjugate @ z \mapsto z' ~\}
\end{axdef}

\subsubsection{\zcmd{conjC}}

We introduce the usual postfix operator notation $z\conjC = complex\_conjugate(z)$.

\begin{zed}
	(\_ \conjC) == complex\_conjugate
\end{zed}

\section{Some Important Complex Numbers}

We next define some important complex numbers.

\subsection{Zero}

\subsubsection{$zero\_complex$}

Let $zero\_complex$ denote the \textit{zero} of $\C$.

\begin{axdef}
	zero\_complex: \C
\where
	zero\_complex = \asRC \zeroR
\end{axdef}

\subsubsection{\zcmd{zeroC}}

We introduce the usual notation $\zeroC \in \C$ for the zero of $\C$.

\begin{zed}
	\zeroC == zero\_complex
\end{zed}

\subsection{One}

\subsubsection{$one\_complex$}

Let $one\_complex$ denote the multiplicative unit in $\C$.

\begin{axdef}
	one\_complex: \C
\where
	one\_complex = \asRC \oneR
\end{axdef}

\subsubsection{\zcmd{oneC}}

We introduce the usual notation $\oneC \in \C$ for the unit of $\C$.

\begin{zed}
	\oneC == one\_complex
\end{zed}

\subsection{The Square Root of $-1$}

\subsubsection{$i\_complex$}

Let $i\_complex$ denote the usual square root of $-1$ in $\C$.

\begin{axdef}
	i\_complex: \C
\where
	i\_complex = complex(\zeroR, \oneR)
\end{axdef}

\subsubsection{\zcmd{iC}}

We introduce the usual notation $\iC = i\_complex$.

\begin{zed}
	\iC == i\_complex
\end{zed}

\section{Arithmetic}

\subsection{Addition}

\subsubsection{$AddComplex$}

We can \textit{add} the complex numbers $z_1$ and $z_2$ to give their \textit{sum} $z' = z_1 + z_2$.
Let the schema $AddComplex$ denote this situation.

\begin{schema}{AddComplex}
	Complex_1 \\
	Complex_2 \\
	Complex'
\where
	x' = x_1 \addR x_2 \\
	y' = y_1 \addR y_2
\end{schema}

\subsubsection{$add\_complex$}

Let $add\_complex(z_1, z_2)$ denote the result of adding $z_1$ and $z_2$.

\begin{axdef}
	add\_complex: \C \cross \C \fun \C
\where
	add\_complex = \{~ AddComplex @ (z_1, z_2) \mapsto z' ~\}
\end{axdef}

\subsubsection{\zcmd{addC}}

We introduce the usual notation $z' = z_1 \addC z_2$ for addition in $\C$.

\begin{zed}
	(\_ \addC \_) == add\_complex
\end{zed}

\subsection{Negation}

\subsubsection{$NegComplex$}

We can \textit{negate} the complex number $z$ to give its \textit{negative} $z' = -z$.
Let the schema $NegComplex$ denote this situation.

\begin{schema}{NegComplex}
	Complex \\
	Complex'
\where
	x' = \negR x \\
	y' = \negR y
\end{schema}

\subsubsection{$neg\_complex(z)$}

Let $neg\_complex(z)$ denote the negative of $z$.

\begin{axdef}
	neg\_complex: \C \fun \C
\where
	neg\_complex = \{~ NegComplex @ z \mapsto z' ~\}
\end{axdef}

\subsubsection{\zcmd{negC}}

We introduce the usual notation $z' = \negC z$ for the negative of $z$.
\begin{zed}
	\negC == neg\_complex
\end{zed}

\subsection{Subtraction}

\subsubsection{$SubComplex$}

We can \textit{subtract} the complex number $z_2$ from $z_1$
to give their \textit{difference} $z' = z_1 - z_2$.
Let the schema $SubComplex$ denote this situation.

\begin{schema}{SubComplex}
	Complex_1 \\
	Complex_2 \\
	Complex'
\where
	x' = x_1 \subR x_2 \\
	y' = y_1 \subR y_2
\end{schema}

\subsubsection{$sub\_complex$}

Let $sub\_complex(z_1, z_2)$ denote the difference $z_1 - z_2$.

\begin{axdef}
	sub\_complex: \C \cross \C \fun \C
\where
	sub\_complex = \{~ SubComplex @ (z_1, z_2) \mapsto z' ~\}
\end{axdef}

\subsubsection{\zcmd{subC}}

We introduce the usual notation $z' = z_1 \subC z_2$ for subtraction in $\C$.

\begin{zed}
	(\_ \subC \_) == sub\_complex
\end{zed}

\subsection{Multiplication}

\subsubsection{$MulComplex$}

We can \textit{multiply} the complex numbers $z_1$ times $z_2$ to give their \textit{product} $z' = z_1 z_2$.
Let the schema $MulComplex$ denote this situation.

\begin{schema}{MulComplex}
	Complex_1 \\
	Complex_2 \\
	Complex'
\where
	x' = x_1 \mulR x_2 \subR y_1 \mulR y_2 \\
	y' = x_1 \mulR y_2 \addR y_1 \mulR x_2
\end{schema}

\subsubsection{$mul\_complex$}

Let $mul\_complex(z_1, z_2)$ denote the product $z_1 z_2$.

\begin{axdef}
	mul\_complex: \C \cross \C \fun \C
\where
	mul\_complex = \{~ MulComplex @ (z_1, z_2) \mapsto z' ~\}
\end{axdef}

\subsubsection{\zcmd{mulC}}

We introduce the usual notation $z' = z_1 \mulC z_2$ for multiplication in $\C$.

\begin{zed}
	(\_ \mulC \_) == mul\_complex
\end{zed}

\subsubsection{$mul\_real\_complex$}

Let the function $mul\_real\_complex(x, z) = xz$ denote the product of the real number $x$ and
the complex number $z$.

\begin{axdef}
	mul\_real\_complex: \R \cross \C \fun \C
\where
	mul\_real\_complex = (\lambda x: \R; z: \C @ (\asRC x) \mulC z)
\end{axdef}

\subsubsection{\zcmd{mulRC}}

We introduce the notation $x \mulRC z = mul\_real\_complex(x, z)$.

\begin{zed}
	(\_ \mulRC \_) == mul\_real\_complex
\end{zed}

\subsubsection{$mul\_complex\_real$}

Similarly, let the function $mul\_complex\_real(z, x) = zx$ denote the product of the complex number $z$ and
the real number $x$.

\begin{axdef}
	mul\_complex\_real: \C \cross \R \fun \C
\where
	mul\_complex\_real = (\lambda z: \C; x: \R @ z \mulC (\asRC x))
\end{axdef}

\subsubsection{\zcmd{mulCR}}

We introduce the notation $z \mulCR z = mul\_complex\_real(z, x)$.

\begin{zed}
	(\_ \mulCR \_) == mul\_complex\_real
\end{zed}

\subsection{Nonzero Complex Numbers}

\subsubsection{$nonzero\_complex$}

Let $nonzero\_complex$ denote the set of nonzero complex numbers.

\begin{axdef}
	nonzero\_complex: \power \C
\where
	nonzero\_complex = \C \setminus \{ \zeroC \}
\end{axdef}

\subsubsection{\zcmd{Cnz}}

We introduce the usual notation $\Cnz$ to denote the set of nonzero complex numbers,
also referred to as the \textit{punctured complex number plane}.

\begin{zed}
	\Cnz == nonzero\_complex
\end{zed}

\subsection{Nonzero Multiplication}

\subsubsection{$MulNonzeroComplex$}

We can restrict multiplication in $\C$ to $\Cnz$.
Let the schema $MulNonzeroComplex$ denote this situation.

\begin{schema}{MulNonzeroComplex}
	MulComplex \\
\where
	z_1 \in \Cnz \\
	z_2 \in \Cnz
\end{schema}

\subsubsection{$mul\_nonzero\_complex$}

Let $mul\_nonzero\_complex(z_1, z_2)$ denote the product of nonzero complex numbers.

\begin{axdef}
	mul\_nonzero\_complex: \Cnz \cross \Cnz \fun \Cnz
\where
	mul\_nonzero\_complex = \{~ MulNonzeroComplex @ (z_1, z_2) \mapsto z' ~\}
\end{axdef}

\subsubsection{\zcmd{mulCnz}}

We introduce the usual notation $z' = z_1 \mulCnz z_2$ to denote the product.

\begin{zed}
	(\_ \mulCnz \_) == mul\_nonzero\_complex
\end{zed}

\subsection{Inversion}

\subsubsection{$InvNonzeroComplex$}

We can \textit{invert}
the nonzero complex number $z$ to get its \textit{inverse} or \textit{reciprocal} $z' = z^{-1}$.
Let the schema $InvNonzeroComplex$ denote this situation.

\begin{schema}{InvNonzeroComplex}
	z, z' : \Cnz
\where
	z \mulCnz z' = \oneC
\end{schema}

\subsubsection{$inv\_nonzero\_complex$}

Let $z' = inv\_nonzero\_complex(z)$ denote the inverse of $z$.

\begin{axdef}
	inv\_nonzero\_complex: \Cnz \fun \Cnz
\where
	inv\_nonzero\_complex = \{~ InvNonzeroComplex @ z \mapsto z' ~\}
\end{axdef}

\subsubsection{\zcmd{invCnz}}

We introduce the usual notation $z' = z \invCnz$ for the inverse.

\begin{zed}
	(\_ \invCnz) == inv\_nonzero\_complex
\end{zed}

\subsection{Division}

\subsubsection{$DivNonzeroComplex$}

We can \textit{divide} the complex number $z_1$ by the nonzero complex number $z_2$ 
to get their \textit{quotient} $z' = z_1 / z_2$.
Let the schema $DivNonzeroComplex$ denote this situation.

\begin{schema}{DivNonzeroComplex}
	z_1, z': \C \\
	z_2: \Cnz
\where
	z_1 = z' \mulCnz z_2
\end{schema}

\subsubsection{$div\_nonzero\_complex$}

Let $z' = div\_nonzero\_complex(z_1, z_2)$ denote $z_1$ divided by $z_2$.

\begin{axdef}
	div\_nonzero\_complex: \C \cross \Cnz \fun \C
\where
	div\_nonzero\_complex = \{~ DivNonzeroComplex @ (z_1, z_2) \mapsto z' ~\}
\end{axdef}

\subsubsection{\zcmd{divCnz}}

We introduce the usual notation $z_1 \divCnz z_2$ to denote division.

\begin{zed}
	(\_ \divCnz \_) == div\_nonzero\_complex
\end{zed}

\subsection{Power}

\subsubsection{$power\_complex\_nat$}

Let the function $power\_complex\_nat(z,n) = z^n$ denote the result of raising the complex number $z$
to the power $n$ where $n$ is a natural number.

\begin{axdef}
	power\_complex\_nat: \C \cross \nat \fun \C
\where
	\forall z: \C @ \\
	\t1	power\_complex\_nat(z, 0) = \oneC
\also
	\forall z: \C; n: \nat_1 @ \\
	\t1	power\_complex\_nat(z, n) = z \mulC power\_complex\_nat(z, n - 1)
\end{axdef}

\begin{remark}
The expression $0^0$ is problematic, but here we define it to be $1$ since this is the most convenient for
complex polynomials.

\begin{zed}
	power\_complex\_nat(\zeroC, 0) = \oneC
\end{zed}
\end{remark}

\subsubsection{\zcmd{powCN}}

We introduce the notation $z \powCN n = power\_complex\_nat(z, n)$.

\begin{zed}
	(\_ \powCN \_) == power\_complex\_nat
\end{zed}

\subsection{Norm}

\subsubsection{$NormComplex$}

The $norm$ of a complex number $z$ is a non-negative real number $r$
equal to the Euclidean length of its underlying pair of real numbers regarded
as a vector in the Euclidean plane.
Let the schema $NormComplex$ denote this situation.

\begin{schema}{NormComplex}
	Complex \\
	r : \R
\where
	r = \sqrtR(x \mulR x \addR y \mulR y)
\end{schema}

\subsubsection{$norm\_complex$}

Let $r = norm\_complex(z)$ be the norm of $z$.

\begin{axdef}
	norm\_complex: \C \fun \R
\where
	norm\_complex = \{~ NormComplex @ z \mapsto r ~\}
\end{axdef}

\subsubsection{\zcmd{normC}}

We introduce the notation $r = \normC(z)$ to denote the norm of $z$.

\begin{zed}
	\normC == norm\_complex
\end{zed}

\section{Some Important Functions}

We declare the usual exponential and trigonometric functions here.

\subsection{Exponential}

\subsubsection{$exp\_complex$}

Let $exp\_complex(z) = e^z$.

\begin{axdef}
	exp\_complex: \C \fun \C
\end{axdef}

\subsubsection{\zcmd{expC}}

We introduce the notation $\expC z = exp\_complex(z)$.

\begin{zed}
	\expC == exp\_complex
\end{zed}

\subsection{Logarithm}

\subsubsection{$log\_complex$}

Let $log\_complex(z) = \log z$.

\begin{axdef}
	log\_complex: \Cnz \fun \C
\end{axdef}

\subsubsection{\zcmd{logC}}

We introduce the notation $\logC z = log\_complex(z)$.

\begin{zed}
	\logC == log\_complex
\end{zed}

\subsection{Sine}

\subsubsection{$sin\_complex$}

Let $sin\_complex(z) = \sin z$.

\begin{axdef}
	sin\_complex: \C \fun \C
\end{axdef}

\subsubsection{\zcmd{sinC}}

We introduce the notation $\sinC z = sin\_complex(z)$.

\begin{zed}
	\sinC == sin\_complex
\end{zed}

\subsection{Cosine}

\subsubsection{$cos\_complex$}

Let $cos\_complex(z) = \cos z$.

\begin{axdef}
	cos\_complex: \C \fun \C
\end{axdef}

\subsubsection{\zcmd{cosC}}

We introduce the notation $\cosC z = cos\_complex(z)$.

\begin{zed}
	\cosC == cos\_complex
\end{zed}

\subsection{Tangent}

\subsubsection{$tan\_complex$}

Let $tan\_complex(z) = \tan z$.

\begin{axdef}
	tan\_complex: \C \pfun \C
\end{axdef}

\subsubsection{\zcmd{tanC}}

We introduce the notation $\tanC z = tan\_complex(z)$.

\begin{zed}
	\tanC == tan\_complex
\end{zed}

\printbibliography

\end{document}