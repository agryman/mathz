\documentclass[11pt, oneside]{article}

\usepackage{preamble}
\addbibresource{../../shared/references.bib}

\usepackage{sets}
\usepackage{topological-spaces}
\usepackage{groups}
\usepackage{real-numbers}

\usepackage{complex-numbers}

\title{Complex Numbers}
\author{Arthur Ryman, {\tt arthur.ryman@gmail.com}}
\date{\today}

\begin{document}

\maketitle

\begin{abstract}
This article contains Z Notation type declarations for the complex numbers, $\C$, and some related objects.
It has been type checked by \fuzz.
\end{abstract}

\section{Introduction}

The complex numbers, $\C$, are foundational to many mathematical objects such as vector spaces and manifolds,
but are not built-in to Z Notation.
This article provides type declarations for $\C$ and related objects so that they can be used and type checked in formal Z specifications.

No attempt has been made to provide complete, axiomatic definitions of all these objects since that would only be of use for proof checking.
Although proof checking is highly desirable, it is beyond the scope of this article.
The type declarations given here are intended to provide a basis for future axiomatization.

\section{Complex Numbers}

Z Notation does not predefine the set of complex numbers, so we define
them and operations on them here.

Although complex number operations are displayed using the same symbols as the 
analogous real number operations,
they are distinct mathematical objects.
This distinction is apparent to the \fuzz\ type-checker and should not cause confusion to the human reader
because the underlying types of the objects will, as a rule, be clear from the context.
Visually distinct symbols will be used in cases where confusion is possible.

\subsection{$COMPLEX$}

A complex number can be thought of a pair of real numbers.
However, it is not correct to view any pair of real numbers as a complex number.
Therefore to denote a complex number we map a pair of real numbers
into the free type $COMPLEX$.

\begin{zed}
	COMPLEX ::= complex \ldata \R \cross \R \rdata
\end{zed}

Here the function $complex$ is a constructor that constructs a
complex number from a pair of real numbers.

\subsection{\zcmd{C}}

We introduce the usual notation $\C = COMPLEX$ for the set of complex numbers.

\begin{zed}
\C == COMPLEX
\end{zed}

\subsection{$Complex$}

Given real numbers $x$ and $y$, we can construct the complex number
$z = complex(x,y)$.
The real numbers $x$ and $y$ are referred to as the \textit{real} and \textit{imaginary} parts of $z$.
It's useful to introduce the schema $Complex$ that relates the complex number $z$ to its
real and imaginary parts $x$ and $y$.

\begin{schema}{Complex}
	z : \C \\
	x, y : \R
\where
	z = complex(x, y)
\end{schema}

\subsection{$real\_complex$}

Let $real\_complex(z)$ denote the real part $x$ of $z$.

\begin{zed}
	real\_complex == \{~ Complex @ z \mapsto x ~\}
\end{zed}

\subsection{\zcmd{realC}}

We introduce the usual notation $x = \realC(z)$ for the real part of $z$.

\begin{zed}
	\realC == real\_complex
\end{zed}

\subsection{$imag\_complex$}

Let $imag\_complex(z)$ denote the imaginary part $y$ of $z$.

\begin{zed}
	imag\_complex == \{~ Complex @ z \mapsto y ~\}
\end{zed}

\subsection{\zcmd{imagC}}

We introduce the usual notation $y = \imagC(z)$ for the imaginary part of $z$.

\begin{zed}
	\imagC == imag\_complex
\end{zed}

\subsection{$AddComplex$}

We can \textit{add} the complex numbers $z_1$ and $z_2$ to give their sum $z'$.
Let the schema $AddComplex$ denote this situation.

\begin{schema}{AddComplex}
	Complex_1 \\
	Complex_2 \\
	Complex'
\where
	x' = x_1 \addR x_2 \\
	y' = y_1 \addR y_2
\end{schema}

\subsection{$add\_complex$}

Let $add\_complex(z_1, z_2)$ denote the result of adding $z_1$ and $z_2$.

\begin{zed}
	add\_complex == \{~ AddComplex @ (z_1, z_2) \mapsto z' ~\}
\end{zed}

\subsection{\zcmd{addC}}

We introduce the usual notation $z' = z_1 \addC z_2$ for addition in $\C$.

\begin{zed}
	(\_ \addC \_) == add\_complex
\end{zed}

\subsection{$zero\_complex$}

Let $zero\_complex$ denote the \textit{zero} of $\C$.

\begin{zed}
	zero\_complex == complex (\zeroR, \zeroR)
\end{zed}

\subsection{\zcmd{zeroC}}

We introduce the usual notation $\zeroC \in \C$ for the zero of $\C$.

\begin{zed}
	\zeroC == zero\_complex
\end{zed}

\subsection{$NegComplex$}

We can \textit{negate} the complex number $z$ to give its negative $z'$.
Let the schema $NegComplex$ denote this situation.

\begin{schema}{NegComplex}
	Complex \\
	Complex'
\where
	x' = \negR x
\also
	y' = \negR y
\end{schema}

\subsection{$neg\_complex(z)$}

Let $neg\_complex(z)$ denote the negative of $z$.

\begin{zed}
	neg\_complex == \{~ NegComplex @ z \mapsto z' ~\}
\end{zed}

\subsection{\zcmd{negC}}

We introduce the usual notation $z' = \negC z$ for the negative of $z$.
\begin{zed}
	\negC == neg\_complex
\end{zed}

\subsection{The Additive Abelian Group $\C$}

\begin{theorem}

The complex numbers $\C$ form an Abelian group under addition.

\begin{zed}
(\_ \addC \_) \in \abgroup \C
\also
\zeroC = identity\_element(\_ \addC \_)
\also
\negC = inverse\_operation(\_ \addC \_)
\end{zed}

\end{theorem}

\subsection{$SubComplex$}

We can \textit{subtract} the complex number $z_1$ from $z_2$
to give their difference $z'$.
Let the schema $SubComplex$ denote this situation.

\begin{schema}{SubComplex}
	Complex_1 \\
	Complex_2 \\
	Complex'
\where
	x' = x_1 \subR x_2
\also
	y' = y_1 \subR y_2
\end{schema}

\subsection{$sub\_complex$}

Let $sub\_complex(z_1, z_2)$ denote the difference of $z_1$ and $z_2$.

\begin{zed}
	sub\_complex == \{~ SubComplex @ (z_1, z_2) \mapsto z' ~\}
\end{zed}

\subsection{\zcmd{subC}}

We introduce the usual notation $z' = z_1 \subC z_2$ for subtraction in $\C$.

\begin{zed}
	(\_ \subC \_) == sub\_complex
\end{zed}

\subsection{$nonzero\_complex$}

Let $nonzero\_complex$ denote the set of nonzero complex numbers.

\begin{zed}
	nonzero\_complex == \C \setminus \{ \zeroC \}
\end{zed}

\subsection{\zcmd{Cnz}}

We introduce the usual notation $\Cnz \subseteq \C$ to denote the set of nonzero complex numbers,
also referred to as the \textit{punctured complex number plane}.

\begin{zed}
	\Cnz == nonzero\_complex
\end{zed}

\subsection{$MulComplex$}

We can \textit{multiply} the complex numbers $z_1$ times $z_2$ to give their product $z'$.
Let the schema $MulComplex$ denote this situation.

\begin{schema}{MulComplex}
	Complex_1 \\
	Complex_2 \\
	Complex'
\where
	x' = x_1 \mulR x_2 \subR y_1 \mulR y_2
\also
	y' = x_1 \mulR y_2 \addR y_1 \mulR x_2
\end{schema}

\subsection{$mul\_complex$}

Let $mul\_complex(z_1, z_2)$ denote $z_1$ multiplied by $z_2$.

\begin{zed}
	mul\_complex == \{~ MulComplex @ (z_1, z_2) \mapsto z' ~\}
\end{zed}

\subsection{\zcmd{mulC}}

We introduce the usual notation $z' = z_1 \mulC z_2$ for multiplication in $\C$.

\begin{zed}
	(\_ \mulC \_) == mul\_complex
\end{zed}

\subsection{$MulNonzeroComplex$}

We can restrict multiplication in $\C$ to $\Cnz$.
Let the schema $MulNonzeroComplex$ denote this situation.

\begin{schema}{MulNonzeroComplex}
	MulComplex \\
\where
	z_1 \in \Cnz
\also
	z_2 \in \Cnz
\end{schema}

\subsection{$mul\_nonzero\_complex$}

Let $mul\_nonzero\_complex(z_1, z_2)$ denote the product of nonzero complex numbers.

\begin{zed}
	mul\_nonzero\_complex == \{~ MulNonzeroComplex @ (z_1, z_2) \mapsto z' ~\}
\end{zed}

\subsection{\zcmd{mulCnz}}

We introduce the usual notation $z' = z_1 \mulCnz z_2$ to denote the product.

\begin{zed}
	(\_ \mulCnz \_) == mul\_nonzero\_complex
\end{zed}

\subsection{$one\_complex$}

Let $one\_complex$ denote the multiplicative unit in $\C$.

\begin{zed}
	one\_complex == complex(\oneR, \zeroR)
\end{zed}

\subsection{\zcmd{oneC}}

We introduce the usual notation $\oneC \in \C$ for the unit of $\C$.

\begin{zed}
	\oneC == one\_complex
\end{zed}

\subsection{$InvNonzeroComplex$}

We can \textit{invert}
the nonzero complex number $z$ to get its inverse or reciprocal $z'$.
Let the schema $InvNonzeroComplex$ denote this situation.

\begin{schema}{InvNonzeroComplex}
	z, z' : \Cnz
\where
	z \mulCnz z' = \oneC
\end{schema}

\subsection{$inv\_nonzero\_complex$}

Let $z' = inv\_nonzero\_complex(z)$ denote the inverse of $z$.

\begin{zed}
	inv\_nonzero\_complex == \{~ InvNonzeroComplex @ z \mapsto z' ~\}
\end{zed}

\subsection{\zcmd{invCnz}}

We introduce the usual notation $z' = z \invCnz$ for the inverse.

\begin{zed}
	(\_ \invCnz) == inv\_nonzero\_complex
\end{zed}

\subsection{$DivNonzeroComplex$}

We can \textit{divide} the nonzero complex numbers $z_1$ by $z_2$ to get their quotient $z'$.
Let the schema $DivNonzeroComplex$ denote this situation.

\begin{schema}{DivNonzeroComplex}
	z_1, z_2, z' : \Cnz
\where
	z_1 = z' \mulCnz z_2
\end{schema}

\subsection{$div\_nonzero\_complex$}

Let $z' = div\_nonzero\_complex(z_1, z_2)$ denote $z_1$ divided by $z_2$.

\begin{zed}
	div\_nonzero\_complex == \{~ DivNonzeroComplex @ (z_1, z_2) \mapsto z' ~\}
\end{zed}

\subsection{\zcmd{divCnz}}

We introduce the usual notation $z_1 \divCnz z_2$ to denote division.

\begin{zed}
	(\_ \divCnz \_) == div\_nonzero\_complex
\end{zed}

\subsection{The Multiplicative Abelian Group $\Cnz$}

\begin{theorem}
The nonzero complex numbers $\Cnz$ form an Abelian group under multiplication.

\begin{zed}
(\_ \mulCnz \_) \in \abgroup \Cnz
\also
\oneC = identity\_element(\_ \mulCnz \_)
\also
(\_ \invCnz) = inverse\_operation(\_ \mulCnz \_)
\end{zed}

\end{theorem}

\subsection{$NormComplex$}

The $norm$ of a complex number $z$ is a non-negative real number $r$
equal to the Euclidean length of its underlying pair of real numbers regarded
as a vector in the Euclidean plane.
Let the schema $NormComplex$ denote this situation.

\begin{schema}{NormComplex}
	Complex \\
	r : \R
\where
	r = \sqrtR(x \mulR x \addR y \mulR y)
\end{schema}

\subsection{$norm\_complex$}

Let $r = norm\_complex(z)$ be the norm of $z$.

\begin{zed}
	norm\_complex == \{~ NormComplex @ z \mapsto r ~\}
\end{zed}

\subsection{\zcmd{normC}}

We introduce the usual notation $r = \normC(z)$ to denote the norm of $z$.

\begin{zed}
	\normC == norm\_complex
\end{zed}

\printbibliography

\end{document}