\documentclass{amsart}

\usepackage{mathz_core}
\usepackage{mathz_sets}
\usepackage{mathz_topological_spaces}
\usepackage{mathz_real_numbers}
\usepackage{mathz_formal_specifications}

\usepackage{mathz_preamble}

\addbibresource{mathz_references.bib}

\begin{document}

\title{The Role of Formalization and \ZN\ }
\author{Arthur Ryman}
\email[Arthur Ryman]{arthur.ryman@gmail.com}
\date{\today}

\begin{abstract}
The notion of {\it type} was introduced into set theory by Russel in an effort to establish a firm foundation for mathematics.
Types are now a standard feature of many computer programming languages.
\ZN\  is a formal language based on typed set theory for specifying computer programs.
Since \ZN\  is a formal language, a Z specification can itself be automatically checked for type errors.
Writing valid Z specifications requires that the author explicitly define all terms.
This encourages a style in which complex terms are gradually built up from simpler, previously defined terms.
This article proposes to use \ZN\  to formally specify mathematical structures, even if there is no intention of using
them in computer programs.
The hoped for benefits of doing so are that the exercise of formalization will lead the author to a better understanding of the subject
matter, resulting in a clearer, higher quality, result.
\end{abstract}

\maketitle

\tableofcontents

\section{Types}

Computer languages can be divided into two major groups, namely those that are strongly typed and those that are weakly typed.
Strongly-typed languages, such as Java, have complex type systems and require that all expressions and variable declarations 
have well-defined types.
Weakly-typed languages, such as Javascript, have simpler type systems and allow greater flexibility in expressions and the use of variables.
Although no compiler can possibly tell you if your program is correct, a compiler for a strongly-typed language can at least detect type errors.
Correcting these errors at compile-time is often quicker than doing so at execution-time.

There are many similarities between mathematics and programming. 
Both require the definition and use of named objects that are built up from simpler objects.
The concept of type applies to both.
Bertrand Russell discovered typed set theory in his efforts to solve some problems in the foundations of mathematics 
long before computers were a reality.
Theoretical computer science now makes extensive use of types.
Mathematics can be used to specify the requirements for programs.
Specifications act as a bridge between mathematics and programs.
\ZN\  is a strongly-typed formal specification language.
However, \ZN\  can be used to formalize mathematics even if the goal is not to program a computer.

\section{Formalization}

\ZN\  can be used to precisely define mathematical objects and a Z specification can be type checked by a program.
I have found that the exercise of expressing mathematics using \ZN\  is a great way make sure that concepts are well-defined.
This is especially helpful, although somewhat labour-intensive, when dealing with complex mathematical structures.
I propose to formalize all definitions using \ZN\  and to validate
the document using the \fuzz\ type checker.

I believe that the effort of precisely defining every needed concept pays off. 
Consider how C\'{e}dric Villani \cite{villani-hbm} described the mathematical writing style of Alexander Grothendieck:
\begin{displayquote}
Let me quote a famous mathematical text by Alexander Grothendieck, 
one of the most famous mathematicians of all times, who wrote long writings.
In one of them he talks about the metaphor \textit{de la noix}.
You know the parable, the metaphor, of the nut
to explain the difference between his style and 
the style of his fellow mathematician Jean-Pierre Serre.
Both of them working in the same area of mathematics, but very different styles. 
Grothendieck said, 
\textquote{Imagine that we have a nut to open.
The style of Serre would be take a
hammer and, Bang!, smash the nut.
My style would be to take the nut and put
it in a sea of acid so that it would be
dissolved very slowly, the crust of the nut, without noticing anything.}
And yes, experts tell us that the style of Grothendieck is like everything is very
incremental from one step to another to
another by very tiny steps so that you
have the impression that nothing occurs
and we are really making no progress, and
at the end it's proven, there it is, the big theorem is done.
\end{displayquote}
Similarly, Pierre Deligne~\cite{artin-ag1} said:
\begin{displayquote}
From Grothendieck I have also learned not to take glory in the difficulty of a proof: 
difficulty means we have not understood.
The ideal is to be able to paint a landscape in which the proof is obvious.
I admire how often he succeeded in reaching this ideal.
\end{displayquote}
The message is clear: a slow, methodical build-up of concepts is a good thing in mathematics.

\section{Given Sets and Signatures}

Although the integers are built into \ZN, the real numbers are not.
In principle, one could first build up the rational numbers from the integers,
and then the real numbers from the rational numbers, but that would take a lot of time and not really
accomplish much.
I'll assume that the real numbers are sufficiently well-understood and do not need a complete formalization.
Instead, I'll define real numbers as a given set and then declare the types and signatures of the usual constants and operations of real arithmetic.

\section{Names and Symbols}

Integers and reals are distinct types in the sense of typed set theory.
But distinct types are disjoint sets.
This implies that the $0$ element of the integers is not the same object as the $0$ element of the reals.
In fact, they aren't even comparable within a Z specification since doing so is a type error and the 
specification would therefore not be valid.
The name \texttt{0} is built-in to \ZN\  as the name of the $0$ element of the integers.
A Z specification must therefore introduce a new name to refer to the $0$ element of the reals.
However, working mathematicians regard the integers as a subset of the reals and therefore have no need to deal with duplicate names.
As Henri Poincar\'{e} \cite{poincare-fm} said:
\begin{displayquote}
I think I have already said somewhere that mathematics is the art of giving the same name to different things. 
It is enough that these things, though differing in matter, should be similar in form, to permit of their being, so to speak, run in the same mould.\end{displayquote}

How then can we make a Z specification valid while still honouring normal mathematical practice?
The way out of this difficulty is that if a mathematician would normally regard two distinct objects as the same
then we give then distinct \ZN\  names, but display them using the same typographic symbol.
To paraphrase Poincar\'{e}:
\begin{quote}
Formalizing mathematics is the art of giving different names, but the same symbol, to different things. 
It is enough that these things, though differing in type, should be similar in typography, 
to permit of their being, so to speak, run in the same mould.
\end{quote}

For example, Mike Spivey \cite{spivey-fm} describes three typographic symbols defined in the \fuzz\ package 
that are each used to display two distinct objects:
\begin{displayquote}
A few symbols have two names, reflecting two different uses for the
symbol in Z:
\begin{itemize}
\item The symbol $\semi$ is called 
\cmd{semi} when it is used as an operation on schemas, and 
\cmd{comp} when it is used for composition of relations.
\item The symbol $\hide$ is called
\cmd{hide} as the hiding operator of the schema calculus, and 
\cmd{setminus} as the set difference operator. 
\item The symbol $\project$ is called
\cmd{project} as the schema projection operator, and 
\cmd{filter} as the filtering operator on sequences.
\end{itemize}
Although the printed appearance of
each of these pairs of symbols is the same, the type checker
recognizes each member of the pair only in the appropriate
\hbox{context}.
\end{displayquote}
This approach keeps both the Z type checker and the mathematician reader happy.
For example, in my formalization of the reals:
\begin{itemize} 
\item The symbol $0$ is called 
\texttt{0} when it is used as an integer, and 
\cmd{zeroR} as a real number.
\item The symbol $+$ is called 
\texttt{+} as the addition operator on the integers, and
\cmd{addR} as the addition operator on the reals.
\end{itemize}
The mathematician reader will always be able to infer the actual type of an object from its context.

\section{Theorems, Examples, and Remarks}

In mathematical writing it is good practice to follow a definition with some examples.
When an example is given, there is a proof obligation associated with it,
namely that whatever the author is asserting about the example is in fact true.
Similarly, when the author makes a remark or states a theorem, there is also a proof obligation.
In each of these cases, the author makes a statement and asserts it to be true, but may not offer any proof.

\ZN\ allows the author to add constraints to a specification.
However, if the constraint is a logical consequence of the preceding definitions then it adds nothing to the specification.
Therefore, we can freely add examples, remarks, and theorems to a specification without changing its meaning.
These statements should be placed in a \cmd{zed} box that is enclosed in an appropriate \LaTeX\ theorem-like environment.

For example, the following statement is true and therefore does not change the meaning of
the specification but will be type-checked by \fuzz, .

\begin{example}

\begin{zed}
	1 + 1 = 2
\end{zed}
	
\end{example}

The benefit of placing theorem-like statements in \cmd{zed} boxes is that the \fuzz\ type checker will check their type-correctness.

\subsection{Generic Theorem-Like Statements}

It will often be the case that a theorem-like statement will involve generic constructs.
Although \ZN\ provides a mechanism for introducing formal generic parameters into the definition of
generic constants and schemas, it does not provide such a mechanism for constraints.
We can work around this limitation by declaring a set of global given sets which are understood to be used in theorem-like
statements where they are taken to represent arbitrary sets.
The article on \href{topological-spaces.pdf}{topological spaces} 
introduces the global given sets $\setX$, $\setY$, and $\setZ$ for this purpose.

Testing hyperlinks to another PDF document:

\begin{verbatim}
\url{https://agryman.github.io/mathz/complex-numbers.pdf}
\end{verbatim}

\url{https://agryman.github.io/mathz/complex-numbers.pdf}

\begin{verbatim}
\url{complex-numbers.pdf}
\end{verbatim}

\url{complex-numbers.pdf}

\begin{verbatim}
\url{./complex-numbers.pdf}
\end{verbatim}

\url{./complex-numbers.pdf}


For example, we can remark that every element $x$ of an arbitrary set $\setX$ is a member of that set.

\begin{remark}

\begin{zed}
	\forall x: \setX @ x \in \setX
\end{zed}

\end{remark}

\section{Simulating Standard Mathematical Notation}

The expression $(a,b)$ usually denotes an ordered pair in mathematical writing.
Sometimes an author asks us to interpret $(a,b)$ as an open interval of the real number line.
In this case we recognize that the symbols used in the expression $(a,b)$ do not have their usual meanings
and we interpret them accordingly.
The notation $(a,b)$ for denoting an open interval of the real number line is
arguably very compact, convenient, and easy to understand. 
In contrast, the \ZN\  function application $\intervalR(a,b)$, defined in the article on real numbers,
may strike the reader as somewhat verbose and clumsy, albeit precise.

%\subsection{Prefix, Infix, and Postfix Operators in \fuzz}
\subsection{Prefix, Infix, and Postfix Operators in fuzz}

There is a way to achieve some of the compactness of standard mathematical notation in Z while
preserving its precision and type-correctness.
The approach is to take advantage of the \fuzz\ type checker's ability to define infix and postfix 
operator symbols. 
In the grammar used by \fuzz, postfix operators have higher precedence than prefix operators,
and prefix operators have higher precedence than infix operators.
Infix operators are assigned numerical priorities of 1 through 6 with higher priorities taking precedence 
over lower ones.
We can use these relative precedences to define operators that allow us to construct expressions 
that resemble usual mathematical writing.

%\subsection{\zcmd{lowerBound}, \zcmd{upperBound}, and \zcmd{intersect}}
\subsection{Defining Intervals with Operators}

First, we use \LaTeX\ bold styling to provide a visual cue that helps the reader
distinguish the interval $\lowerBound a \intersect b \upperBound$ from the
usual ordered pair $(a, b)$.
Next, we interpret the symbols as prefix, infix, and postfix operators.
The symbol $\lowerBound$ is a prefix operator that maps $a$ to $\lowerBound a$, 
the open interval bounded below by $a$.
Similarly, $\upperBound$ is a postfix operator that maps $b$ to $b \upperBound$, 
the open interval bounded above by $b$.
Finally, we interpret the symbol $\intersect$ as an infix operator that forms the intersection of the two intervals.

\begin{axdef}
	\lowerBound: \R \fun \power \R \\
	\_ \upperBound: \R \fun \power \R \\
	\_ \intersect \_ : \power \R \cross \power \R \fun \power \R
\where
	\forall a: \R @ \lowerBound a = \{~ x: \R | a \ltR x ~\}
\also
	\forall b: \R @ b \upperBound = \{~ x: \R | x \ltR b ~\}
\also
	\forall a, b: \R @ \lowerBound a \intersect b \upperBound = \{~ x: \R | a \ltR x \ltR b ~\}
\end{axdef}

This design is a step in the right direction, but it doesn't prevent the writer from abusing the notation.
For example, the following paragraph looks odd but makes perfect sense to \fuzz.

\begin{zed}
	\forall a, b: \R @ \lowerBound a \intersect b \upperBound = b \upperBound \intersect \lowerBound a 
\end{zed}

When we are asked to interpret $(a,b)$ as an interval, we are, in a sense, being asked to parse a sentence
in a new mathematical mini-language that defines intervals.
If we want to enforce the usual syntax for intervals,
we can introduce new types to represent fragments of the expression so that the operators can only be applied to
fragments in some prescribed order.

To illustrate this point, let's expand the example to include closed intervals $[a,b]$ as well as semi-closed intervals
$(a,b]$ and $[a,b)$.
Regard $(a$ and $[a$ as open and closed right half lines, and $b)$ and $b]$ as open and closed left half lines.
Only allow a right half line to be combined with a left half line.

\subsection{Right Half Lines}

Let \hypertarget{RightHalfLine}{$RightHalfLine$} denote the type of syntactic fragments that define open and closed right half-lines.

\begin{zed}
	RightHalfLine ::= \openLowerBound \ldata \R \rdata | \closedLowerBound \ldata \R \rdata
\end{zed}

\begin{example}

\begin{zed}
	\openLowerBound \zeroR \in RightHalfLine
\also
	\closedLowerBound \oneR \in RightHalfLine
\end{zed}

\end{example}

Let $rightSet(R)$ denote the set of real numbers that the right half-line syntactic fragment $R$ represents.

\begin{axdef}
	rightSet: RightHalfLine \fun \power \R
\where
	\forall a : \R @ rightSet(\openLowerBound a) = \{~ x: \R | a \ltR x ~\}
\also
	\forall a : \R @ rightSet(\closedLowerBound a) = \{~ x: \R | a \leR x ~\}
\end{axdef}

For example

\begin{zed}
	\zeroR \notin rightSet(\openLowerBound \zeroR)
\also
	\oneR \in rightSet(\closedLowerBound \oneR)
\end{zed}

\subsection{Left Half Lines}

Let $LeftHalfLine$ denote the type of syntactic fragments that define open and closed left half-lines.

\begin{zed}
	LeftHalfLine ::= (\_ \openUpperBound)  \ldata \R \rdata | (\_ \closedUpperBound)  \ldata \R \rdata
\end{zed}

For example

\begin{zed}
	\zeroR \openUpperBound \in LeftHalfLine
\also
	\oneR \closedUpperBound \in LeftHalfLine
\end{zed}

Let $leftSet(L)$ denote the set of real numbers that the left half-line syntactic fragment $L$ represents.

\begin{axdef}
	leftSet: LeftHalfLine \fun \power \R
\where
	\forall b : \R @ leftSet(b \openUpperBound) = \{~ x: \R | x \ltR b ~\}
\also
	\forall b : \R @ leftSet(b \closedUpperBound) = \{~ x: \R | x \leR b ~\}
\end{axdef}

For example

\begin{zed}
	\zeroR \notin leftSet(\zeroR \openUpperBound)
\also
	\oneR \in leftSet(\oneR \closedUpperBound)
\end{zed}

\subsection{Intersection of Half Lines}

Let $R \intersectRightLeft L$ denote the set of real numbers in the intersection of the sets represented by
the half-line syntactic fragments $R$ and $L$.

\begin{axdef}
	\_ \intersectRightLeft \_ : RightHalfLine \cross LeftHalfLine \fun \power \R
\where
	\forall R: RightHalfLine; L: LeftHalfLine @ \\
	\t1	R \intersectRightLeft L = rightSet~R \cap leftSet~L
\end{axdef}

For example

\begin{zed}
	\zeroR \in \closedLowerBound \zeroR \intersectRightLeft \oneR \openUpperBound
\end{zed}

\subsection{Improvements}

After reading the above, my reaction is that it would be clearer to first define the relevant objects using meaningful names
and then define the symbols as abbreviations for them.

TODO:
\begin{itemize}
\item Fix up the above discussion.
\item Use the example environment for examples.
\item Improve the Summary. It should summarize the whole article, not just how to simulate mini-languages.
\end{itemize}


\subsection{Summary}

A lot of standard mathematical notation can be written directly using the built-in capabilities of \ZN\  and \fuzz.
However, mathematics contains many specialized notations including those for intervals, probability, and quantum mechanics (Dirac bra-ket notation).
One can view these specialized notations as mathematical mini-languages.
As demonstrated above for the case of intervals, it is possible to simulate these notations in Z by introducing new syntactic types along with infix and postfix operators that construct, combine, and reduce fragments of this syntax.
The general pattern is for the opening and closing symbols to correspond to prefix and postfix operators that
construct fragments, and for internal symbols to correspond to infix operators that combine and then finally 
reduce fragments.
Reduction occurs when the final infix operator is applied and maps the completed mini-language sentence to a 
some usual mathematical type.

\printbibliography

\end{document}  